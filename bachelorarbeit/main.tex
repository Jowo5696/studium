\documentclass[a4paper,10pt]{article}

%{{{ packages
\usepackage{blindtext}
\usepackage{lipsum} % lorem ipsum
\usepackage[english]{babel} % correct typesetting
\usepackage{csquotes} % correct quotation
\usepackage[tiny]{titlesec} % section title size
\usepackage{index}
%\usepackage[onehalfspacing]{setspace} % line spacing
\usepackage[headings]{fullpage} % makes document wider
\usepackage{pdfpages} % \includepdf
\usepackage{fancyhdr} % fancy headers
\usepackage{authblk} % author formatting
\usepackage{hyperref} % "clickable" references in text
\hypersetup{colorlinks=true,allcolors=blue}
\usepackage[backend=biber,style=phys]{biblatex}
\addbibresource{refs.bib}

\usepackage{booktabs} % toprule,midrule,bottomrule
\usepackage{multirow} % multiple rows in tabular
\usepackage{multicol} % multiple columns in tabular
\usepackage{longtable}
\usepackage{graphicx}
\usepackage{svg} % \includesvg{}
\usepackage{tikz}
\usepackage[european,siunitx]{circuitikz}
\usepackage{import} % can import files from other directories
\usepackage{pgfplots}
\usepackage{gnuplottex}
\usepackage{wrapfig} % let text wrap around figure

\usepackage{amsmath,amsfonts,amssymb}
\usepackage{tensor} % for right index order 
\usepackage{dsfont} % double stroke font
\usepackage{cancel} % to cancel in fraction
\usepackage{bm} % bold math font (if error is produced use \bm{{}})
\usepackage{mathtools}
\usepackage[ISO]{diffcoeff} % differentiation
\usepackage[locale=DE]{siunitx}
\usepackage[official]{eurosym} % euro symbol
\usepackage{mathrsfs} % calligraphy
\usepackage{physics}
\usepackage[a]{esvect} % vector arrows
\usepackage{bigints} % big integrals
%\usepackage[frak=esstix]{mathalpha} % disable if LaTeX uses too many alphabets
\usepackage[ruled,vlined,linesnumbered]{algorithm2e}
\usepackage{listings}

\usepackage{hyphsubst}
\usepackage[]{caption} % [figurename=, tablename=]
\usepackage{xr} % crossreferencing between documents \externaldocument{}
\usepackage{enumitem} % for enumerate environment
\usepackage{lineno}
\usepackage{makecell}
\usepackage{xcolor}
\usepackage{color}

\allowdisplaybreaks % allows equations to be broken (e.g. by multicols)

\usetikzlibrary{arrows}
\pgfplotsset{compat=1.15}

\newcommand{\td}{\,\text{d}}
%\newcommand{\RN}[1]{\uppercase\expandafter{\romannumeral#1}}
\newcommand{\zz}{\mathrm{Z\kern-.3em\raise-0.5ex\hbox{Z} }}
\newcommand{\id}{\mathds{1}}

\newcommand\inlineeqno{\stepcounter{equation}\ {(\theequation)}}
\newcommand\inlineeqnoa{(\theequation.\text{a})}
\newcommand\inlineeqnob{(\theequation.\text{b})}
\newcommand\inlineeqnoc{(\theequation.\text{c})}

\newcommand\inlineeqnowo{\stepcounter{equation}\ {(\theequation)}}
\newcommand\inlineeqnowoa{\theequation.\text{a}}
\newcommand\inlineeqnowob{\theequation.\text{b}}
\newcommand\inlineeqnowoc{\theequation.\text{c}}

\renewcommand{\refname}{Source}
\renewcommand{\sfdefault}{phv}

\newcommand{\todo}[1]{\textcolor{red}{TODO: #1}}

% for multicols figures and then use \captionof{figure}{}
\iffalse
\newenvironment{Figure}
  {\par\medskip\noindent\minipage{\linewidth}}
  {\endminipage\par\medskip}
\fi

\sloppy % block text

\numberwithin{equation}{section}

\titleformat{\subsection}{}{\thesubsection}{1em}{\itshape}
\titleformat{\subsubsection}{}{\thesubsubsection}{1em}{\itshape}

%}}}
\usepackage{newtxtext}
\usepackage{newtxmath}

\begin{document}

%{{{ Titelseite

\begin{titlepage}
  %\title{\scalebox{.8}[3]{Thermodynamics of Black Holes}}
  \title{Thermodynamics of Black Holes}
  \author{Jonas Wortmann}
  \affil{\vspace{2cm}Bachelorarbeit in Physik angefertigt im\\\vspace{1cm}Physikalischen Institut\\\vspace{1cm}vorgelegt der Mathematisch-Naturwissenschaftlichen Fakultät der\\\vspace{1cm}Rheinischen Friedrich-Wilhelms-Universität Bonn\\\vspace{2cm}}
\end{titlepage}

\maketitle
\pagenumbering{gobble}

%\renewcommand\abstractname{Abstract}
%\abstract{\noindent}

\clearpage

\vspace*{\fill}
\noindent Ich versichere, dass ich diese Arbeit selbstständig verfasst und keine anderen als die angegebenen
Quellen und Hilfsmittel benutzt sowie die Zitate kenntlich gemacht habe.\\\\\\
\begin{minipage}{0.45\textwidth}
  Bonn,\hrule\par\vspace{1mm}
  Datum
\end{minipage}\hfill
\begin{minipage}{0.45\textwidth}
  \phantom{Bonn,}\hrule\par\vspace{1mm}
  Unterschift
\end{minipage}\\\\
\begin{enumerate}[label=\arabic*]
  \item Gutachter: Priv.-Doz.\ Dr.\ Stefan Förste
  \item GutachterIn:
\end{enumerate}

%}}}

\clearpage

%{{{ Inhaltsverzeichnis

\renewcommand*\contentsname{Contents}
\tableofcontents

%}}}

\clearpage

\pagestyle{fancy}

\fancyhead[R]{\leftmark}
%\fancyhead[R]{\leftmark\\\rightmark}
\fancyhead[L]{Thermodynamics of Black Holes}
\fancyfoot[C]{}
\fancyfoot[R]{\thepage}

\pagenumbering{arabic}

%{{{

%\begin{multicols*}{2}

%{{{ Introduction
\section{Introduction}
This work shall present a concise and clear derivation of all four laws of black hole thermodynamcis.
%}}}

%{{{ Theoretical Background 
\section{Theoretical Background}

%{{{ Einstein Field Equation and Black Holes
\subsection{Einstein Field Equation and Black Holes}
Einstein's field equations relates the geometry of spacetime to the energy and matter of the bodies which move through it
\begin{align} 
  R\indices{_\mu _\nu }-\dfrac{1}{2}g\indices{_\mu _\nu }R +\Lambda g\indices{_\mu _\nu }&= \kappa T\indices{_\mu _\nu } \label{eq:efe}
\,.\end{align} 
where 
\begin{align} 
  R\indices{^\rho _\mu _\sigma _\nu } &= \partial_\sigma \Gamma \indices*{_\nu _\mu ^\rho} -\partial_\nu \Gamma _{\sigma \mu }^\rho +\Gamma \indices*{_\sigma _\lambda ^\rho} \Gamma _{\nu \mu }^\lambda -\Gamma _{\nu \lambda }^\rho \Gamma _{\sigma \mu }^\lambda & R\indices{^\lambda _\mu _\lambda _\nu } &= R\indices{_\mu _\nu } & g\indices{^\mu ^\nu }R\indices{_\mu _\nu }=R\indices{^\mu _\mu } &= R \label{eq:riemann tensor}
\,.\end{align} 
is the Riemann tensor, the Ricci tensor and the Ricci scalar; $g_{\mu \nu }$ is the metric, $T_{\mu \nu }$ the energy-momentum tensor, $\Lambda $ the cosmological constant and $\kappa =\tfrac{8\pi G}{c^4}$ (SI units).
From here on we will set $c=1$ such that $\kappa =8\pi G$\,.
It is an equation of motion for the metric tensor, which dictates the path test particles take by the geodesics equation
\begin{align} 
   \dfrac{\text{D}}{\td \lambda }\diff[]{x^\mu }{\lambda }\equiv\diff[]{x^\nu }{\lambda }\nabla _\nu \diff[]{x^\mu }{\lambda }=\diff[2]{x^\mu }{\lambda }+\Gamma _{\rho \sigma}^\mu \diff[]{x^\rho }{\lambda }\diff[]{x^\sigma }{\lambda }=0\label{eq:geodesics equation}
\,.\end{align} 
where $\lambda $ is the affine parameter, $\Gamma $ is the Christoffel symbol (i.e.\ the Levi-Civita connection is implied) and $\nabla $ the covariant derivative.
If a geodesic is not affinely parameterized it will in general not be zero.
For example we can find another tangent vector which is proportional to the original tangent vector such that
\begin{align} 
  t^\mu \equiv\diff[]{\tilde{x}^\mu }{\lambda }=f(\lambda )\diff[]{x^\mu }{\lambda }\equiv f\diff[]{x^\mu }{\lambda } \label{eq:tangent vector proportionality}
\,.\end{align} 
which will yield
\begin{align} 
  0=\diff[]{x^\nu }{\lambda }\nabla _\nu \diff[]{x^\mu }{\lambda } &= \dfrac{1}{f}t^\nu \nabla _\nu \left(\dfrac{1}{f}t^\mu \right) = \dfrac{1}{f}t^\nu \left(\partial_\nu \dfrac{1}{f}\right)t^\mu +\dfrac{1}{f^2}t^\nu \left(\nabla _\nu t^\mu \right) = \dfrac{1}{f}t^\nu \left(-\dfrac{1}{f^2}\partial_\nu f\right)t^\mu +\dfrac{1}{f^2}t^\nu \left(\nabla _\nu t^\mu \right)\\
                                                                   &= -\dfrac{1}{f^2}t^\nu \left(\partial_\nu \ln f\right)t^\mu +\dfrac{1}{f^2}t^\nu \left(\nabla _\nu t^\mu \right)
\,.\end{align} 
then
\begin{align} 
  t^\nu \nabla _\nu t^\mu  &= t^\nu \left(\partial_\nu \ln f\right)t^\mu \equiv \kappa t^\mu \label{eq:not affine geodesics}
\,.\end{align} 
which is the geodesic equation for non-affinely parameterized curves \cite{Townsend1997}\footnote{Do not interchange $\kappa $ from the non-affine parametrization with $\kappa $ as the Einstein-constant.}.
The constant can also be expressed as \cite{Townsend1997} \todo{could show}
\begin{align} 
  \kappa =\sqrt{-\dfrac{1}{2}\left(\nabla ^\mu t^\nu \right)\left(\nabla _\mu t_\nu \right)}
\,.\end{align} 

Since we will be dealing with the vacuum space around a gravitating mass, we are only interested in the the solutions of the metric which fulfill the equation for $T_{\mu \nu }=\Lambda =0$\,, s.t.\
\begin{align} 
  R_{\mu \nu }-\dfrac{1}{2}g_{\mu \nu }R = R_{\mu \nu }-\dfrac{1}{2}g_{\mu \nu }g^{\rho \sigma }R_{\rho \sigma } = 0\quad \Leftrightarrow \quad R_{\mu \nu } = 0 \label{eq:efe vacuum}
\,.\end{align} 
The equation (\ref{eq:efe vacuum}) is the Einstein field equation in vacuum, restricting the possible solutions of the metric to be Ricci flat.
%}}}

%{{{ Schwarzschild
\subsection*{Schwarzschild}
In \cite{Schwarzschild1916} an analytic solution to (\ref{eq:efe vacuum}) found by Schwarzschild is presented, which is given by\footnote{The form is taken from \cite{Carroll}. A derivation is also given in the book. This holds also for the other metrics presented.}
\begin{align} 
  \td s^2 = g_{\mu \nu }\td x^\mu \td x^\nu  &= -\left(1-\dfrac{2GM}{r}\right)\td t^2+\left(1-\dfrac{2GM}{r}\right)^{-1}\td r^2+r^2\td \Omega ^2 \label{eq:schwarzschild metric}
\,.\end{align} 
where $c=1$\,, $\td \Omega ^2=\td \theta ^2+\sin ^2\theta \td \phi ^2$ and $\left\{t,r,\theta ,\phi \right\}$ are the spherical coordinates.
Upon inspecting the metric we can notice two values of $r$ for which components of $\td s^2$ diverge.
The $r=r_S$ divergence is a mere coordinate dependent phenomenon.
In SI units we will call it the Schwarzschild radius, which is given by
\begin{align} 
  r_S &= \dfrac{2GM}{c^2} \label{eq:schwarzschild radius}
\,.\end{align} 
where $M$ is the mass of the gravitating body, defined by the weak field limit.
Although this singularity can be lifted, it is still of physical significance\footnote{As opposed to be thought by Schwarzschild himself, who regarded the singularity as an artifact of the analytical solution \cite{Schwarzschild1916}.}.
The singularity at $r=0$ is however not liftable, since $R\indices{^\mu ^\nu ^\rho ^\sigma }R\indices{_\mu _\nu _\rho _\sigma }=\tfrac{48G^2M^2}{r^6}$\,, which is a coordinate independent scalar, has a singularity at that point.

Now, we will call objects for which the Schwarzschild metric holds even at $r=2GM\equiv r_S$ \enquote{black holes} (they also exist in other spacetimes, as we will see).
Black holes exhibit some very interesting features, some of which we will study throughout this work.

Let us now investigate the surface $S=r-r_S$ by studying the behaviour of light cones.
Imagine a null ray closing in on $r_S$ for $\td \Omega =0$
\begin{align} 
  \td \tilde{s}^2 = 0 &= -\left(1-\dfrac{r_S}{r}\right)\td t^2+\left(1-\dfrac{r_S}{r}\right)^{-1}\td r^2\\
  \diff[]{r}{t} &= 1-\dfrac{r_S}{r}
\,.\end{align} 
For $r\rightarrow r_S$ 
\begin{align} 
  \lim_{r\rightarrow r_S}\diff[]{r}{t}=0\quad\Leftrightarrow \quad \lim_{r\rightarrow r_S}\diff[]{t}{r}=\infty
\,.\end{align} 
The light ray will cease to make progress towards the surface at $r_S$; at least in this coordinate system.
Let us check the normal vector to this surface \cite{Townsend1997}.
A vector $l$ is normal to a surface $S$ if
\begin{align} 
  l^\mu =\partial^\mu S=g^{\mu \nu }\partial_\nu S
\,.\end{align} 
Then
\begin{align} 
  l^2\equiv l^\mu l_\mu  &= l^\mu g_{\mu \nu }l^\nu =g^{\mu \rho }\partial_\rho Sg_{\mu \nu }g^{\nu \sigma }\partial_\sigma S=g^{\mu \rho }\partial_\rho S\partial_\mu S\\
               &= g^{t t}(\partial_tS)^2+g^{r r}(\partial_rS)^2+g^{\theta \theta }(\partial_\theta S)^2+g^{\phi \phi }(\partial_\phi S)^2\\
               &= \left(1-\dfrac{r_S}{r}\right)(1)^2
\,.\end{align} 
where
\begin{align} 
  \left. l^\mu l_\mu \right|_{r=r_S}=0
\,.\end{align} 
Therefore $S|_{r=r_S}$ is indeed a null hypersurface.
For $r<r_S$ the vector $l^2<0$ is becoming timelike.
Thus, for any light cone crossing this hypersurface, all future directed paths will point towards the center of the black hole, since the normal vector is neccessarily inside the light cone.
This is because the normal vector is also the greatest distance to traverse towards the center (i.e.\ an upper bound), thus any other vector, whose norm is less, will also be timelike.
In general, distances become timelike and one must move towards the center.

The same line of argumentation can be done with the Killing vector $K=\partial _t$\,, since
\begin{align} 
  \left.K^2\right|_{r=r_S}=K^\mu K_\mu =K^\mu g_{\mu \nu }K^\nu =g_{t t}=\dfrac{r_S}{r_S}-1=0
\,.\end{align} 
which is also null on the surface and spacelike $\left(K^2>0\right)$ within.
The time-translation Killing vector outside the surface is in general timelike $\left(K^2<0\right)$ which means, that an observers worldline can lie along the Killing vector.
The worldline of $K=\partial _t$ is one where all coordinates apart from $t$ remain unchanged.
If now $K$ is becoming spacelike, then $K$ cannot describe an observers worldline anymore; an observer cannot have constant spatial coordinates within the region behind the surface.
Assume that such an observer would go through the surface by decreasing the radius $r$ to the center.
If she would like to turn around, she has to reverse her direction, i.e.\ traverse a point where $r$ would be a constant for a certain amount of time and then increase again.
However, this is not allowed because the point of constant radius is not part of her worldline.
\todo{but could she not increase $\theta $ and $\phi $ so much as to cancle the negative sign of $t$?}
Therefore the radial coordinate cannot change sign, once it is determined from outside the surface; the observer neccessarily has to move in the direction of decreasing $r$ and reach the singularity at $r=0$\,.
We will call a surface which exhibits these properties an \enquote{event horizon}.

The question becomes, can these coordinates be extended such that an observer could cross the event horizon?
To cut short the derivation \cite{Carroll} for a maximally extended coordinate chart, we will introduce the Kruskal coordinates $\left\{T,R,\theta ,\phi \right\}$ which relate to $\left\{r,t,\theta ,\phi \right\}$ via 
\begin{align} 
  T &= \sqrt{\left(\dfrac{r}{r_S}-1\right)}\text{e}^{\tfrac{r}{2r_S}}\sinh \left(\dfrac{t}{2r_S}\right) &
  R &= \sqrt{\left(\dfrac{r}{r_S}-1\right)}\text{e}^{\tfrac{r}{2r_S}}\cosh \left(\dfrac{t}{2r_S}\right)
\,.\end{align} 
The metric takes the form
\begin{align} 
  \td s^2 &= \dfrac{4r_S^3}{r}\text{e}^{-\tfrac{r}{r_S}}\left(-\td T^2+\td R^2\right)+r^2\td \Omega ^2
\,.\end{align} 
where $r$ has to be implicitely defined\footnote{The solution for $r$ is the Lambert-W function $r=r_S\left(W\left(\tfrac{R^2-T^2}{\text{e}}\right)+1\right)$\,.} by $R^2-T^2=\left(1-\tfrac{r}{r_S}\right)\text{e}^{r/r_S}$.
In these coordinates the light cones are at $45^\circ$ everywhere,
\begin{align} 
  \diff[]{R}{T} &= \pm 1
\,.\end{align} 
This will therefore also hold at $r=r_S$; the event horizon of the black hole is therefore a null hypersurface, as we have already seen.
But, since the light cones behave as they would in flat space, it is no problem for an observer to cross the event horizon in a finite amount of their proper time (although to an outside observer, the infalling observer will still only reach it asymptotically).

The singularity at $r=0$ is however not liftable.
Such a singularity is called a spacetime singularity and is characterized by geodesic incompleteness.
That is, if there exists a geodesic curve, which cannot be continued to all values of the affine parameter, then there exists a spacetime singularity, such that it is a manifest property of the manifold (i.e.\ it cannot be lifted by a mere coordinate transform) \cite{Townsend1997}.\todo{explicit calculation (short)}
%}}}

%{{{ Kerr
\subsection*{Kerr}
For a rotating mass (\ref{eq:efe vacuum}) was solved by Kerr \cite{Carroll} \cite{Kerr1961}
\iffalse\begin{multline} 
  \td s^2 = -\left(1-\dfrac{2GMr}{r^2+\tfrac{J^2}{M^2}\cos ^2\theta }\right)\td t^2
  -\dfrac{2GMr\tfrac{J}{M}\sin ^2\theta }{r^2+\tfrac{J^2}{M^2}\cos ^2\theta }\left(\td t\td \phi +\td \phi \td t\right)
  +\dfrac{r^2+\tfrac{J^2}{M^2}\cos ^2\theta }{r^2-2GMr+\tfrac{J^2}{M^2}}\td r^2\\
  +\left(r^2+\dfrac{J^2}{M^2}\cos ^2\theta  \right)\td \theta ^2
  +\dfrac{\sin ^2\theta }{r^2+\tfrac{J^2}{M^2}\cos ^2\theta }\left( \left(r^2+\dfrac{J^2}{M^2}\right)^2-\dfrac{J^2}{M^2}\left(r^2-2GMr+\dfrac{J^2}{M^2}\right)\sin ^2\theta \right)\td \phi ^2 \label{eq:kerr metric}
,\end{multline} 
\fi
\begin{multline}
  \td s^2 = -\left(1-\dfrac{2GMr}{\rho ^2}\right)\td t^2-\dfrac{2GMra\sin ^2\theta }{\rho ^2}(\td t\td \phi +\td \phi \td t)+\dfrac{\rho ^2}{\Delta }\td r^2+\rho ^2\td \theta ^2+\dfrac{\sin ^2\theta }{\rho ^2}\left( \left(r^2+a^2\right)^2-a^2\Delta \sin ^2\theta \right)\td \phi ^2 \label{eq:kerr metric}
\end{multline}
where 
\begin{align} 
  \Delta =r^2-2GMr+a^2\qquad  \rho ^2=r^2+a^2\cos ^2\theta \qquad a=J/M 
\end{align} 
and $J$ is the total angular momentum of the spinning mass.
The coordinates $\left\{t,r,\theta ,\phi \right\}$ are Boyer-Lindquist coordinates with cartesian representation
\begin{align} 
  x=\sqrt{r^2+a^2}\sin \theta \cos \phi \qquad y=\sqrt{r^2+a^2}\sin \theta \sin \phi \qquad z=r\cos \theta
\,.\end{align} 
This metric has divergences at $\Delta =0$ and $\rho =0$ 
\begin{align} 
  &r^2-2GMr+a^2 = 0 &&r^2+a^2\cos \theta ^2 = 0\\
  \Rightarrow \quad&r_\pm = GM\pm \sqrt{G^2M^2-a^2} &\Rightarrow \quad&r=0 \land \theta =\dfrac{\pi }{2}
\,.\end{align} 
whereas $r_\pm$ is again a consequence of coordinates, which will be the event horizon ($r_+$ outer event horizon, $r_-$ inner event horizon) and $r=0\land \theta =\tfrac{\pi }{2}$ the singularity.
Due to the particular form of coordinates the singularity is not a point in space (as it has been in the Schwarzschild solution) but a disc
\begin{align} 
  x_S=a\cos \phi \qquad y_S=a\sin \phi \qquad z_S=0
\,.\end{align} 
with radius $a$\,.
For $a=\text{const}$ the singularity has the form of a ring.

There are now three solutions for the event horizon.

\noindent $G^2M^2<a^2$: For this case there will be no real solution of $r_\pm$ thus no event horizon and the singularity will be a naked singularity.
This stands in violation with the cosmic censorship conjecture (discussed in \textit{Black Holes}).\\
\noindent $G^2M^2=a^2$: This is the extreme Kerr black hole, which is unstable.\\
\noindent $G^2M^2>a^2$: Now both solutions of $r$ are real and can be of physical significance.
To check the structure of the surface we will use
\begin{align} 
  l^\mu =g^{\mu \nu }\partial_\nu S\qquad S=r-r_+
\,.\end{align}
then
\begin{align} 
  l^\mu l_\mu  &= g^{\mu \nu }(\partial_\mu S\partial_\nu S) = g^{r r}(\partial_r S)^2 =\dfrac{\Delta ^2}{\rho ^2}
\,.\end{align} 
which is a null hypersurface, since
\begin{align} 
  \left.l^2\right|_{r=r_+}=0
\,.\end{align} 
Now that there are two different event horizons, there is an interesting property about the spacetime between these surfaces
\begin{align} 
  \left.l^2\right|_{r=r_+}=\dfrac{\Delta (r_+)^2}{\rho ^2}=0\quad \left.l^2\right|_{r_-<r<r_+}=\dfrac{\Delta (r_-<r<r_+)^2}{\rho ^2}<0\quad \left.l^2\right|_{r=r_-}=\dfrac{\Delta (r_-)^2}{\rho ^2}=0\quad \left.l^2\right|_{r<r_-}=\dfrac{\Delta (r<r_-)^2}{\rho ^2}>0
\,.\end{align} 
where $\rho $ has to be understood as a function of the respective value of $r$\,.
Clearly, the normal vector outside the event horizon is spacelike, at the event horizon null and inside timelike.
Thus $r_+$ behaves just like the event horizon of the Schwarzschild metric at $r_S$\,.
But there is a second event horizon at $r_-$ which is again a null hypersurface.
Behind the second horizon, $l^2$ is spacelike, as it was outside the event horizon.
Thus, if an observer enters the Kerr black hole from $r_+$\,, she neccessarily has to cross the event horizon at $r_-$, but after that is free to move however she likes.
This opens up the possibility of moving back out the black hole (on the other side), as in a Kerr \enquote{white hole}, where you neccessarily have to move out of $r_+$ once you have crossed $r_-$\,.
Further discussion and implications can be found in \cite{Carroll}.

Another important surface appears for
\begin{align} 
  1-\dfrac{2GMr}{\rho ^2}=0\quad \Rightarrow \quad r^2-2GMr+a^2\cos ^2\theta =0\quad \Rightarrow \quad r_\pm^E=GM\pm\sqrt{G^2M^2-a^2\cos ^2\theta }
\,.\end{align} 
where the region inside $r_+^E$ is called the \enquote{ergosphere}.
It coincides with the outer event horizon at $\theta =0$ or $\theta =\pi $\,.
The surface $S=r-r_+^E$ with
\begin{align} 
  \left.l^2\right|_{r=r_+^E}=\dfrac{\Delta ^2(r_+^E)}{\rho ^2(r_+^E)}\neq 0
\,.\end{align} 
is evidently not null (except at $\theta =0$ and $\theta =\pi $) so it doesn't act as yet another event horizon. 
However, what will yield an interesting result is the killing vector $K=\partial _t$\,.
Its norm is
\begin{align} 
  \left.K^2\right|_{r=r_+^E}=g_{t t}=0
\,.\end{align} 
clearly, since we have just calculated the particular surface with this condition.
Just as in the Schwarzschild case, curves of constant spatial coordinates are not worldlines of observers anymore.
However this surface is not an event horizon, therefore $r$ is not timelike and one can move in whatever radial direction she wants (including out the surface once entered).
Still, one cannot remain stationary and is therefore dragged with the rotation of the black hole.
\todo{Lens-Thirring effect}

The speed at which the black hole rotates can be understood as the angular velocity at the outer event horizon.
To calculate we will consider a null trajectory at $\theta =\tfrac{\pi }{2}$ at a constant radius $r_+$ (really at $r_+$ is not possible since the surface is null so realistically it is infinitesimally above $r_+$).
The induced metric becomes %(with $g_{t t}=0$ at $r_+$)
\begin{align} 
  \td\tilde{s}^2 &= g_{t t}\td t^2+g_{t \phi }\left(\td t\td \phi +\td \phi \td t\right)+g_{\phi \phi }\td \phi ^2 = 0
\,.\end{align} 
with components
\begin{align} 
  g_{t t} &= \dfrac{2GMr_+}{\rho ^2(r_+)}-1=\dfrac{2G^2M^2+2GM\sqrt{G^2M^2-a^2}}{2G^2M^2-a^2+2GM\sqrt{G^2M^2-a^2}}-1=\dfrac{r_+^2+a^2}{r_+^2}-1\\
  g_{\phi \phi } &= \dfrac{\left(r_+^2+a^2\right)^2}{r_+^2}\\
  g_{t \phi } &= -\dfrac{2GMr_+a}{r_+^2}=-\dfrac{a\left(2G^2M^2+2GM\sqrt{G^2M^2-a^2}\right)}{r_+^2}=-\dfrac{a\left(r_+^2+a^2\right)}{r_+^2}
\,.\end{align} 
Therefore
\begin{align} 
  \left(\diff[]{\phi }{t}\right)^2 + 2g^{\phi \phi }g_{t \phi }\diff[]{\phi }{t}+g^{\phi \phi }g_{t t}=0
\,.\end{align} 
with solution
\begin{align} 
  \diff[]{\phi }{t} &= -g^{\phi \phi }g_{t \phi }\pm \sqrt{\left(g^{\phi \phi }g_{t \phi }\right)^2-g^{\phi \phi }g_{t t}}\\
                    &= \dfrac{a}{r_+^2+a^2}\pm \sqrt{\left(\dfrac{a}{r_+^2+a^2}\right)^2-\dfrac{1}{r_+^2+a^2}+\dfrac{r_+^2}{\left(r_+^2+a^2\right)^2}}=\dfrac{a}{r_+^2+a^2}
\,.\end{align} 
\todo{only one solution??? check calculation but could make sense.}
We will define the rotation speed of the black hole thus
\begin{align} 
  \Omega \equiv \dfrac{a}{r_+^2+a^2} \label{eq:kerr rotation speed}
\,.\end{align} 

Lastly we will want to calculate the gravity of the black hole at $r_+$\,.
The surface at $r_+$ is a null hypersurface which is also a Killing horizon for $K=\partial_t$\,.
Therefore $K$ is proportional to the normal vector on this surface, however it need not be affinely parameterized and will in general solve only (\ref{eq:not affine geodesics}). 
The proportionality constant $\kappa $ of (\ref{eq:not affine geodesics}) is called \enquote{surface gravity} if the equation holds for a Killing vector on a Killing horizon \cite{Townsend1997}.
This is the quantity we now want to calculate.
We will be using
\begin{align} 
  \kappa ^2=-\dfrac{1}{2}\left(\nabla ^\mu K^\nu \right)\left(\nabla _\mu K_\nu \right)
\,.\end{align} 
with $K=\partial _t$ and the Christoffel symbols from \cite{Mueller2010}, which yields \todo{could show}
\begin{align} 
  \kappa =\dfrac{r_+-r_-}{2\left(r_+^2+a^2\right)}=\dfrac{\Omega \left(r_+-r_-\right)}{2a}=\dfrac{\Omega \sqrt{G^2M^2-a^2}}{2a}
\,.\end{align} 
%}}}

%{{{ Reissner-Nordström
\subsection*{Reissner-Nordström}
The analytical solution of the Einstein field equation in vacuum for a gravitating and electrically charged mass is given by the Reissner-Nordström metric with the magnetic charge $P=0$ \cite{Carroll} \cite{Reissner1916}
\begin{align} 
  \td s^2 &= -\left(1-\dfrac{2GM}{r}+\dfrac{GQ^2}{r^2}\right)\td t^2+\left(1-\dfrac{2GM}{r}+\dfrac{GQ^2}{r^2}\right)^{-1}\td r^2+r^2\td \Omega ^2 \label{eq:reissner-nordström metric}
\,.\end{align} 
where $Q$ is the electric charge of the body.
In the case of a neutral mass the solution reduces to the Schwarzschild metric (\ref{eq:schwarzschild metric}). 
%}}}

%{{{ Kerr-Newman
\subsection*{Kerr-Newman}
Upon replacing $2GMr\rightarrow 2GMr-GQ^2 \inlineeqno\label{eq:kerr-newman metric}$ in (\ref{eq:kerr metric}) we arrive at the solution of (\ref{eq:efe vacuum}) for a rotating and charged mass \cite{Carroll}
\iffalse\begin{multline} 
  \td s^2 = -\left(1-\dfrac{2GMr-GQ^2}{r^2+\tfrac{J^2}{M^2}\cos ^2\theta }\right)\td t^2
  -\dfrac{(2GMr-GQ^2)\tfrac{J}{M}\sin ^2\theta }{r^2+\tfrac{J^2}{M^2}\cos ^2\theta }\left(\td t\td \phi +\td \phi \td t\right)
  +\dfrac{r^2+\tfrac{J^2}{M^2}\cos ^2\theta }{r^2-2GMr+GQ^2+\tfrac{J^2}{M^2}}\td r^2\\
  +\left(r^2+\dfrac{J^2}{M^2}\cos ^2\theta  \right)\td \theta ^2
  +\dfrac{\sin ^2\theta }{r^2+\tfrac{J^2}{M^2}\cos ^2\theta }\left( \left(r^2+\dfrac{J^2}{M^2}\right)^2-\dfrac{J^2}{M^2}\left(r^2-2GMr+GQ^2+\dfrac{J^2}{M^2}\right)\sin ^2\theta \right)\td \phi ^2 \label{eq:kerr-newman metric}
,\end{multline} 
\fi
%}}}

%{{{ Black Holes
\subsection*{Black Holes} \label{sec:black holes}
Black holes in general are only described by a small number of properties.
Israel \cite{Israel1967} \cite{Israel1968} was the first to deduce this property by showing that, for a static, asymptotically flat spacetime with only one gravitating mass, the Schwarzschild metric (\ref{eq:schwarzschild metric}) is the only metric for which the event horizon is regular (i.e.\ is non-singular).
This characterizes the solution by Schwarzschild to be unique.
By further analysis he also deduced that for a vacuum, in which the source-free Maxwell equations hold (all charges are said to be within the black hole), only the Reissner-Nordström metric (\ref{eq:reissner-nordström metric}) is a solution with regular event horizon.
Thus it is also unique.
Analoguously Robinson \cite{Robinson1975} proved the uniqueness of the Kerr metric (\ref{eq:kerr metric}) and Mazur \cite{Mazur1982} proved the uniqueness of the Kerr-Newman metric (\ref{eq:kerr-newman metric}) for their respective vacua.
In a general form, we can write this as a no-hair theorem or uniqueness theorem (formulation taken literally from \cite[][p.\ 238]{Carroll}).
\enquote{Stationary, asympotically flat black hole solutions to general relativity coupled to electromagnetism that are non-singular outside the event horizon are fully characterized by the parameters of mass, electric and magnetic charge, and angular momentum.}.
The impact of this theorem might not be as important right now, but it will be later when discussing its implications in the light of the information content a black hole can have (or the way two black holes can be quite similar although their respective collapsing star was not).

Two important theorems are left to mention.
The first is the still unproven \enquote{cosmic censorship conjecture} by Penrose \cite{Penrose1969} which says, that there exists no naked singularity, i.e.\ a singularity not surrounded by an event horizon, which resulted by means of gravitational collapse\footnote{It is thought that the singularity from which our universe originated could in fact be a naked singularity \cite{Hawking1972}.}.
This statement is reasonable to accept, because it would verify that the classical theory of general relativity, which breaks down at the singularity, does not so in fact, because the information, which a singularity holds, is sealed behind the event horizon, thus diminishing its impact on the theory, since it could not escape anyway \cite{Hawking1972}.

The second is the area theorem by Hawking \cite{Hawking1972}.\todo{derivation? understanding?}
It states, under the assumption of the weak energy condition and the cosmic censorship conjecture that the area of a future event horizon in an asymptotically flat spacetime cannot decrease \cite{Carroll} \cite{Hawking1972}.
%}}}

%{{{ Thermodynamics
\subsection{Thermodynamics}
The theory of theromodynamics is a phenomenological theory of heat and work.
It has been mostly established by Boltzmann, Carnot and Clausius throughout the $19^{\text{th}}$ century and relies on the following four fundamental laws.
\begin{itemize}
  \item [0.] Suppose system A and system B as well as system B and system C are in thermal equilibrium, then also system A and system C are in thermal equilibrium.
    The temperature $T$ is defined as the number which is equal for systems in thermal equilibrium by heat exchange.
    Systems in thermal equilibrium have thus a constant temperature $T$\,.
  \item [1.] The internal energy of an isolated system is conserved.
    The rate of change of internal energy of a system by interaction with another system is given by
    \begin{align} 
      \td U = T\td S+\sum_{i}^{}Y_i\td X_i
    \,.\end{align} 
    where $Y_i\td X_i$ is a way for the system to do mechanical work, $U$ is the internal energy, $T$ is the temperature, $S$ is the entropy.
  \item [2.] The total entropy of an isolated system can never decrease
    \begin{align} 
      \td S\geq 0
    \,.\end{align} 
    Equivalentely, there exists no cyclic process, which passes heat from a colder body $T_1$ to a warmer body $T_2>T_1$ while serving no other purpose.
    Equivalentely, not all mechanical work done can be converted to heat.
    A process is called reversible when $\td S=0$ and irreversible when $\td S>0$\,.
  \item [3.] The absolute zero-point temperature $T=\SI{0}{K}$ can never be reached by a finite amount of thermodynamic cycles.
    For $T$ towards zero, the entropy takes a constant value.
\end{itemize}
Thermodynamics also has a deep link to statistical mechanics by the Gibbs entropy and ensemble theory
\begin{align} 
  S &= -k_B\sum_{\mathfrak{m}}^{}p_\mathfrak{m}\ln p_\mathfrak{m}
\,.\end{align} 
where $\mathfrak{m}$ is a microstate and $p_\mathfrak{m}$ the probability that a certain microstate realizes a macrostate.
For example in the microcanonical ensemble, which can be characterized by the state variables $(N,V,E)$\,, particle number, volume, energy, the probability that a microstate realizes a certain macrostate is given by a uniform distribution which maximizes the entropy under the constraint that $\sum_{\mathfrak{m}}^{}p_\mathfrak{m}-1=0$, which is
\begin{align} 
  p = \dfrac{1}{\Omega (N,V,E)} \qquad S = k_B\ln \Omega (N,V,E)
\,.\end{align} 
where $\Omega (N,V,E)$ is the number of microstates available to the system for constant $(N,V,E)$\,, or equivalentely, which lie on the hypersurface of constant $(N,V,E)$ in the phase space.
%}}}

%}}}

%{{{ Geroch's Gedanken Experiment
\section{Geroch's Gedankenexperiment}
Geroch \cite{Geroch1971} \cite{Bekenstein1972} formulated a gedankenexperiment to venture about the second law of thermodynamcis, when black holes are considered part of the system.
The preassumption is that black holes are described by a macrostate, which is realized by exactly one microstate, since they are analytical solutions to the Einstein equation (\ref{eq:efe vacuum}); i.e.\ a black hole per se is not a statistical system described by a probability distribution of certain microstates.

Now we consider the following construct:
An observer is equipped with an ideal rope and a body of mass $m$ and she resides near a Schwarzschild black hole, such that she is at rest compared to it (e.g.\ at infinity).
The metric is therefore (\ref{eq:schwarzschild metric}) and we will set $\td \Omega ^2=0$
\begin{align} 
  \td s^2 &= -\left(1-\dfrac{r_S}{r}\right)\td t^2+\left(1-\dfrac{r_S}{r}\right)^{-1}\td r^2
\,.\end{align} 
Now she lowers the body towards the black hole's event horizon.
To make claims about the work done we first need to calculate the body's energy as measured by the observer.

Since none of the metric coefficients depend on time, there exists a timelike Killing vector
\begin{align} 
  K_\mu  = g_{\mu \nu }(\partial_t)^\nu = g_{\mu \nu }\delta _t^\nu = \left(-\left(1-\dfrac{r_S}{r}\right)\quad 0\quad 0\quad 0\right)
\,.\end{align} 
with a constant of motion 
\begin{align} 
  K_\mu \diff[]{x^\mu }{\tau }=\text{const}=-\left(1-\dfrac{r_S}{r}\right)\diff[]{t}{\tau }
\,.\end{align} 
where we have chosen $\lambda =\tau $ the proper time as the affine parameter.
From Noether's theorem it follows that the constant of motion must be the energy.
We will therefore set
\begin{align} 
  E &= \left(1-\dfrac{r_S}{r}\right)\diff[]{t}{\tau }
\,.\end{align} 
If we move with the body on its worldline, then the metric will be
\begin{align} 
  \td s^2 &= -\left(1-\dfrac{r_S}{r}\right)\td t^2 = -\td \tau ^2
\,.\end{align} 
This fixes the derivative in our equation for the energy 
\begin{align} 
  E &= \dfrac{\left(1-\dfrac{r_S}{r}\right)}{\sqrt{1-\dfrac{r_S}{r}} }=\sqrt{1-\dfrac{r_S}{r}}
\,.\end{align} 
This energy can be understood as the energy per unit mass, such that
\begin{align} 
  E &= m\sqrt{1-\dfrac{r_S}{r}}
\end{align} 
is the total energy of our body with mass $m$ \cite{Carroll}.

This is sufficient to further construct the experiment.
As she lowers the body towards the event horizon, the body performs work on her which is proportional to its initial mass $m$\,.
When the body reaches the event horizon its energy is
\begin{align} 
  \lim_{r\rightarrow r_S}E=0
\,.\end{align} 
Now the body is allowed to radiate into the black hole resulting in a decrease of its mass $m'=m-\Delta m$\,.
But since the body has no energy at the event horizon as measured from infinity, the black hole will also have no increase in its mass due to the radiation.
Afterwards, the observer raises the body again and has performed a total amount of work $\Delta m$\,.
The body has therefore converted all work $\Delta m$ into heat, since nothing about the black hole has changed. \todo{is the proess cyclic?} This is in clear violation of the second law of thermodynamics, that not all heat can be converted into work within a cyclic process.
To solve this apparent paradox, it is needed for the process to not be cyclic, i.e.\ there needs to be a quantity that changes irreversibly after one cycle.

A logical candidate for this quantity is the area $A$ of the black hole, since it is never decreasing as stated in Hawking's theorem.
In the process where our observer then lowers the mass $m$ to the event horizon, the area of the black hole increases (while lowering or when the body radiates) irreversibly, resulting in the process not being reversible thus providing a solution to the paradox.
The heat $\Delta m$ has not been converted to work completely, but instead some of it has gone into increasing the area of the black hole.
Now that there exists such a non-decreasing quantity, which respects the second law of thermodynamics, Bekenstein \cite{Bekenstein1972} constructed a form of entropy for black holes
\begin{align} 
  S_{bh}\equiv S_{b.h.} &= \eta k_BL_p^{-2}A
\,.\end{align} 
where $\eta $ is a constant number of order unity, $k_B$ is the Boltzmann constant, $L_p$ is the Planck length. \todo{explanation of constants, reason}
Consequently, the second law of thermodynamics has to be rephrased to capture also the entropy of the black hole \cite{Carroll}:
The total entropy of matter and of black holes in an isolated system can never decrease
\begin{align} 
  \td \left(S+S_{bh}\right)\geq 0
\,.\end{align} 
%}}}

%{{{ Penrose Process
\section{Penrose Process}
To further see a link between thermodynamics and black hole theory, we should construct a relation between the area of a black hole $A$ and the other parameters mass $M$\,, angular momentum $J$ and charge $Q$\,.
Since these are sufficient to describe every black hole as stated by the no-hair or uniqueness theorem, we should be able to find a relation which characterizes how each quantity changes in relation to the other.
Let us think about which of these could be readily accessed.
The area is always non-decreasing therefore not prone to change as we would like;
the mass could be increased by simple means but not readily extracted;
for the charge we have yet to link the Einstein equation with Maxwell's equation, thus we will set it to zero and will be of no further interest.
This leaves us with the angular momentum, i.e.\ Kerr black holes, which we, in principle, have access to.
We should therefore think about how we could extract angular momentum from it.
Such a process has been developed by Penrose \cite{Penrose1969} \cite{Carroll}.

Consider a Kerr black hole with angular momentum $J$ and a particle with four-momentum $P^\mu $\,.
The energy of the particle is given by 
\begin{align} 
  E = K^\mu P_\mu = P_t \qquad K = \partial_t
\,.\end{align} 
where $K$ is a Killing vector of the Kerr metric (\ref{eq:kerr metric}) with norm $K^2=g_{t t}=\tfrac{2GMr}{\rho ^2}-1$
\begin{align} 
  \lim_{r \rightarrow \infty}K^2 = -1\qquad \left.K^2\right|_{r=r_+^E} = 0\qquad \left.K^2\right|_{r<r_+^E} > 0
\end{align} 
Now imagine the particle enters the ergosphere, traverses it for a moment and at a certain point in time, splits into two components
\begin{align} 
  P^\mu =\tensor[^1]{P}{^\mu }+\tensor[^2]{P}{^\mu }
\,.\end{align} 
The particle manages to do it in such a way that
\begin{align} 
  \tensor[^1]{P}{^0}<0\quad \Rightarrow \quad \tensor[^2]{P}{^0}>P^0
\,.\end{align} 
where the second particle escapes from the ergosphere to infinity.
It is entirely possible for the energy to be negative since the Killing vector $K=\partial_t$ is spacelike inside the ergophere.
Now, since particle 2 has more energy than the initial particle (from where it originated), energy has been extracted from somewhere, otherwise this would be a violation of conservation of energy.
As it turns out, this energy comes from the reduction of the angular momentum $J$ of the black hole.
To understand how $J$ has been reduced we will take another look at the symmetry of the Kerr metric; not only $K=\partial_t$ but also $R=\partial _\phi $ is a Killing vector, such that
\begin{align} 
  L=R^\mu P_\mu =P_\phi \qquad R=\partial_\phi 
\end{align} 
is a conserved quantity, namely the angular momentum of the particle.
Since every event horizon in a stationary spacetime is a Killing horizon \cite{Carroll}, then the Killing vector
\begin{align} 
  \chi ^\mu =K^\mu +\Omega R^\mu\qquad \Omega \stackrel{(\ref{eq:kerr rotation speed})}{=}\dfrac{a}{r_+^2+a^2}
\,.\end{align} 
is therefore null along the outer event horizon.
Since we want to understand how $J$ has been reduced we will track the path that particle 1 takes.
Particle 1 has negative energy, so it will not escape to infinity but eventually cross the outer event horizon $r_+$\,.
This happenes, when
\begin{align} 
  \chi ^\mu \tensor[^1]{P}{_\mu }<0
\,.\end{align} 
as $\chi ^\mu $ will become spacelike.
This also implies that
\begin{align} 
  K^\mu \tensor[^1]{P}{_\mu }+\Omega R^\nu \tensor[^1]{P}{_\nu }=\tensor[^1]{E}{}+\Omega \tensor[^1]{L}{}<0\quad \Rightarrow \quad \tensor[^1]{L}{}<\dfrac{\tensor[^1]{E}{}}{\Omega }
\,.\end{align} 
where $\tensor[^1]{E}{}<0$ and $\Omega >0$ such that $\tensor[^1]{L}{}<0$\,.
Particle 1 therefore adds negative energy and negative angular momentum to the hole, which decreases its mass and its angular momentum
\begin{align} 
  \delta M=\tensor[^1]{E}{}\qquad \delta J=\tensor[^1]{L}{}\quad \Rightarrow \quad \delta J\leq \dfrac{\delta M}{\Omega } \label{eq:angular momentum mass relation}
\,.\end{align} 
where the equality holds in the most ideal process.
We could argue that $\tensor[^1]{E}{}$ and $\tensor[^1]{L}{}$ are somewhat arbitrary and it is therefore possible to completely extract all mass and rotational energy from the black hole, but this is not the case. It has been shown, that at most 29\% of its total energy can be extracted this way \cite{Carroll}.
The reason is the irreducible mass of a black hole \cite{Christodoulou1970} \cite{Carroll}.
If one were to deplete the Kerr black hole of all its angular momentum, then a Schwarzschild black hole of mass equal to the irreducible mass remains.
The irreducible mass can be understood as the energy, which is contained within the outer event horizon $r_+$ and can therefore not leave the black hole.
It has been derived by Christodoulou \cite{Christodoulou1970} as a mass-energy formula\todo{argument for formula from christodoulou}
\begin{align}
  G^2M^2=G^2M_\text{irr}^2+\dfrac{G^2J^2}{4G^2M_\text{irr}^2}=G^2M_\text{irr}^2+\dfrac{a^2G^2M^2}{4G^2M_\text{irr}^2}
\,.\end{align} 
We can solve this equation for $GM_\text{irr}$ 
\begin{align} 
  G^4M_\text{irr}^4-G^2M_\text{irr}^2G^2M^2+\dfrac{a^2G^2M^2}{4}=0
\end{align} 
and we will only consider the \enquote{+} solution, since we want the mass within $r_+$ 
\begin{align} 
  G^2M_\text{irr}^2=\dfrac{1}{2}\left( G^2M^2+\sqrt{G^2M^2\left( G^2M^2-a^2\right)}\right)=\dfrac{1}{2}GM\left(GM+\sqrt{G^2M^2-a^2}\right)=\dfrac{1}{2}GMr_+ \label{eq:irreducible mass 0}
\,.\end{align} 
Finally
\begin{align} 
  M_\text{irr}=\sqrt{\dfrac{Mr_+}{2G}} \label{eq:irreducible mass 1}
\end{align} 
As of now this expression cannot be further simplified so we will take a look at the area.
We can calculate it by integrating over the surface with the induced metric on the outer event horizon
\begin{align} 
  \td \tilde{s}^2=\tilde{g}_{\mu \nu }\td x^\mu \td x^\nu =\td s^2\left(\td t=\td r=0\right)|_{r=r_+} &= \rho ^2(r_+)\td \theta ^2+\dfrac{\sin ^2\theta }{\rho ^2(r_+)}\left( \left(r_+^2+a^2\right)^2-a^2\Delta (r_+)\sin ^2\theta \right)\td \phi ^2\\
                                                                                               &= \rho ^2(r_+)\td \theta ^2+\dfrac{\sin ^2\theta }{\rho ^2(r_+)}\left(r_+^2+a^2\right)^2\td \phi ^2
\,.\end{align} 
With $\Delta (r_+)=0$\,, thereby
\begin{align} 
  A &= \int_{}^{}\td \tilde{s}\,=\int_{}^{}\td \theta \td \phi \,\sqrt{\tilde{g}}=\int_{0}^{\pi }\td \theta \,\int_{0}^{2\pi }\td \phi \,\sin \theta \left(r_+^2+a^2\right)=4\pi \left(r_+^2+a^2\right) \label{eq:kerr area 1}
\,.\end{align} 
We can plug $r_+$ into this equation to get
\begin{align} 
  A=4\pi \left(r_+^2+a^2\right) &= 4\pi \left(G^2M^2+G^2M^2-a^2+2GM\sqrt{G^2M^2-a^2}+a^2\right)\\
                                &= 4\pi \left(2G^2M^2+2GM\sqrt{G^2M^2-a^2}\right)=8\pi GMr_+ \label{eq:kerr area 2}
\end{align} 
Now with (\ref{eq:irreducible mass 1}) and (\ref{eq:kerr area 2})
\begin{align} 
  M_\text{irr}=\sqrt{\dfrac{A}{16\pi G^2}} \label{eq:irreducible mass 2}
\,.\end{align} 
However to do further calculations, we should relate $M_\text{irr}$ to $M$ and $J$ again
\begin{align} 
  M_\text{irr}=\sqrt{\dfrac{r_+^2+a^2}{4G^2}} &= \sqrt{\dfrac{G^2M^2+G^2M^2-a^2+2GM\sqrt{G^2M^2-a^2}+a^2}{4G^2}}\\
                                                  &= \sqrt{\tfrac{1}{2}M^2+\tfrac{M}{2G}\sqrt{G^2M^2-\tfrac{J^2}{M^2} }}\\
                                                  &= \dfrac{1}{\sqrt{2}}\sqrt{M^2+\sqrt{M^4-\tfrac{J^2}{G^2} }} \label{eq:irreducible mass}
\,.\end{align} 
We can now take the variation of (\ref{eq:irreducible mass}) to have a formula for the variation of the area.
Since both $M_\text{irr}$ and $A$ are related their variation will surely be positive, so we can use it to verify (\ref{eq:angular momentum mass relation}).
\begin{align} 
  \delta M_\text{irr} &= \dfrac{1}{\sqrt{2}}\delta \left(\sqrt{M^2+\sqrt{M^4-\tfrac{J^2}{G^2} }}\right)\\
                      &= \dfrac{1}{\sqrt{2}}\dfrac{1}{\sqrt{2}M_\text{irr}}\dfrac{1}{2}\delta \left(M^2+\sqrt{M^4-\tfrac{J^2}{G^2}}\right)\\
                      &= \dfrac{1}{4M_\text{irr}}\left(2M\delta M+\dfrac{4M^3\delta M-\tfrac{2J}{G^2}\delta J}{2\sqrt{M^4-\tfrac{J^2}{G^2} }}\right)\\
                      &= \dfrac{1}{4M_\text{irr}}\left( \left(\dfrac{M\sqrt{M^4-\tfrac{J^2}{G^2}}+2M^3}{\sqrt{M^4-\tfrac{J^2}{G^2} }}\right)\delta M-\dfrac{J}{G^2\sqrt{M^4-\tfrac{J^2}{G^2} }}\delta J\right)\\
                      \intertext{using $\sqrt{M^4-\tfrac{J^2}{G^2}}=\tfrac{M}{G}\sqrt{G^2M^2-a^2}$}
                      &= \dfrac{1}{4M_\text{irr}}\left( \left(\dfrac{\tfrac{M^2}{G}\sqrt{G^2M^2-a^2}+2M^3}{\tfrac{M}{G}\sqrt{G^2M^2-a^2}}\right)\delta M-\dfrac{J}{GM\sqrt{G^2M^2-a^2}}\delta J\right)\\
                      &= \dfrac{1}{4M_\text{irr}\sqrt{G^2M^2-a^2}}\left( \left(M\sqrt{G^2M^2-a^2}+2GM^2\right)\delta M-\dfrac{a}{G}\delta J\right)\\
                      \intertext{now multiply with $\tfrac{G}{G}$ to get equal powers of $G$ and $M$ and factor out $a$}
                      &= \dfrac{a}{4GM_\text{irr}\sqrt{G^2M^2-a^2}}\left( \dfrac{1}{a}\left(GM\sqrt{G^2M^2-a^2}+2G^2M^2\right)\delta M-\delta J\right)\\
                      \intertext{recall $\Omega =\tfrac{a}{r_+^2+a^2}=a\left(2G^2M^2+2GM\sqrt{G^2M^2-a^2}\right)^{-1}$}
                      &= \dfrac{a}{4GM_\text{irr}\sqrt{G^2M^2-a^2}}\left(\dfrac{\delta M}{\Omega }-\delta J\right)=\delta M_\text{irr}
\,.\end{align} 
As we have said
\begin{align} 
  0<\delta M_\text{irr}\propto \dfrac{\delta M}{\Omega }-\delta J\quad \Leftrightarrow \quad \delta J\leq \dfrac{\delta M}{\Omega }
\,.\end{align} 
where the equality holds if $M_\text{irr}$ does not undergo change, i.e.\ only in the ideal case.

As we said in the beginning we want to relate $M$ with $A$ and $J$\,, therefore
\begin{align} 
  \delta M_\text{irr} &= \dfrac{1}{2M_\text{irr}}\dfrac{\delta A}{16\pi G^2}\\
  \dfrac{a}{4GM_\text{irr}\sqrt{G^2M^2-a^2}}\left(\dfrac{\delta M}{\Omega }-\delta J\right) &= \dfrac{1}{2M_\text{irr}}\dfrac{\delta A}{16\pi G^2}\\
  \delta M &= \dfrac{1}{2M_\text{irr}}\dfrac{\delta A}{16\pi G^2}\dfrac{4GM_\text{irr}\sqrt{G^2M^2-a^2}\Omega }{a}+\Omega \delta J\\
  \delta M &= \dfrac{\sqrt{G^2M^2-a^2}\Omega }{8\pi Ga}\delta A+\Omega \delta J\\
  \intertext{with the surface gravity $\kappa =\tfrac{\Omega \sqrt{G^2M^2-a^2}}{a}$}
  \Aboxed{\delta M &= \dfrac{\kappa }{8\pi G}\delta A+\Omega \delta J}
\,.\end{align} 
Thus we have the promised relation between the quantities describing a Kerr black hole.\footnote{Do not interchange $\kappa $ (the surface gravity) with the Einstein gravitational constant, also called $\kappa $\,,  which happenes to be equal to $8\pi G$\,.}
It is not that we have done this for no reason; recall from earlier that we already concluded the second law of thermodynamics could be violated by the means of black holes, and promptly found a more general form which lifted this paradox and also accounted for black holes.
However we have not violated the first law of thermodynamics with this procedure, nevertheless we see a clear correspondence to it (at least from the looks of the formula)
\begin{align} 
  \td U &= \delta Q+\delta W=T\td S+\delta W
\,.\end{align} 
It seems that 
\begin{align} 
  \delta M \leftrightarrow \td U\qquad \dfrac{\kappa }{8\pi G}\delta A\leftrightarrow T\td S\qquad \Omega \delta J\leftrightarrow \delta W
\,.\end{align} 
The identification makes sense, the mass should be equal to the internal energy, since the only property a Schwarzschild black hole has is its mass\footnote{Aside from its area, but the entropy will be present nevertheless.}, the area has already been equated with the entropy and the angular momentum has been said to extract energy from the black hole which would translate to work done on the system.
An important conclusion we could draw is that the surface gravity $\kappa $ is somehow related to the temperature $T$\,, but no exact relation is known until here, since only a proportionality has been proposed for the area and entropy.
%}}}

%{{{ Hawking Radiation
\section{Hawking Radiation}
As of now, we have only dealt with classical general relativity, in the extend to not include quantum mechanical effects into our calculations.
This yielded some very profound results, but we have not been able to fix the equations relating quantities of black holes to quantities of classical thermodynamics.
As we have seen in thermodynamics, the entropy connects the microscopic quantum mechanical properties of systems with the macroscopic variables of state. 
Therefore, it will be useful if we could get an exact equation which relates the entropy with the area of a black hole.
For this, we will now consider quatum field theory in curved spacetime.

Quantum field theory deals with the quantization of fields.
We will have a vacuum state $\ket{0}$\,, which is defined such that the annihilation operator $a$ yields zero upon acting on the vacuum, $a\ket{0}=0$\,.
Any other exited state can be generated by $\ket{n_k}=\tfrac{1}{\sqrt{n_k!}}\left(a^\dagger _k\right)^{n_k}\ket{0}$\,, where $k$ denotes exitations with the same momentum $\vv{k}$\,.
An exitation will in general be a plane wave, also called \enquote{mode}, $\psi \left(x,t\right)=\psi _0\text{e}^{\text{i}k^\mu x_\mu }$\,, which can be understood as a particle upon approximating to first quantization.
The motion of plane waves are governed by the Klein-Gordon equation $\partial ^2\psi -m^2\psi -\xi R\psi =0$\,, where $m$ is the mass and $\xi $ denotes the coupling strength to the curvature scalar $R$\,.
To these modes exists a conjugate momentum $\pi \left(t,x\right)=\partial_t \psi \left(t,x\right)$\,.
These solutions will in general fulfil the canonical commutation relations
\begin{align} 
  \left[\psi \left(t,x\right),\psi \left(t,x'\right)\right]=\left[\pi \left(t,x\right),\pi \left(t,x'\right)\right]=0\qquad \left[\psi \left(t,x\right),\pi \left(t,x'\right)\right]=\text{i}\delta \left(x-x'\right) \label{eq:ccr}
\,.\end{align} 
such that the theory can be quantized by expansion in terms of the annihilation and creation operators, which fulfil the same commutation relations \cite{Carroll}.

The following discussion is be based on the derivation by Hawking \cite{Hawking1975} and Wald \cite{Wald1984} with insight on the derivation of the modes $f$ and $p$ gained from \cite{Donne2019}.
A cruicial fact is that if the modes $f$ solve the Klein-Gordon equation, then a linear combination of these modes will also solve the Klein-Gordon equation.
Therefore any field $\psi $ can be expanded with these modes
\begin{align} 
  \psi =\sum_{i}^{}\left(f_ia_i+\overline{f}_ia^\dagger _i\right) \label{eq:linear combination}
\,.\end{align} 
where $a$ and $a^\dagger $ are the annihilation and creation operators.
The modes fulfil the following property
\begin{align} 
  \partial _t f = -\text{i}\omega f\text{ \enquote{positive frequency}} \qquad \partial _t \overline{f}= \text{i}\omega \overline{f}\text{ \enquote{negative frequency}}
\,.\end{align} 
(although $\omega >0$) while also
\begin{align} 
  \left(a,b\right)\equiv \text{i}\int_{\Sigma }^{}\td \mu (\Sigma )\,\left(\overline{b}\partial _t a-a \partial _t \overline{b}\right)\qquad \left(f_\mu ,f_\nu \right)=\delta _{\mu \nu }\qquad \left(\overline{f}_\mu ,\overline{f}_\nu \right)=-\delta _{\mu \nu }\qquad \left(f_\mu ,\overline{f}_\nu \right)=0
\,.\end{align} 
where $\Sigma $ is a Cauchy surface and $\left(\cdot ,\cdot \right)$ is the Klein-Gordon scalar product, thus they form a complete set.
This will hold in general if the spacetime is static and flat, because if those properties are not given, we cannot readily define positive and negative frequency modes with respect to a time derivative, since the scalar product $k^\mu x_\mu $ then has mixed time and space components \todo{actuall not, only if $g$ is not static then then derivatives don't separate into sum (marc). if metric is not flat then also no problem}.
In turn, we cannot decompose the solution $\psi $ into the closed form (\ref{eq:linear combination}) and therefore cannot determine how $\psi $ splits into $a$ and $a^\dagger $\,.
We will see that we are faced with exactly this problem, because in this work, we are concerned about black holes as well as their creation.
Thus, if we allow there to exist an asymptotically flat spacetime in the infinite past, say at $\mathscr{I}^-$\,, let matter accumulate over a time span such that a black hole forms, and at the end the spacetime will be asymptotically flat in the infinite future, say at $\mathscr{I}^+$, then this spacetime is definitely not static nor flat.
What we can do however, is to define a set of positive and negative frequency modes in the flat and static spacetime which is in our reach.
For example we could say that on $\mathscr{I}^-$\,, there exists a set of modes $f$, such that
\begin{align} 
  \psi =\sum_{i}^{}\left(f_ia_i+\overline{f}_ia^\dagger _i\right) \label{eq:ingoing solution}
\,.\end{align} 
and $\psi $ will solve the Klein-Gordon equation on $\mathscr{I}^-$\,.
We will choose $f$ such that they are positive frequency with respect to the affine parameter at $\mathscr{I}^-$\,.
Now since this set of modes is completely valid on $\mathscr{I}^-$ as it is in flat spacetime, we can use it to define $a$ and $a^\dagger $ as the annihilation and creation operator of modes on $\mathscr{I}^-$\,.
We can identify this with ingoing particles from $\mathscr{I}^-$\,.
Analoguously on $\mathscr{I}^+$ 
\begin{align} 
  \psi = \sum_{i}^{}\left(p_ib_i+\overline{p}_ib^\dagger _i\right) \label{eq:outgoing solution}
\,.\end{align} 
where we will identify the creation and annihilation operators $b$ and $b^\dagger $ with outgoing particles to $\mathscr{I}^+$\,.
Since both (\ref{eq:ingoing solution}) and (\ref{eq:outgoing solution}) completely determine $\psi $\,, they are valid for all times.
We also have to complete sets of modes $f$ and $p$\,, which means that they can be expanded into each other \todo{bogolyubov transform}
\begin{align} 
  p_i &= \sum_{j}^{}\left(\alpha _{ij}f_j+\beta _{ij}\overline{f}_j\right)\qquad b_i= \sum_{j}^{}\left(\alpha _{ij}a_j+\beta _{ij}a^\dagger _j\right) \label{eq:bogolyubov transform 1}\\
  f_i &= \sum_{j}^{}\left(\overline{\alpha }_{ij}p_j-\overline{\beta }_{ij}\overline{p}_j\right)\qquad a_i = \sum_{j}^{}\left(\overline{\alpha }_{ij}b_j-\overline{\beta }_{ij}b^\dagger \right) \label{eq:bogolyubov transform 2}
\,.\end{align} 
where $\alpha $ and $\beta $ are coefficients, which fulfil
\begin{align} 
  \sum_{j}^{}\left(\alpha _{ik}\overline{\alpha }_{jk}-\beta _{ik}\overline{\beta }_{jk}\right)=\delta _{ij}\qquad \sum_{j}^{}\left(\alpha _{ik}\beta _{jk}-\beta _{ik}\alpha _{jk}\right)=0 \label{eq:bogolyubov coefficients norm}
\,.\end{align} 
The coefficients can therefore also be expressed as a scalar product of the modes
\begin{align} 
  \alpha _{ij}=\left(p_i,f_j\right)\qquad \beta _{ij}=-\left(p_i,\overline{f}_j\right)
\,.\end{align} 
\todo{could show}
We are now faced with a different problem though.
As we can see from (\ref{eq:bogolyubov transform 2}), the annihilation operator for $\mathscr{I}^-$ is a linear combination of the annihilation and creation operator of $\mathscr{I}^+$\,.
This means, that where $b$ defines a vacuum, $a$ might not define it as a vacuum, since when acting with $b$\,, so will act $a$ as well as $a^\dagger $\,.
The creation and annihilation operators do not share the same vacuum.
We will define the vacuum on $\mathscr{I}^-$ as $\ket{0_-}$ and the vacuum at $\mathscr{I}^+$ as $\ket{0_+}$\,, then we have
\begin{align} 
  a\ket{0_-} &= 0\qquad a\ket{0_+}\neq 0\\
  b\ket{0_-} &\neq 0\qquad b\ket{0_+}=0
\,.\end{align} 
We could do even more and ask, what will be the expected number of particles in the vacuum $\ket{0_-}$\,, upon defining the modes on $\mathscr{I}^+$
\begin{align} 
  \bra{0_-}b^\dagger _ib_i \ket{0_-} &= \bra{0_-}\sum_{j}^{}\left(\overline{\alpha }_{ij}a^\dagger _j+\overline{\beta }_{ij}a_j\right)\left(\alpha _{ij}a_j+\beta _{ij}a^\dagger _j\right)\ket{0_-}\\
                                 &= \bra{0_-}\sum_{j}^{}\left(|\alpha _{ij}|^2a^\dagger _ja_j+\overline{\alpha }_{ij}a^\dagger _j\beta _{ij}a^\dagger _j+\overline{\beta }_{ij}a_j\alpha _{ij}a_j+|\beta _{ij}|^2a_ja^\dagger _j\right)\ket{0_-}\\
                                 &= \sum_{j}^{}|\beta _{ij}|^2 \label{eq:beta squared}
\,.\end{align} 
which will not be zero in general.
Just from the fact that the modes cannot be continued throughout the whole spacetime due to curvature, there is a different notion of vacuum and one observer with a different set of modes will define particles, where another observer will define a vacuum.
\todo{detect these particles: Carroll p.\ 399: need proper time}

We are now tasked to calculate $|\beta _{ij}|^2$ to obtain information about the spectrum of particles.
To perform explicit calculation we will consider a massless field in the Schawrzschild metric (\ref{eq:schwarzschild metric}) in Eddington-Finkelstein with tortoise-coordinates
\begin{align} 
  v=t+r^*\qquad u=t-r^*\qquad r^*\equiv r+2GM\ln\left(\dfrac{r}{2GM}-1\right)
\,.\end{align} 
where $v$ is the ingoing coordinate and $u$ is the outgoing coordinate, such that ingoing radial null geodesics are $v=\text{const}$ and outgoing $u=\text{const}$\,.
The metric takes the form \cite{Carroll}
\begin{align} 
  \td s^2\left(t,r^*,\theta ,\phi \right) &= \left(1-\dfrac{2GM}{\tilde{r}}\right)\left(-\td t^2+\td {r^*}^2\right)+\tilde{r}^2\td \Omega ^2\\
  \td s^2\left(v,r,\theta ,\phi \right) &= -\left(1-\dfrac{2GM}{r}\right)\td v^2+\td v\td r+\td r\td v+r^2\td \Omega ^2
\,.\end{align} 
for, the usual time coordinate but tortoise radial coordinate, and, the Eddington-Finkelstein time coordinate with usual radial coordinate, respectively.
We set $\tilde{r}\equiv r(r^*)$ the radial coordinate as a function of the tortoise coordinate.
The field solves the massless Klein-Gordon equation with $\xi =0$ since we are interested in solutions on $\mathscr{I}^\pm$ 
\begin{align} 
  \partial ^2f = \dfrac{1}{\sqrt{-g}}\partial _\mu \left(\sqrt{-g}g^{\mu \nu }\partial _\nu f \right) = 0 \label{eq:massless klein-gordon}
\,.\end{align} 
In the standard Schwarzschild metric (\ref{eq:schwarzschild metric}), (\ref{eq:massless klein-gordon}) becomes
\begin{align} 
  -\dfrac{1}{1-\tfrac{2GM}{r}}\partial ^2_t f +\dfrac{1}{r^2}\partial _r\left(r^2\left(1-\dfrac{2GM}{r}\right)\partial _r f \right)+\dfrac{1}{r^2}\laplace _{\theta ,\phi }f =0
\,.\end{align} 
We switch to the tortiose coordinate $r^*$ (and use $\tilde{r}=r(r^*)$ and thereby $\partial _r\rightarrow \partial _{\tilde{r}}$) to simplify the equation
\begin{align} 
  \partial _{\tilde{r}}=\diffp[]{r^*}{\tilde{r}}\partial _{r^*}=\left(1+2GM\dfrac{\tfrac{1}{2GM}}{\tfrac{\tilde{r}}{2GM}-1}\right)\partial _{r^*}=\dfrac{1}{1-\tfrac{2GM}{\tilde{r} }}\partial _{r^*}
\end{align} 
then
\begin{align} 
  -\partial _t^2f +\dfrac{1}{\tilde{r}^2}\partial _{r^*}\left(\tilde{r}^2\partial _{r^*}f \right)+\left(1-\dfrac{2GM}{\tilde{r}}\right)\dfrac{1}{\tilde{r}^2}\laplace _{\theta ,\phi }f =0
\,.\end{align} 
Since the Schwarzschild spacetime is spherically symmetric we can write $f$ as a product of a radial function and the spherical harmonics.
We also choose the appropriate normalization to be $\tfrac{1}{\sqrt{2\pi \omega }}$\,.
\begin{align} 
  f_{\omega 'l m}\equiv \dfrac{1}{\sqrt{2\pi \omega '}}\cdot \dfrac{1}{r}R_{\omega '}(r,t)\cdot Y_{l m}(\theta ,\phi )
\,.\end{align} 
whereby the factor $\tfrac{1}{r}$ will simplify the equation.
Then, plugging $f$ into the wave equation, multiplying with $\sqrt{2\pi \omega '}$ and leaving the dependencies for better readability
\begin{align} 
  \Leftrightarrow \quad -Y\dfrac{1}{\tilde{r}}\partial _t^2R+Y\dfrac{1}{\tilde{r}^2}\partial _{r^*}\left(\tilde{r}^2\partial _{r^*}\left(\dfrac{1}{\tilde{r}}R\right)\right)+\left(1-\dfrac{2GM}{\tilde{r}}\right)\dfrac{1}{\tilde{r}^2}\dfrac{1}{\tilde{r}}R\laplace _{\theta ,\phi }Y &= 0
  \intertext{we can use $\laplace _{\theta ,\phi }Y_{l m}(\theta ,\phi )=l(l+1)Y_{l m}(\theta ,\phi )$ and the divide by $Y_{l m}(\theta ,\phi )\tfrac{1}{\tilde{r}}$}
  \Leftrightarrow \quad -\partial _t^2R+\dfrac{1}{\tilde{r}}\partial _{r^*}\left(\tilde{r}^2\partial _{r^*}\left(\dfrac{1}{\tilde{r}}R\right)\right)+\left(1-\dfrac{2GM}{\tilde{r}}\right)\dfrac{l(l+1)}{\tilde{r}^2}R &= 0\\
  \Leftrightarrow \quad -\partial _t^2 R+\dfrac{1}{\tilde{r}}\partial _{r^*}\left(-\diffp[]{\tilde{r}}{r^*}R+\tilde{r}\partial _{r^*}R\right)+\left(1-\dfrac{2GM}{\tilde{r}}\right)\dfrac{l(l+1)}{\tilde{r}^2}R &= 0\\
  \Leftrightarrow \quad -\partial _t^2 R-\dfrac{1}{\tilde{r}}\diffp[2]{\tilde{r}}{r^*}R-\dfrac{1}{\tilde{r}}\diffp[]{\tilde{r}}{r^*}\partial _{r^*}R+\dfrac{1}{\tilde{r}}\diffp[]{\tilde{r}}{r^*}\partial _{r^*}R+\partial _{r^*}^2R +\left(1-\dfrac{2GM}{\tilde{r}}\right)\dfrac{l(l+1)}{\tilde{r}^2}R &= 0\\
  \intertext{now use $\diffp[]{\tilde{r}}{r^*}=1-\tfrac{2GM}{\tilde{r}}$ and $\diffp[2]{\tilde{r}}{r^*}=-\tfrac{2GM}{\tilde{r}^2}\diffp[]{\tilde{r}}{r^*}=-\tfrac{2GM}{\tilde{r}^2}\left(1-\tfrac{2GM}{\tilde{r}}\right)$}
  \Leftrightarrow \quad -\partial _t^2 R_{\omega '}(\tilde{r},t)+\partial _{r^*}^2 R_{\omega '}\left(\tilde{r},t\right)+\left(1-\dfrac{2GM}{\tilde{r}}\right)\left(\dfrac{l(l+1)}{\tilde{r}^2}+\dfrac{2GM}{\tilde{r}^3}\right)R_{\omega '}(\tilde{r},t) &= 0
\,.\end{align} 
This form allows us to readily use Eddington-Finkelstein coordinates and switch back to the standard Schwarzschild radius $r$ 
\begin{align} 
  v &= t+r^* & t &= \dfrac{u+v}{2} & \partial _t &= \dfrac{1}{2}\partial _u+\dfrac{1}{2}\partial _v\\
  u &= t-r^* & r^* &= \dfrac{u-v}{2} & \partial _{r^*} &= \dfrac{1}{2}\partial _u-\dfrac{1}{2}\partial _v
\,.\end{align} 
Thereby 
\begin{align} 
  -\partial _t^2+\partial _{r^*}^2=\dfrac{1}{4}(-\partial _u^2-\partial _v^2+\partial _u^2+\partial _v^2)-\dfrac{1}{2}\partial _u\partial _v-\dfrac{1}{2}\partial _u\partial _v=-\partial _u\partial _v
\end{align} 
and
\begin{align} 
  -\partial _v \partial _u R_{\omega '}(r,t)+\left(1-\dfrac{2GM}{r}\right)\left(\dfrac{l(l+1)}{r^2}+\dfrac{2GM}{r^3}\right)R_{\omega '}(r,t)=0 \label{eq:edd-fink radial part}
\,.\end{align} 
This equation can be solved with the ansatz $R_{\omega '}(r,t)=R_{\omega '}^1(v)+R_{\omega '}^2(u)$\,.
We will pick $R_{\omega '}^1(v)$ since we are interested in the ingoing solutions from $\mathscr{I}^-$\,, i.e.\ $f$\,.
The ansatz $R_{\omega '}^1(v)=F_{\omega '}(r)\text{e}^{\text{i}\omega 'v}$ will work because (\ref{eq:edd-fink radial part}) is an ODE.
Since $v=t+r^*$\,, there is no need for $F$ to be dependent on time and we will define $F$ such that it solves the radial part of our equation.

We can go through the same argumentation for the modes $p$ and we will end up with the same answer, although the time and radial part will be solved by $R_{\omega }^2(u)=P_\omega (r)\text{e}^{\text{i}\omega u}$ and be valid on $\mathscr{I}^+$\,.
We thus have an expression for the modes $f$ and $p$ on $\mathscr{I}^-$ and $\mathscr{I}^+$ in the Schwarzschild metric respectively
\begin{align} 
  f_{\omega 'l m} &= \dfrac{1}{\sqrt{2\pi \omega '}}\dfrac{1}{r}F_{\omega '}(r)\text{e}^{\text{i}\omega 'v}Y_{l m}(\theta ,\phi )\\
  p_{\omega l m} &= \dfrac{1}{\sqrt{2\pi \omega }}\dfrac{1}{r}P_\omega (r)\text{e}^{\text{i}\omega u}Y_{l m}(\theta ,\phi )
\,.\end{align} 
These modes are given in \cite{Hawking1975} equation (2.11) and (2.12), and are valid throughout the whole spacetime, since they are completely determined by their respective data on $\mathscr{I}^\pm$\,.

As seen from (\ref{eq:beta squared}), the number of particles on $\mathscr{I}^-$ is non-zero for an observer at $\mathscr{I}^+$ who defines the modes as $p$ and vacuum $\ket{0_+}$\,.
We are therefore interested in the back-propagation of $p$ to $\mathscr{I}^-$ to gain information about the waves which arrive at $\mathscr{I}^+$\,, i.e.\ to relate $f$ to $p$\,.
In the Schwarzschild metric with gravitational collapse (i.e.\ no eternal black hole) there exists only the black hole event horizon at $r=r_S$\,.
Since the black hole is not eternal, there will be a moment in the past at a time $v_0$\,, where geodesics going into the collapsing body will be incomplete because there is no continuation to $i^+$ or $\mathscr{I}^+$, i.e.\ the geodesic will be inside the event horizon once it has formed;
we define this time as $v_0\equiv 0$ \footnote{This would be the white hole horizon in a maximally extended Schwarzschild metric}.
The modes of interest are the ones which arrive on $\mathscr{I}^-$ at a time $v\rightarrow v_0$ and went through the collapsing body on their way (the other part of the modes will just scatter back to $\mathscr{I}^-$ without going through the body).
To an observer on $\mathscr{I}^-$ the waves will be arbitrarily blueshifted as they close in on the event horizon from the future for $v\rightarrow v_0$ (as they would be redshifted if they were to come from $\mathscr{I}^-$ and reach an observer on $\mathscr{I}^+$).
In this case the frequency of the wave will be large compared to the local curvature such that it allows for the geometrical optics approximation of the wave.
This means that the wave can be described as a plane wave $p _{\omega l m}=P_0\text{e}^{\text{i}\omega u}$ where $u$ is a null hypersurface of constant phase.
$K=\partial _u$ is a Killing vector in the Eddington-Finkelstein metric, then at the horizon $r=2GM$ the Killing vector is null $\left.K^2\right|_{r=2GM}=\left.-\left(1-\tfrac{2GM}{r}\right)\right|_{r=2GM}=0$; the accuracy of the approximation increases as $v\rightarrow v_0$ justifying our approach \cite{Wald1984}.
The coordinate $u=u(\lambda )$ depends on the geodesic parameter $\lambda $\,, such that $K=\partial _u=\diff[]{\lambda }{u}\diff[]{}{\lambda }$\,.
If we consider geodesics parameterized by $u$\,, these would not be affinely parameterized, which results in a surface gravity since we are dealing with a Killing vector.
The non-affine parameterization can be expressed as a proportionality between two different tangent vectors, where the proportionality constant depends on $\lambda $ as in (\ref{eq:tangent vector proportionality}).
This means that
\begin{align} 
  \diffp[]{}{u(\lambda )}=\diff[]{\lambda }{u}\diff[]{}{\lambda }\equiv f \diff[]{}{\lambda }
\,.\end{align} 
and there exists a surface gravity defined by (\ref{eq:not affine geodesics}) such that
\begin{align} 
  \kappa =\partial _u \ln(f)\quad \Leftrightarrow \quad f=\pm f_0\text{e}^{\kappa u}
\,.\end{align} 
however $f=\diff[]{\lambda }{u}$ 
\begin{align} 
  \diff[]{\lambda }{u}=\pm f_0\text{e}^{\kappa u}\quad \Leftrightarrow \quad \lambda =\pm \text{e}^{\kappa u} \quad \Leftrightarrow \quad u=\dfrac{1}{\kappa }\ln (\mp \lambda )
\,.\end{align} 
upon choosing the constant $f_0=\kappa $ and setting the integration constant to zero \cite{Townsend1997}.
In the Schwarzschild metric $\kappa =\tfrac{1}{4M}$ \todo{could show appendix}.

$\mathscr{I}^-$ is a null hypersurface which is generated by null geodesics parameterized by the coordinate $v$ \todo{need show}.
We can therefore choose $v\propto \lambda |_{\mathscr{I}^-}$ on $\mathscr{I}^-$ and use this to parameterize our plane waves on $\mathscr{I}^-$\,.
Thereby
\begin{align} 
  p(v)=P_0\,\text{exp}\left(\text{i}\dfrac{\omega }{\kappa }\ln(-\alpha v)\right)\quad \text{on}\quad \mathscr{I}^- \quad \text{and}\quad v<0\label{eq:p on scri^-}
\,.\end{align} 
where $\alpha $ denotes the constant of proportionality.
This solution is only valid for $v<0$ since, on the one hand the logarithm is only defined on the positive real axis and on the other hand there exist no modes $p$ for $v>0$\,, since this means that $v>v_0$ for which no modes can escape to $\mathscr{I}^+$\,.

From (\ref{eq:bogolyubov transform 1}), $p$ is itself an expansion into modes $f$\,, therefore we can take the Fourier-transform of (\ref{eq:p on scri^-}) to obtain the decomposition w.r.t.\ the frequency $\sigma $
\begin{align} 
  \tilde{p}(\sigma )\equiv \mathcal{F}\left(p(v)\right)(\sigma ) &= \dfrac{1}{\sqrt{2\pi }}\int_{-\infty}^{0}\td v\,P_0\, \text{exp}\left(\text{i}\dfrac{\omega }{\kappa }\ln (-\alpha v)\right)\text{e}^{\text{i}v\sigma }\\
                                                                 &= \dfrac{P_0\alpha ^{i\tfrac{\omega }{\kappa } }}{\sqrt{2\pi }}\int_{-\infty}^{0}\td v\,\left(-v\right)^{\text{i}\frac{\omega }{\kappa }}\text{e}^{\text{i}v \sigma }\\
                                                                 \intertext{set $v=-v'\equiv -v$}
                                                                 &= \dfrac{P_0\alpha ^{i\tfrac{\omega }{\kappa } }}{\sqrt{2\pi }}\int_{0}^{\infty}\td v\,v^{\text{i}\frac{\omega }{\kappa }}\text{e}^{-\text{i}v \sigma }\\
                                                                 \intertext{to make this integral look similar to $\Gamma (z)=\int_{0}^{\infty}\td t\,t ^{z-1}\text{e}^{-t}$ set $v=-\tfrac{i}{\sigma }t$ with $\sigma =\Im(\sigma )$ such that $v \in \mathds{R}$ \todo{complex $\sigma $?}}
                                                                 &= \dfrac{P_0\alpha ^{\text{i}\tfrac{\omega }{\kappa } }}{\sqrt{2\pi }}\int_{0}^{\infty}\td t\,\left(-\dfrac{\text{i}}{\sigma }\right)v ^{\text{i}\tfrac{\omega }{\kappa }}\left(-\dfrac{\text{i}}{\sigma }\right)^{\text{i}\tfrac{\omega }{\kappa }}\text{e}^{-t}\\
                                                                 &= \dfrac{P_0\alpha ^{\text{i}\tfrac{\omega }{\kappa } }}{\sqrt{2\pi }}\left(-\text{i}\sigma \right)^{-1-\text{i}\tfrac{\omega }{\kappa }}\Gamma \left(1+\text{i}\dfrac{\omega }{\kappa }\right)\\
                                                                 \intertext{where $(-1)^{-1-\text{i}\tfrac{\omega }{\kappa }}=(-1)^{-1}(-1)^{-\text{i}\tfrac{\omega }{\kappa }}=-\left(\text{e}^{i \pi }\right)^{-\text{i}\tfrac{\omega }{\kappa }}=-\text{e}^{\pi \tfrac{\omega }{\kappa }}$}
                                                                 &= -\text{e}^{\pi \tfrac{\omega }{\kappa }}\dfrac{P_0\alpha ^{\text{i}\tfrac{\omega }{\kappa } }}{\sqrt{2\pi }}\left(\text{i}\sigma \right)^{-1-\text{i}\tfrac{\omega }{\kappa }}\Gamma \left(1+\text{i}\dfrac{\omega }{\kappa }\right)
\,.\end{align} 
The modes $\tilde{p}(\sigma )$ correspond to the positive frequency part of $f$ and analoguosly $\tilde{p}(-\sigma )$ to the negative frequency part of $f$\,.
The relation between these modes is 
\begin{align} 
  \tilde{p}(-\sigma )=-\text{e}^{-\pi \tfrac{\omega }{\kappa }}\tilde{p}(\sigma )
\,.\end{align} 
which is equation (14.3.6) in \cite{Wald1984}.
This can be directly converted to the appropriate Bogolyubov coefficients from (\ref{eq:bogolyubov transform 1}), where 
\begin{align} 
  \beta _{\omega \omega '}=-\text{e}^{-\pi \tfrac{\omega }{\kappa }}\alpha _{\omega \omega '}
\,.\end{align} 
This information about the coefficients suffices to calculate the expected number of particles (which is now a continuous integral) (\ref{eq:beta squared})
\begin{align} 
  \bra{0_-}b_{\omega }^\dagger b_{\omega }\ket{0_-}=\int_{0}^{\infty}\td \omega '\,|\beta _{\omega \omega '}|^2
\,.\end{align} 
The normalization condition (\ref{eq:bogolyubov coefficients norm}) is also continuous \todo{carroll different indices}
\begin{align} 
  \int_{0}^{\infty}\td k\,\left(\alpha _{\omega k}\overline{\alpha }_{\omega 'k}-\beta _{\omega k}\overline{\beta }_{\omega 'k}\right)=\delta _{\omega \omega '}
\,.\end{align} 
We will set $\omega =\omega '$ and relabel $k\equiv \omega '$ for consistency
\begin{align} 
  \Leftrightarrow \quad \int_{0}^{\infty}\td \omega '\,\left(|\alpha _{\omega '\omega }|^2-|\beta _{\omega '\omega }|^2\right) &= 1\\
  \Leftrightarrow \quad \int_{0}^{\infty}\td \omega '\,\left(\text{e}^{2\pi \tfrac{\omega }{\kappa }}-1\right)|\beta _{\omega '\omega }|^2 &= 1\\
  \Leftrightarrow \quad \int_{0}^{\infty}\td \omega '\,|\beta _{\omega '\omega }|^2 &= \dfrac{1}{\text{e}^{2\pi \tfrac{\omega }{\kappa }}-1} \label{eq:spectrum}
\,.\end{align} 
\todo{in hawking it is fraction} where the RHS of (\ref{eq:spectrum}) is a blackbody spectrum with temperature \cite{Hawking1975}
\begin{align} 
  T=\dfrac{\kappa }{2\pi }
\,.\end{align} 
Thus we have finally fixed the equation relating entropy and area of a black hole
\begin{align} 
  \dfrac{A}{4G}=S
\,.\end{align} 
\todo{black hole evaporation, area theorem not violated (carroll p.\ 417)}
%}}}

%{{{ Surface Gravity
\section{Surface Gravity}
A system in equilibrium has a constant temperature $T$\,.
In black hole mechanics, this translates to a constant surface gravity $\kappa $ on the horizon.

To compute $\kappa $\,, we have to start with the geodesic equation in non-affine parametrization
\begin{align} 
  t\indices{^\nu }\nabla \indices{_\nu }t\indices{_\mu }=\kappa t\indices{_\mu }
\,,\end{align} 
where $t$ is a Killing vector.
If we want to compute $\kappa $ \textit{on} the horizon, we need to choose $t$ such that it is orthogonal to the horizon \todo{?? or generator?}.
\begin{align} 
  t\indices{_[ _\rho }\nabla \indices{_\lambda _]}\left(t\indices{^\nu }\nabla \indices{_\nu }t\indices{_\mu }\right) &= t\indices{_[_\rho }\nabla \indices{_\lambda _]}\left(\kappa t\indices{_\mu }\right)\\
  \Leftrightarrow \left(t\indices{_[_\rho }\nabla \indices{_\lambda _]}t\indices{^\nu }\right)\nabla \indices{_\nu }t\indices{_\mu } 
  + t\indices{^\nu }\left(t\indices{_[_\rho }\nabla \indices{_\lambda _]}\nabla \indices{_\nu }t\indices{_\mu }\right) &= \left(t\indices{_[_\rho }\nabla \indices{_\lambda _]}\kappa \right)t\indices{_\mu }
  + \kappa \left(t\indices{_[_\rho }\nabla \indices{_\lambda _]}t\indices{_\mu }\right)
\,.\end{align} 
Term $(a)$ by Frobenius theorem and Killing eq cancles term $(d)$
\begin{align} 
  (a) = -\dfrac{1}{2}t\indices{^\nu }\left(\nabla \indices{_\rho }t\indices{_\lambda }\right)\left(\nabla \indices{_\nu }t\indices{_\mu } \right)
  = -\dfrac{1}{2}\kappa t\indices{_\mu }\left(\nabla \indices{_\rho }t\indices{_\lambda }\right)
  = \kappa \left(t\indices{_[_\rho }\nabla \indices{_\lambda _]}t\indices{_\mu }\right)
\,.\end{align} 
such that
\begin{align} 
  \Leftrightarrow t\indices{^\nu }\left(t\indices{_[_\rho }\nabla \indices{_\lambda _]}\nabla \indices{_\nu }t\indices{_\mu }\right) &= \left(t\indices{_[_\rho }\nabla \indices{_\lambda _]}\kappa \right)t\indices{_\mu }\\
  \Leftrightarrow t\indices{^\nu }\dfrac{1}{2}\left(t\indices{_\rho }\nabla \indices{_\lambda }\nabla \indices{_\nu }t\indices{_\mu }-t\indices{_\lambda }\nabla \indices{_\rho }\nabla \indices{_\nu }t\indices{_\mu }\right) &= \left(t\indices{_[_\rho }\nabla \indices{_\lambda _]}\kappa \right)t\indices{_\mu }\\
  \Leftrightarrow t\indices{^\nu }\dfrac{1}{2}\left(t\indices{_\rho }R\indices{_\mu _\nu _\lambda ^\sigma }t\indices{_\sigma }-t\indices{_\lambda }R\indices{_\mu _\nu _\rho ^\sigma }t\indices{_\sigma }\right) &= a\\
  \Leftrightarrow t\indices{^\nu }R\indices{_\mu _\nu _[_\lambda ^\sigma }t\indices{_\rho _]}t\indices{_\sigma } &= a\\
  \Leftrightarrow t\indices{^\nu }t\indices{_[_\rho }R\indices{_\lambda _]_\nu _\mu ^\sigma }t\indices{_\sigma } &= \left(t\indices{_[_\rho }\nabla \indices{_\lambda _]}\kappa \right)t\indices{_\mu } \label{eq:const surface gravity 1}\\
\end{align} 

Frobenius theorem
\begin{align} 
  t\indices{_[_\mu }\nabla \indices{_\nu }t\indices{_\rho _]} &= 0 \label{eq:frobenius}\\
  \Leftrightarrow \dfrac{1}{3!}\left(t\indices{_\mu }\nabla \indices{_\nu }t\indices{_\rho }+t\indices{_\nu }\nabla \indices{_\rho }t\indices{_\mu }+t\indices{_\rho }\nabla \indices{_\mu }t\indices{_\nu }-t\indices{_\mu }\nabla \indices{_\rho }t\indices{_\nu }-t\indices{_\nu }\nabla \indices{_\mu }t\indices{_\rho }-t\indices{_\rho }\nabla \indices{_\nu }t\indices{_\mu }\right) &= 0\\
  \intertext{using Killing's equation $\nabla \indices{_(_\mu }t\indices{_\nu _)} = 0$}
  \Leftrightarrow t\indices{_\mu }\nabla \indices{_\nu }t\indices{_\rho }+t\indices{_\nu }\nabla \indices{_\rho }t\indices{_\mu }+t\indices{_\rho }\nabla \indices{_\mu }t\indices{_\nu } &= 0 \label{eq:frobenius 2}\\
  \Leftrightarrow t\indices{_\mu }\nabla \indices{_\nu }t\indices{_\rho } &= -2t\indices{_[_\nu }\nabla \indices{_\rho _]}t\indices{_\mu } \label{eq:frobenius 3}
\end{align} 

To simplify further we can use $t\indices{_[ _\rho }\nabla \indices{_\lambda _]}$ on (\ref{eq:frobenius 3})
\begin{align} 
  t\indices{_[ _\rho }\nabla \indices{_\lambda _]}\left(t\indices{_\mu }\nabla \indices{_\nu }t\indices{_\sigma}\right) &= t\indices{_[ _\rho }\nabla \indices{_\lambda _]}\left(-2t\indices{_[_\nu }\nabla \indices{_\sigma _]}t\indices{_\mu }\right)\\
  \Leftrightarrow \left(t\indices{_[ _\rho }\nabla \indices{_\lambda _]}t\indices{_\mu }\right)\nabla \indices{_\nu }t\indices{_\sigma }+t\indices{_\mu }\left(t\indices{_[ _\rho }\nabla \indices{_\lambda _]}\nabla \indices{_\nu }t\indices{_\sigma }\right) &= -2\left(t\indices{_[ _\rho }\nabla \indices{_\lambda _]}t\indices{_[_\nu }\right)\nabla \indices{_\sigma _]}t\indices{_\mu }-2\left(t\indices{_[ _\rho }\nabla \indices{_\lambda _]}\nabla \indices{_[_\sigma }t\indices{_|_\mu _|}\right)t\indices{_\nu _]}\\
  \Leftrightarrow \left(t\indices{_[ _\rho }\nabla \indices{_\lambda _]}t\indices{_\mu }\right)\nabla \indices{_\nu }t\indices{_\sigma } +t\indices{_\mu }R\indices{_\sigma _\nu _[_\lambda ^\tau }t\indices{_\rho _]}t\indices{_\tau }&= -2\left(t\indices{_[ _\rho }\nabla \indices{_\lambda _]}t\indices{_[_\nu }\right)\nabla \indices{_\sigma _]}t\indices{_\mu }+2t\indices{_[_\nu }R\indices{_\sigma _]_\mu _[_\lambda ^\tau }t\indices{_\rho _]}t\indices{_\tau }
\end{align} 

Now calculate
\begin{align} 
  \left(t\indices{_[ _\rho }\nabla \indices{_\lambda _]}t\indices{_\mu }\right)\nabla \indices{_\nu }t\indices{_\sigma } &= -\dfrac{1}{2}\left(t\indices{_\mu }\nabla \indices{_\rho }t\indices{_\lambda }\right)\nabla \indices{_\nu }t\indices{_\sigma }=-\dfrac{1}{2}\left(t\indices{_\mu }\nabla \indices{_\nu }t\indices{_\sigma }\right)\nabla \indices{_\rho }t\indices{_\lambda }
\end{align} 
and
\begin{align} 
  -2\left(t\indices{_[ _\rho }\nabla \indices{_\lambda _]}t\indices{_[_\nu }\right)\nabla \indices{_\sigma _]}t\indices{_\mu } &= -\left(\left(t\indices{_[ _\rho }\nabla \indices{_\lambda _]}t\indices{_\nu }\right)\nabla \indices{_\sigma }t\indices{_\mu }-\left(t\indices{_[_\rho }\nabla \indices{_\lambda _]}t\indices{_\sigma }\right)\nabla \indices{_\nu }t\indices{_\mu }\right)\\
                                                                                                                               &= \dfrac{1}{2}\left(t\indices{_\nu }\nabla \indices{_\rho }t\indices{_\lambda }\right)\nabla \indices{_\sigma }t\indices{_\mu }-\dfrac{1}{2}\left(t\indices{_\sigma }\nabla \indices{_\rho }t\indices{_\lambda }\right)\nabla \indices{_\nu }t\indices{_\mu }\\
                                                                                                                               &= \dfrac{1}{2}\left(t\indices{_\nu }\nabla \indices{_\sigma }t\indices{_\mu }\right)\nabla \indices{_\rho }t\indices{_\lambda }-\dfrac{1}{2}\left(t\indices{_\sigma }\nabla \indices{_\nu }t\indices{_\mu }\right)\nabla \indices{_\rho }t\indices{_\lambda }\\
                                                                                                                               &= \dfrac{1}{2}\left(t\indices{_\nu }\nabla \indices{_\sigma }t\indices{_\mu }\right)\nabla \indices{_\rho }t\indices{_\lambda }+\dfrac{1}{2}\left(t\indices{_\sigma }\nabla \indices{_\mu }t\indices{_\nu }\right)\nabla \indices{_\rho }t\indices{_\lambda }
\end{align} 
Thereby with Frobenius' Theorem
\begin{align} 
   -\dfrac{1}{2}\left(t\indices{_\mu }\nabla \indices{_\nu }t\indices{_\sigma }\right)\nabla \indices{_\rho }t\indices{_\lambda }+t\indices{_\mu }R\indices{_\sigma _\nu _[_\lambda ^\tau }t\indices{_\rho _]}t\indices{_\tau }&= \dfrac{1}{2}\left(t\indices{_\nu }\nabla \indices{_\sigma }t\indices{_\mu }\right)\nabla \indices{_\rho }t\indices{_\lambda }+\dfrac{1}{2}\left(t\indices{_\sigma }\nabla \indices{_\mu }t\indices{_\nu }\right)\nabla \indices{_\rho }t\indices{_\lambda }+2t\indices{_[_\nu }R\indices{_\sigma _]_\mu _[_\lambda ^\tau }t\indices{_\rho _]}t\indices{_\tau }\\
   \Leftrightarrow t\indices{_\mu }R\indices{_\sigma _\nu _[_\lambda ^\tau }t\indices{_\rho _]}t\indices{_\tau } &= 2t\indices{_[_\nu }R\indices{_\sigma _]_\mu _[_\lambda ^\tau }t\indices{_\rho _]}t\indices{_\tau }+\dfrac{1}{2}\left(t\indices{_\mu }\nabla \indices{_\nu }t\indices{_\sigma }\right)\nabla \indices{_\rho }t\indices{_\lambda }+\dfrac{1}{2}\left(t\indices{_\nu }\nabla \indices{_\sigma }t\indices{_\mu }\right)\nabla \indices{_\rho }t\indices{_\lambda }+\dfrac{1}{2}\left(t\indices{_\sigma }\nabla \indices{_\mu }t\indices{_\nu }\right)\nabla \indices{_\rho }t\indices{_\lambda }\\
  \Leftrightarrow t\indices{_\mu }R\indices{_\sigma _\nu _[_\lambda ^\tau }t\indices{_\rho _]}t\indices{_\tau } &= 2t\indices{_[_\nu }R\indices{_\sigma _]_\mu _[_\lambda ^\tau }t\indices{_\rho _]}t\indices{_\tau }\\
  \Leftrightarrow t\indices{_\mu }R\indices{_\sigma _\nu _[_\lambda ^\tau }t\indices{_\rho _]}t\indices{_\tau } +t\indices{_\nu }R\indices{_\mu _\sigma _[_\lambda ^\tau }t\indices{_\rho _]}t\indices{_\tau }+t\indices{_\sigma }R\indices{_\nu _\mu _[_\lambda ^\tau }t\indices{_\rho _]}t\indices{_\tau } &= 0
\end{align} 
Now contract with $g\indices{^\mu ^\lambda }$ and use $t\indices{^\lambda }t\indices{_\lambda }=0$ because on $t$ is null on the horizon
\begin{align} 
  t\indices{_\mu }R\indices{_\sigma _\nu _[_\lambda ^\tau }t\indices{_\rho _]}t\indices{_\tau }g\indices{^\mu ^\lambda } &= t\indices{^\lambda }R\indices{_\sigma _\nu _[_\lambda ^\tau }t\indices{_\rho _]}t\indices{_\tau }=\dfrac{1}{2}t\indices{^\lambda }R\indices{_\sigma _\nu _\lambda ^\tau }t\indices{_\rho }t\indices{_\tau }=\dfrac{1}{2}t\indices{^\lambda }R\indices{_\sigma _\nu _\lambda _\tau }t\indices{_\rho }t\indices{^\tau }=0
\end{align} 
because $R\indices{_\sigma _\nu _\lambda _\tau }$ is antisymmetric and $t\indices{^\lambda }t\indices{^\tau }$ is symmetric in $\lambda $ and $\tau $\,.
The other term
\begin{align} 
  t\indices{_\nu }R\indices{_\mu _\sigma _[_\lambda ^\tau }t\indices{_\rho _]}t\indices{_\tau }g\indices{^\mu ^\lambda }+t\indices{_\sigma }R\indices{_\nu _\mu _[_\lambda ^\tau }t\indices{_\rho _]}t\indices{_\tau }g\indices{^\mu ^\lambda } &= \dfrac{1}{2}t\indices{_\nu }R\indices{_\sigma ^\tau }t\indices{_\rho }t\indices{_\tau }-\dfrac{1}{2}t\indices{_\nu }R\indices{_\lambda _\sigma _\rho ^\tau }t\indices{^\lambda }t\indices{_\tau }-\dfrac{1}{2}t\indices{_\sigma }R\indices{_\nu ^\tau }t\indices{_\rho }t\indices{_\tau }-\dfrac{1}{2}t\indices{_\sigma }R\indices{_\nu _\lambda _\rho ^\tau }t\indices{^\lambda }t\indices{_\tau }\\
                                                                                                                                                                                                                                                &= t\indices{_[_\nu }R\indices{_\sigma _]^\tau }t\indices{_\rho }t\indices{_\tau }+t\indices{_[_\nu }R\indices{_\sigma _]_\lambda _\rho ^\tau }t\indices{^\lambda }t\indices{_\tau }
\end{align} 
Therefore
\begin{align} 
  t\indices{_[_\nu }R\indices{_\sigma _]_\lambda _\rho ^\tau }t\indices{^\lambda }t\indices{_\tau }=-t\indices{_[_\nu }R\indices{_\sigma _]^\tau }t\indices{_\rho }t\indices{_\tau }
\end{align} 
This can be used in (\ref{eq:const surface gravity 1})
\begin{align} 
   \left(t\indices{_[_\rho }\nabla \indices{_\lambda _]}\kappa \right)t\indices{_\mu } &= t\indices{^\nu }t\indices{_[_\rho }R\indices{_\lambda _]_\nu _\mu ^\sigma }t\indices{_\sigma }\\
                                                                                       &= -t\indices{_[_\rho }R\indices{_\lambda _]^\sigma }t\indices{_\sigma }t\indices{_\mu }\\
   \Leftrightarrow t\indices{_[_\rho }\nabla \indices{_\lambda _]}\kappa &= -t\indices{_[_\rho }R\indices{_\lambda _]^\sigma }t\indices{_\sigma }
\,.\end{align} 
The Ricci tensor depends on the matter content in the universe at hand.
In the simple case of vacuum, $R\indices{_\mu _\nu }=0$ and the equation is fulfilled trivially.
However, we need not be so strict.
We can use Einstein's equation, so solve for $R\indices{_\mu _\nu }$
\begin{align} 
g\indices{^\nu ^\sigma }\left(R\indices{_\mu _\nu }-\dfrac{1}{2}g\indices{_\mu _\nu }R\right) &= g\indices{^\nu ^\sigma }8\pi G\,T\indices{_\mu _\nu }\\
 \Leftrightarrow R\indices{_\mu ^\sigma } -\dfrac{1}{2}\delta \indices*{_\mu ^\sigma }R &= 8\pi G\,T\indices{_\mu ^\sigma }
\,.\end{align} 
Now
\begin{align} 
  t\indices{_[_\rho }\nabla \indices{_\lambda _]}\kappa  &= \dfrac{1}{2}\left(-t\indices{_\rho }8\pi G\,T\indices{_\lambda ^\sigma }t\indices{_\sigma }-\dfrac{1}{2}t\indices{_\rho }\delta \indices*{_\lambda ^\sigma }R\,t\indices{_\sigma }+t\indices{_\lambda }8\pi G\,T\indices{_\rho ^\sigma }t\indices{_\sigma }+\dfrac{1}{2}t\indices{_\lambda }\delta \indices*{_\rho ^\sigma }R\,t\indices{_\sigma }\right)\\
                                                         &= -8\pi G\,t\indices{_[_\rho }T\indices{_\lambda _]^\sigma }t\indices{_\sigma }
\,.\end{align} 
Whether the equation yields zero, depends on the conditions imposed on the energy-momentum tensor.
Consider Raychaudhuri's equation \todo{discuss in appendix}
\begin{align} 
  \diff[]{\theta }{\lambda } &= -\dfrac{1}{2}\theta ^2-\hat{\sigma }\indices{_\lambda _\sigma }\hat{\sigma }\indices{^\lambda ^\sigma }+\hat{\omega }\indices{_\lambda _\sigma }\hat{\omega }\indices{^\lambda ^\sigma }-R\indices{_\lambda _\sigma }t\indices{^\lambda }t\indices{^\sigma }
\,.\end{align} 
For null geodesic congruences generating the horizon of a static black hole (which is the case for every black hole which has eventually settled in a state of equilibrium) the expansion, shear and rotation are zero \todo{cf \cite{Hawking1972}}, yielding
\begin{align} 
  0=R\indices{_\lambda _\sigma }t\indices{^\lambda }t\indices{^\sigma }=8\pi G\,T\indices{_\lambda _\sigma }t\indices{^\lambda }t\indices{^\sigma }+\dfrac{1}{2}g\indices{_\lambda _\sigma }Rt\indices{^\lambda }t\indices{^\sigma }=8\pi G\, T\indices{_\lambda _\sigma }t\indices{^\lambda }t\indices{^\sigma }
\,.\end{align} 
This means, that $T\indices{_\lambda ^\sigma }t\indices{_\sigma }$ either points in the direction of $t\indices{_\lambda }$ or is spacelike orthogonal \todo{show this?} to it.
However, by the (Null) Dominant Energy Condition ((N)DEC), the quantity $T\indices{_\lambda ^\sigma }t\indices{_\sigma }$ is a future-pointing non-spacelike vector, thus it has to be parallel to $t\indices{_\lambda }$\,.
Now, $t\indices{_[_\rho }T\indices{_\lambda _]^\sigma }t\indices{_\sigma }=\tfrac{1}{2}\left(t\indices{_\rho }T\indices{_\lambda ^\sigma }t\indices{_\sigma }-t\indices{_\lambda }T\indices{_\rho ^\sigma }t\indices{_\sigma }\right)$\,.
Since $T\indices{_\lambda ^\sigma }t\indices{_\sigma }\propto t\indices{_\lambda }$\,, then $\left(t\indices{_\rho }t\indices{_\lambda }-t\indices{_\lambda }t\indices{_\rho }\right)=0$ and thereby $t\indices{_[_\rho }T\indices{_\lambda _]^\sigma }t\indices{_\sigma }=0$\,.
This leads to
\begin{align} 
  \boxed{t\indices{_[_\rho }\nabla \indices{_\lambda _]}\kappa =0}
\,.\end{align} 
The surface gravity is constant tangential to or on the event horizon of the black hole.
We have thus shown that the zeroth law of thermodynamics is recovered from black hole mechanics.
It should be noted that the zeroth law only holds, if the (N)DEC is imposed on the energy-momentum tensor, while the other laws hold for any form of $T$ \todo{true?}.
%}}}

%{{{ Laws of Black Hole Thermodynamics
\section{Laws of Black Hole Thermodynamics}
\texttt{quick list of the laws of black hole thermodynamics.} 
\textit{Carroll} 

\begin{enumerate}[label=\arabic*.]
  \item Stationary black hole in equilibrium.
\end{enumerate}

\begin{enumerate}[label=\arabic*.]
  \item see Penrose-process.
\end{enumerate}

\begin{enumerate}[label=\arabic*.]
  \item The generalized second law as by Bekenstein.
\end{enumerate}

\begin{enumerate}[label=\arabic*.]
  \item The surface gravity never vanishes.
    Although in extremal black holes this is not the case.
\end{enumerate}
 
%}}}

%{{{ Information Loss Paradox
\section{Information Loss Paradox}
\texttt{complex systems such as a star can collapse to a black hole with only the no-hair theorem (i.e.\ less information). in classical gr the information (in the complexity of the star) can just be behind the event horizon without problem. but in gr with qft the black hole evaporates, which would destroy the information inevitably such that the total entropy of the universe will decrease.
also if two such very different stars would collapse to the same sort of black hole which would decay into indistinguishable thermal radiation the information about the difference of the stars is lost.
this is a violation of the unitarity of quantum mechanics.} 
\textit{Carroll} 
\begin{enumerate}[label=\arabic*.]
  \item Discussion of the information loss paradox as explained above.
  \item Outlook as an active research topic and the need to further understand the compatibility of gravity and quantum mechanics.
\end{enumerate}
%}}}

\clearpage

%{{{ Notes
\section*{Notes}
\begin{enumerate}[label=--]
  \item Townsend "Killing Horizons" definition of surface gravity.
\end{enumerate}
%}}}

%{{{ Questions
\section*{Questions}
\begin{enumerate}[label=--]
  \item irreducible mass (carroll p.\ 270) and area theorem no clear derivation? \textbf{stability of black hole $\delta J<\delta M$ mass irr area} \textbf{area shrinks but doesnt violate theorem since the condition is not vaild anyway (weak energy condition) $\rightarrow $ gets hotter with shrinking horizon}
  \item does the event horizon imply a singularity? i.e.\ since all matter has to converge. \textbf{yes} 
  \item does non-singular event horizon mean that it can be crossed? 
    since non-singular would imply that the hypersurface is geodesic complete. \textbf{yes also $g_{00}$ höhenlinien therefore no other gravitating source} 
\end{enumerate}
\vspace{3cm}
\begin{enumerate}[label=--]
  \item hawking paper derivation of $p_\omega ^{(2)}$ townsend formel (2.49) (geodesic imcompleteness (2.41))
  \item townsend: geodesic not affinely parameterized if $l\rightarrow f \xi $? but this is not the parameter and nevertheless reparametrization $\lambda \rightarrow \alpha \lambda +\beta $ is still affine. \textbf{the following does not leave the geodesic affine $\lambda \rightarrow f(\lambda )\lambda $ because the function $f(\lambda )$ has to be respected in the derivative. this is equivalent to $l\rightarrow f(\lambda )\xi $\,.} 
\end{enumerate}
\vspace{3cm}
\begin{enumerate}[label=--]
  \item in event horizon cannot $r=\text{const}$? see todo
  \item which Killing vector to use (for e.g.\ surface gravity).
    at $r_+$ $K=\partial _t$ null.
  \item hawking: byrrel davis qft in curved spacetime. kommutator zeigen []=0
\end{enumerate}
%}}}

%\end{multicols*}

\clearpage
%\listoffigures
%\listoftables
%\bibliographystyle{plain} % alpha apalike
%\bibliography{refs}
\printbibliography

%}}}

\end{document}
