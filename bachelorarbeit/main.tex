\documentclass[a4paper,10pt]{article}

%{{{ packages
\usepackage{blindtext}
\usepackage{lipsum} % lorem ipsum
\usepackage[english]{babel} % correct typesetting
\usepackage{csquotes} % correct quotation
\usepackage[tiny]{titlesec} % section title size
\usepackage{index}
%\usepackage[onehalfspacing]{setspace} % line spacing
\usepackage[headings]{fullpage} % makes document wider
\usepackage{pdfpages} % \includepdf
\usepackage{fancyhdr} % fancy headers
\usepackage{authblk} % author formatting
\usepackage{hyperref} % "clickable" references in text
\hypersetup{colorlinks=true,allcolors=blue}
\usepackage[backend=biber,style=numeric]{biblatex}
\addbibresource{refs.bib}

\usepackage{booktabs} % toprule,midrule,bottomrule
\usepackage{multirow} % multiple rows in tabular
\usepackage{multicol} % multiple columns in tabular
\usepackage{longtable}
\usepackage{graphicx}
\usepackage{svg} % \includesvg{}
\usepackage{tikz}
\usepackage[european,siunitx]{circuitikz}
\usepackage{import} % can import files from other directories
\usepackage{pgfplots}
\usepackage{gnuplottex}
\usepackage{wrapfig} % let text wrap around figure

\usepackage{amsmath,amsfonts,amssymb}
\usepackage{tensor} % for right index order 
\usepackage{dsfont} % double stroke font
\usepackage{cancel} % to cancel in fraction
\usepackage{bm} % bold math font (if error is produced use \bm{{}})
\usepackage{mathtools}
\usepackage[ISO]{diffcoeff} % differentiation
\usepackage[locale=DE]{siunitx}
\usepackage[official]{eurosym} % euro symbol
\usepackage{mathrsfs} % calligraphy
\usepackage{physics}
\usepackage[a]{esvect} % vector arrows
\usepackage{bigints} % big integrals
%\usepackage[frak=esstix]{mathalpha} % disable if LaTeX uses too many alphabets
\usepackage[ruled,vlined,linesnumbered]{algorithm2e}
\usepackage{listings}

\usepackage{hyphsubst}
\usepackage[]{caption} % [figurename=, tablename=]
\usepackage{xr} % crossreferencing between documents \externaldocument{}
\usepackage{enumitem} % for enumerate environment
\usepackage{lineno}
\usepackage{makecell}
\usepackage{xcolor}
\usepackage{color}

\allowdisplaybreaks % allows equations to be broken (e.g. by multicols)

\usetikzlibrary{arrows}
\pgfplotsset{compat=1.15}

\newcommand{\td}{\,\text{d}}
%\newcommand{\RN}[1]{\uppercase\expandafter{\romannumeral#1}}
\newcommand{\zz}{\mathrm{Z\kern-.3em\raise-0.5ex\hbox{Z} }}
\newcommand{\id}{\mathds{1}}

\newcommand\inlineeqno{\stepcounter{equation}\ {(\theequation)}}
\newcommand\inlineeqnoa{(\theequation.\text{a})}
\newcommand\inlineeqnob{(\theequation.\text{b})}
\newcommand\inlineeqnoc{(\theequation.\text{c})}

\newcommand\inlineeqnowo{\stepcounter{equation}\ {(\theequation)}}
\newcommand\inlineeqnowoa{\theequation.\text{a}}
\newcommand\inlineeqnowob{\theequation.\text{b}}
\newcommand\inlineeqnowoc{\theequation.\text{c}}

\renewcommand{\refname}{Source}
\renewcommand{\sfdefault}{phv}

\newcommand{\todo}[1]{\textcolor{red}{TODO: #1}}

% for multicols figures and then use \captionof{figure}{}
\iffalse
\newenvironment{Figure}
  {\par\medskip\noindent\minipage{\linewidth}}
  {\endminipage\par\medskip}
\fi

\sloppy % block text

\numberwithin{equation}{section}

\titleformat{\subsection}{}{\thesubsection}{1em}{\itshape}
\titleformat{\subsubsection}{}{\thesubsubsection}{1em}{\itshape}

%}}}
\usepackage{newtxtext}
\usepackage{newtxmath}

\begin{document}

%{{{ Titelseite

\begin{titlepage}
  %\title{\scalebox{.8}[3]{Thermodynamics of Black Holes}}
  \title{Thermodynamics of Black Holes}
  \author{Jonas Wortmann}
  \affil{\vspace{2cm}Bachelorarbeit in Physik angefertigt im\\\vspace{1cm}Physikalischen Institut\\\vspace{1cm}vorgelegt der Mathematisch-Naturwissenschaftlichen Fakultät der\\\vspace{1cm}Rheinischen Friedrich-Wilhelms-Universität Bonn\\\vspace{2cm}}
\end{titlepage}

\maketitle
\pagenumbering{gobble}

%\renewcommand\abstractname{Abstract}
%\abstract{\noindent}

\clearpage

\vspace*{\fill}
\noindent Ich versichere, dass ich diese Arbeit selbstständig verfasst und keine anderen als die angegebenen
Quellen und Hilfsmittel benutzt sowie die Zitate kenntlich gemacht habe.\\\\\\
\begin{minipage}{0.45\textwidth}
  Bonn,\hrule\par\vspace{1mm}
  Datum
\end{minipage}\hfill
\begin{minipage}{0.45\textwidth}
  \phantom{Bonn,}\hrule\par\vspace{1mm}
  Unterschift
\end{minipage}\\\\
\begin{enumerate}[label=\arabic*]
  \item GutachterIn:
  \item GutachterIn:
\end{enumerate}

%}}}

\clearpage

%{{{ Inhaltsverzeichnis

\renewcommand*\contentsname{Contents}
\tableofcontents

%}}}

\clearpage

\pagestyle{fancy}

\fancyhead[R]{\leftmark}
%\fancyhead[R]{\leftmark\\\rightmark}
\fancyhead[L]{\thepage}
\fancyfoot[C]{}

\pagenumbering{arabic}

%{{{

%\begin{multicols}{2}

%{{{ Introduction
\section{Introduction}
\texttt{}

%}}}

%{{{ Theoretical Background 
\section{Theoretical Background}

%{{{ Einstein Field Equation / Black Holes
\subsection*{Einstein Field Equation / Black Holes}
Einstein's field equations relates the geometry of spacetime to the energy and matter of the bodies which move through it
\begin{align} 
  R\indices{_\mu _\nu }-\dfrac{1}{2}g\indices{_\mu _\nu }R +\Lambda g\indices{_\mu _\nu }&= \kappa T\indices{_\mu _\nu } \label{eq:efe}
,\end{align} 
where 
\begin{align} 
  R\indices{^\rho _\mu _\sigma _\nu } &= \partial_\sigma \Gamma _{\nu \mu }^\rho -\partial_\nu \Gamma _{\sigma \mu }^\rho +\Gamma _{\sigma \lambda }^\rho \Gamma _{\nu \mu }^\lambda -\Gamma _{\nu \lambda }^\rho \Gamma _{\sigma \mu }^\lambda & R\indices{^\lambda _\mu _\lambda _\nu } &= R\indices{_\mu _\nu } & g\indices{^\mu ^\nu }R\indices{_\mu _\nu }=R\indices{^\mu _\mu } &= R \label{eq:riemann tensor}
,\end{align} 
is the Riemann tensor, the Ricci tensor and the Ricci scalar; $g_{\mu \nu }$ is the metric, $T_{\mu \nu }$ the stress-energy tensor, $\Lambda $ the cosmological constant and $\kappa =\tfrac{8\pi G}{c^4}$.
It is an equation of motion for the metric tensor, which dictates the path test particles take by the geodesics equation
\begin{align} 
  \diff[2]{x^\mu }{\lambda }+\Gamma _{\rho \sigma}^\mu \diff[]{x^\rho }{\lambda }\diff[]{x^\sigma }{\lambda } &= 0 \label{eq:geodesics equation}
,\end{align} 
where $\lambda $ is the affine parameter and $\Gamma $ is the Christoffel symbol.
Since we will be dealing with the vacuum space around a gravitating mass, we are only interested in the the solutions of the metric which fulfill the equation for $T_{\mu \nu }=\Lambda =0$, s.t.\
\begin{align} 
  R_{\mu \nu }-\dfrac{1}{2}g_{\mu \nu }R &= R_{\mu \nu }-\dfrac{1}{2}g_{\mu \nu }g^{\rho \sigma }R_{\rho \sigma } = 0\\
  R_{\mu \nu } &= 0 \label{eq:efe vacuum}
.\end{align} 
The equation (\ref{eq:efe vacuum}) is the Einstein field equation in vacuum, restricting the possible solutions of the metric to be Ricci flat.

%{{{ Schwarzschild
\subsection*{Schwarzschild}
In \cite{Schwarzschild1916} an analytic solution to (\ref{eq:efe vacuum}) found by Schwarzschild is presented, which is given by\footnote{The form is taken from \cite{Carroll}. A derivation is also given in the book.}
\begin{align} 
  ds^2 = g_{\mu \nu }dx^\mu dx^\nu  &= -\left(1-\dfrac{2GM}{r}\right)\td t^2+\left(1-\dfrac{2GM}{r}\right)^{-1}\td r^2+r^2\td \Omega ^2 \label{eq:schwarzschild metric}
,\end{align} 
where $c=1$, $\td \Omega ^2=\td \theta ^2+\sin ^2\theta \td \phi ^2$ and $\left\{t,r,\theta ,\phi \right\}$ are the spherical coordinates.
In SI units the Schwarzschild radius is given by
\begin{align} 
  r_S &= \dfrac{2GM}{c^2} \label{eq:schwarzschild radius}
,\end{align} 
where $M$ is the mass of the gravitating body, defined by the weak field limit.
Upon inspecting the metric we can notice two values of $r$ for which $\td s^2$ diverges.
The $r=r_S$ divergence is a mere coordinate dependent phenomenon.
Although this singularity can be lifted, it is still of physical significance\footnote{As opposed to be thought by Schwarzschild himself, who regarded the singularity as an artifact of the analytical solution \cite{Schwarzschild1916}.}, as it characterizes the event horizon of a black hole (this will be explained in more detail later).
The singularity at $r=0$ is however not liftable, since $R\indices{^\mu ^\nu ^\rho ^\sigma }R\indices{_\mu _\nu _\rho _\sigma }=\tfrac{48G^2M^2}{r^6}$, which is a coordinate independent scalar, has a singularity at that point.
\todo{motivation / reason for introduction}
%}}}

%{{{ Reissner-Nordström
\subsection*{Reissner-Nordström}
The analytical solution of the Einstein field equation in vacuum for a gravitating and electrically charged mass is given by the Reissner-Nordström metric with the magnetic charge $P=0$ \cite{Carroll} \cite{Reissner1916}
\begin{align} 
  \td s^2 &= -\left(1-\dfrac{2GM}{r}+\dfrac{GQ^2}{r^2}\right)\td t^2+\left(1-\dfrac{2GM}{r}+\dfrac{GQ^2}{r^2}\right)^{-1}\td r^2+r^2\td \Omega ^2 \label{eq:reissner-nordström metric}
,\end{align} 
where $Q$ is the electric charge of the body.
In the case of a neutral mass the solution reduces to the Schwarzschild metric (\ref{eq:schwarzschild metric}). 
%}}}

%{{{ Kerr
\subsection*{Kerr}
For a rotating mass (\ref{eq:efe vacuum}) was solved by Kerr \cite{Carroll} \cite{Kerr1961}
\begin{multline} 
  \td s^2 = -\left(1-\dfrac{2GMr}{r^2+\tfrac{J^2}{M^2}\cos ^2\theta }\right)\td t^2
  -\dfrac{2GMr\tfrac{J}{M}\sin ^2\theta }{r^2+\tfrac{J^2}{M^2}\cos ^2\theta }\left(\td t\td \phi +\td \phi \td t\right)
  +\dfrac{r^2+\tfrac{J^2}{M^2}\cos ^2\theta }{r^2-2GMr+\tfrac{J^2}{M^2}}\td r^2\\
  +\left(r^2+\dfrac{J^2}{M^2}\cos ^2\theta  \right)\td \theta ^2
  +\dfrac{\sin ^2\theta }{r^2+\tfrac{J^2}{M^2}\cos ^2\theta }\left( \left(r^2+\dfrac{J^2}{M^2}\right)^2-\dfrac{J^2}{M^2}\left(r^2-2GMr+\dfrac{J^2}{M^2}\right)\sin ^2\theta \right)\td \phi ^2 \label{eq:kerr metric}
,\end{multline} 
where $J$ is the total angular momentum of the spinning mass.
%}}}

%{{{ Kerr-Newman
\subsection*{Kerr-Newman}
Upon replacing $2GMr\rightarrow 2GMr-GQ^2$ in (\ref{eq:kerr metric}) we arrive at the solution of (\ref{eq:efe vacuum}) for a rotating and charged mass \cite{Carroll}
\begin{multline} 
  \td s^2 = -\left(1-\dfrac{2GMr-GQ^2}{r^2+\tfrac{J^2}{M^2}\cos ^2\theta }\right)\td t^2
  -\dfrac{(2GMr-GQ^2)\tfrac{J}{M}\sin ^2\theta }{r^2+\tfrac{J^2}{M^2}\cos ^2\theta }\left(\td t\td \phi +\td \phi \td t\right)
  +\dfrac{r^2+\tfrac{J^2}{M^2}\cos ^2\theta }{r^2-2GMr+GQ^2+\tfrac{J^2}{M^2}}\td r^2\\
  +\left(r^2+\dfrac{J^2}{M^2}\cos ^2\theta  \right)\td \theta ^2
  +\dfrac{\sin ^2\theta }{r^2+\tfrac{J^2}{M^2}\cos ^2\theta }\left( \left(r^2+\dfrac{J^2}{M^2}\right)^2-\dfrac{J^2}{M^2}\left(r^2-2GMr+GQ^2+\dfrac{J^2}{M^2}\right)\sin ^2\theta \right)\td \phi ^2 \label{eq:kerr-newman metric}
,\end{multline} 
\todo{are the metrics really needed??}
%}}}

%{{{ Black Holes
\subsection*{Black Holes}
As we have briefly said, in (\ref{eq:schwarzschild metric}) there exists a coordinate singularity at $r=r_S=2GM$, which we denoted as the event horizon of a Schwarzschild black hole.
In general, we will call objects for which the Schwarzschild metric holds even at $r=2GM$ black holes.

Let us now investigate the behaviour of light cones at this particular surface.
Imagine a null ray closing in on $r_S$ for $\td \Omega =0$
\begin{align} 
  \td s^2 = 0 &= -\left(1-\dfrac{r_S}{r}\right)\td t^2+\left(1-\dfrac{r_S}{r}\right)^{-1}\td r^2\\
  \diff[]{r}{t} &= 1-\dfrac{r_S}{r}
.\end{align} 
For $r\rightarrow r_S$ 
\begin{align} 
  \lim_{r\rightarrow r_S}\diff[]{r}{t}=0\quad\Leftrightarrow \quad \lim_{r\rightarrow r_S}\diff[]{t}{r}=\infty
.\end{align} 
The light ray will cease to make progress towards the surface at $r_S$; at least in this coordinate system.
To cut short the derivation \cite{Carroll} we will introduce the Kruskal coordinates $\left\{T,R,\theta ,\phi \right\}$ which relate to $\left\{r,t,\theta ,\phi \right\}$ via 
\begin{align} 
  T &= \,\sqrt[]{\left(\dfrac{r}{r_S}-1\right)}\text{e}^{\tfrac{r}{2r_S}}\sinh \left(\dfrac{t}{2r_S}\right) &
  R &= \,\sqrt[]{\left(\dfrac{r}{r_S}-1\right)}\text{e}^{\tfrac{r}{2r_S}}\cosh \left(\dfrac{t}{2r_S}\right)
.\end{align} 
The metric becomes
\begin{align} 
  \td s^2 &= \dfrac{4r_S^3}{r}\text{e}^{-\tfrac{r}{r_S}}\left(-\td T^2+\td R^2\right)+r^2\td \Omega ^2
,\end{align} 
where $r$ has to be implicitely defined\footnote{The solution for $r$ is the Lambert-W function $r=r_S\left(W\left(\tfrac{R^2-T^2}{\text{e}}\right)+1\right)$.} by $R^2-T^2=\left(1-\tfrac{r}{r_S}\right)\text{e}^{r/r_S}$.
In these coordinates the light cones are at $45^\circ$ everywhere,\todo{Are $T$ and $R$ valid time and space coordinates which can be indentified with time and space in the general sense?}
\begin{align} 
  \diff[]{R}{T} &= \pm 1
.\end{align} 
This will therefore also hold at $r=r_S$; the event horizon of the black hole is therefore a null hypersurface.\todo{rigorous definition in townsend.}
Thus, for any light cone crossing this hypersurface, all future directed paths will point towards the center of the black hole.
But, since the light cones behave as they would in flat space, it is no problem for an observer to cross the event horizon in a finite amount of their proper time (although to an outside observer, the infalling observer will still only reach it asymptotically).

The singularity at $r=0$ is however not liftable.
Such a singularity is called a spacetime singularity and is characterized by geodesic incompleteness.
That is, if there exists a geodesic curve, which cannot be continued to all values of the affine parameter, then there exists a spacetime singularity, such that it is a manifest property of the manifold (i.e.\ it cannot be lifted by a mere coordinate transform) \cite{Townsend1997}.\todo{explicit calculation (short)}

Black holes in general are only described by a small number of properties.
Israel \cite{Israel1967} \cite{Israel1968} was the first to deduce this property by showing that, for a static, asymptotically flat spacetime \todo{simply connected equipotential lines $g_{00}$; important?} the Schwarzschild metric (\ref{eq:schwarzschild metric}) is the only metric for which the event horizon is regular (i.e.\ is non-singular\todo{see question}).
This characterizes the solution by Schwarzschild to be unique.
By further analysis he also deduced that for a vacuum, in which the source-free Maxwell equations hold (all charges are said to be within the black hole), only the Reissner-Nordström metric (\ref{eq:reissner-nordström metric}) is a solution with regular event horizon.
Thus it is also unique.
Analoguously Robinson \cite{Robinson1975} proved the uniqueness of the Kerr metric (\ref{eq:kerr metric}) and Mazur \cite{Mazur1982} proved the uniqueness of the Kerr-Newman metric (\ref{eq:kerr-newman metric}) for their respective vacua.
In a general form, we can write this as a no-hair theorem or uniqueness theorem (formulation taken literally from \cite[][p.\ 238]{Carroll}).
\enquote{Stationary, asympotically flat black hole solutions to general relativity coupled to electromagnetism that are non-singular outside the event horizon are fully characterized by the parameters of mass, electric and magnetic charge, and angular momentum.}.
The impact of this theorem might not be as important right now, but it will be later when discussing its implications in the light of the information content a black hole can have (or the way two black holes can be quite similar although their respective collapsing star was not).

Two important theorems are left to mention.
The first is the still unproven \enquote{cosmic censorship conjecture} by Penrose \cite{Penrose1969} which says, that there exists no naked singularity, i.e.\ a singularity not surrounded by an event horizon, which resulted by means of gravitational collapse\footnote{It is thought that the singularity from which our universe originated could in fact be a naked singularity \cite{Hawking1972}.}.
This statement is reasonable to accept, because it would verify that the classical theory of general relativity, which breaks down at the singularity, does not so in fact, because the information, which a singularity holds, is sealed behind the event horizon, thus diminishing its impact on the theory, since it could not escape anyway \cite{Hawking1972}.

The second is the area theorem by Hawking \cite{Hawking1972}.
It states, under the assumption of the weak energy condition and the cosmic censorship conjecture that the area of a future event horizon in an asymptotically flat spacetime cannot decrease \cite{Carroll} \cite{Hawking1972}.
%}}}
%}}}

%{{{ Thermodynamics
\subsection*{Thermodynamics}
\texttt{laws of thermodynamics.} 
\begin{enumerate}[label=\arabic*.]
  \item All four laws of thermodynamics.
  \item Notion of reversible and irreversible processes and the quantity entropy.
\end{enumerate}
%}}}

%}}}

%{{{ Geroch's Gedanken Experiment
\section{Geroch's Gedanken Experiment}
\texttt{gerochs heat engine and the need for entropy / a non decreasing quantity for bh.
first conjectured by bekenstein.}
\textit{ferrari:\url{https://arxiv.org/pdf/gr-qc/0505008}, bekenstein:\url{https://doi.org/10.1007/BF02757029}} 
\begin{enumerate}[label=\arabic*.]
  \item Gerochs heat engine:
    An observer lowers a mass $m$ on an ideal string to the event horizon of a black hole.
    The body will then radiate energy $\Delta m$ into the black hole.
    Since its energy as measured from inifnity is zero, it will remain unchanged after the body radiates into the black hole.
    After that the mass $m-\Delta m$ is pulled back up and has done a total work of $\Delta m$ on the observer.
    This work however has completely been converted into heat thus violating the second law.
  \item This process only works in a reversible fashion, since otherwise a portion of the work of the body would have gone into the irreversible process.
  \item This needs for a irreversibly increasing quantity of the black hole, analoguous to the entropy.
  \item A good candidate is in this case is the area, see the area theorem.
    Therefore the area could be proportional to the black hole entropy.
\end{enumerate}
%}}}

%{{{ Penrose Process
\section{Penrose Process}
\texttt{penrose process to extract work from bh.
area theorem.
formula for area analog to 1st hs (can area <-> entropy?).
not sure abt constants only the equation could hold.} 
\textit{kerr:\url{https://doi.org/10.1103/PhysRevLett.11.237}, penrose:\url{https://doi.org/10.1038/physci229177a0}, carroll}
\begin{enumerate}[label=\arabic*.]
  \item To further see a link between thermodynamics and black hole dynamics one can also consider the Penrose-process.
    This leads to the analog of the first law of thermodynamics and a more definite relation between the entropy and area of a black hole.
    Although at this point only the ratio is known.
\end{enumerate}
%}}}

%{{{ Unruh effect and Hawking Radiation
\section{Unruh effect and Hawking Radiation}
\texttt{introduction with unruh effect (qft in curved spacetime carroll) derivation / argumentation for hawking radiation. need for an event horizon since then there is no timelike killing vector throughout the whole spacetime s.t.\ the modes match minkowski.
no violation of the area theorem due to hawking radiation (see carroll p.\ 417).
identification of S - A and T - kappa.} 
\textit{carroll, unruh:\url{https://doi.org/10.1103/PhysRevD.14.870}} 
\begin{enumerate}[label=\arabic*.]
  \item Upon taking a look at quantum field theory in curved space time, we observe the Unruh effect.
  \item With the equivalence principle, the acceleration of such a Rindler observer is indistinguishable from gravity.
    Therefore the acceleration can be associated with the surface gravity of the black hole, which leads to the black hole being a black body.
  \item The black body temperature is the temperature of the radiation produced by the black hole called Hawking radiation.
  \item With a formula for the temperature the entropy for the black hole can be defined properly.
\end{enumerate}
%}}}

%{{{ Laws of Black Hole Thermodynamics
\section{Laws of Black Hole Thermodynamics}
\texttt{quick list of the laws of black hole thermodynamics.} 
\textit{Carroll} 

\begin{enumerate}[label=\arabic*.]
  \item Stationary black hole in equilibrium.
\end{enumerate}

\begin{enumerate}[label=\arabic*.]
  \item see Penrose-process.
\end{enumerate}

\begin{enumerate}[label=\arabic*.]
  \item The generalized second law as by Bekenstein.
\end{enumerate}

\begin{enumerate}[label=\arabic*.]
  \item The surface gravity never vanishes.
    Although in extremal black holes this is not the case.
\end{enumerate}
 
%}}}

%{{{ Information Loss Paradox
\section{Information Loss Paradox}
\texttt{complex systems such as a star can collapse to a black hole with only the no-hair theorem (i.e.\ less information). in classical gr the information (in the complexity of the star) can just be behind the event horizon without problem. but in gr with qft the black hole evaporates, which would destroy the information inevitably such that the total entropy of the universe will decrease.
also if two such very different stars would collapse to the same sort of black hole which would decay into indistinguishable thermal radiation the information about the difference of the stars is lost.
this is a violation of the unitarity of quantum mechanics.} 
\textit{Carroll} 
\begin{enumerate}[label=\arabic*.]
  \item Discussion of the information loss paradox as explained above.
  \item Outlook as an active research topic and the need to further understand the compatibility of gravity and quantum mechanics.
\end{enumerate}
%}}}

\clearpage

%{{{ Notes
\section*{Notes}
\begin{enumerate}[label=--]
  \item Townsend "Killing Horizons" definition of surface gravity.
\end{enumerate}
%}}}

%{{{ Questions
\section*{Questions}
\begin{enumerate}[label=--]
  \item how is $\mathcal{H}^+$ event horizon different from $\mathscr{I}^+$ conformal null infinity. the world line should just be able to cross (see german wikipedia penrose diagram)?
  \item in the collapsing star penrose diagram why is $i^+$ not directly above $i^-$? are the constant $r$ lines sill valid?
  \item energy at infintiy $E=m(1-2\tfrac{M}{r})^{1/2}$ in bekenstein. derivation?
  \item irreducible mass (carroll p.\ 270) and area theorem no clear derivation?
  \item does the event horizon imply a singularity? i.e.\ since all matter has to converge.
  \item does non-singular event horizon mean that it can be crossed? 
    since non-singular would imply that the hypersurface is geodesic complete.
\end{enumerate}
%}}}

%\end{multicols}

\clearpage
%\listoffigures
%\listoftables
%\bibliographystyle{plain} % alpha apalike
%\bibliography{refs}
\printbibliography

%}}}

\end{document}
