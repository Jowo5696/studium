\externaldocument{main}
\newpage
\section{Thermodynamik}
\subsection{Temperaturskala (Kelvin)}
Die Fixpunkte der Kelvinskala liegen bei
\begin{enumerate}[label=--]
        \item $\SI{0}{K}$: Absoluter Nullpunkt; das System ist enthält keine kinetische Energie mehr.
        \item $\SI{273,16}{K}\,\text{und}\,\SI{613}{Pa}$: Tripelpunkt von Wasser (\ref{fig:phasendiagramme}), der Zustand bei dem alle Phasen von Wasser miteinander im Gleichgewicht stehen.
\end{enumerate}
Der Tripelpunkt ist in dem Phasendiagramm dargestellt, welches eine gängige Methode ist, die Phase eines Stoffes oder Systems mit Hilfe des Drucks und der Temperatur darzustellen.

\subsection{Wärmeausdehnung von Festkörpern}
Stoffe dehnen sich bei Wärmezufuhr aus, da das \textsc{Lennard}--\textsc{Jones}--Potential eine asymmetrische Form hat. 
Wird Energie in Form von Wärme zu einem Stoff hinzugefügt, erhöht sich die kinetische Energie der Moleküle und so auch die Schwingungsampiltude.
Da das \textsc{Lennard}--\textsc{Jones}--Potential asymmetrisch ist, erhöht sich der mittlere Abstand zwischen den Molekülen.\\\indent
Bei genügend Wärme findet ein Phasenübergang statt.
Da die Form des \textsc{Lennard}--\textsc{Jones}--Potential eine Material-- bzw.\ Stoffeigenschaft ist, haben unterschiedliche Materialien unterschiedlich starke Ausdehnung.\\\indent
Die Länge des Materials nach der Ausdehnung ist gegeben als 
\begin{align} 
        l&=l_0+l_0\alpha _V\Delta T=l_0\left(1+\alpha _l\Delta T\right)&\dfrac{\alpha _l^a}{\alpha _l^b}&=\text{const.}
.\end{align} 
Analog gilt für das Volumen (auch das Gesetz von \textsc{Gay--Lussac})
\begin{align} 
        V&=V_0+V_0\alpha _V\Delta T=V_0\left(1+\alpha _V\Delta T\right)&\dfrac{\alpha _V^a}{\alpha _V^b}&=\text{const.}
.\end{align} 
Für ideale Gase gilt sogar
\begin{align} 
        \dfrac{\alpha _V^a}{\alpha _V^b}=1
.\end{align} 

\subsection{Ideale Gase / \textsc{Van--der--Waals}--Gase}
Ein ideales Gas besteht aus ausdehnungslosen Massenpunkten, belegen also in ihrem Raum kein Volumen.
Zudem sind die Teilchen frei und üben keine Wechselwirkung aufeinander aus.
Ideale Gasteilchen rotieren bzw.\ vibrieren nicht; die Energie ist ausschließlich durch ihre kinetische Energie gegeben.
Ein reales Gas ist näherungsweise ein ideales Gas, wenn es sich weit von seinen Phasenübergängen befindet.\\\\
Das Gesetz von \textsc{Gay--Lussac} besagt
\begin{align} 
        V\left(p_0,\Delta T\right)=V\left(p_0,\Delta T\right)\left(1+\alpha \Delta T\right)
.\end{align} 
Das Gesetz von \textsc{Boyle--Mariotte} besagt
\begin{align} 
        pV\left(p,\Delta T\right)&=p_0V\left(p_0,\Delta T\right)
.\end{align} 
Es folgt also das \textsc{Boyle--Mariotte--Gay--Lussac} Gesetz
\begin{align} 
        pV&=p_0V\left(p_0,\Delta T\right)=p_0V_0\left(1+\alpha \Delta T\right)
.\end{align} 
Für ideale Gase ist $\alpha =\left(\SI{273,15}{\celsius}\right)^{-1}$ bei einer Temperatur von $T=\SI{0}{\celsius}$.\\\indent
Es foglt das ideale Gasgesetz für ein System aus $n$ mol Teilchen
\begin{align} 
        pV&=nRT&R&=\dfrac{p_NV_M}{T_{>0}}
,\end{align} 
mit $V_M$ dem Molvolumen und $p_N$ dem Normaldruck.\\\indent
Für ein reales Gas gilt allerdings näherungsweise
\begin{align} 
        \left(p+\dfrac{an^2}{V^2}\right)\left(V-b\right)&=nRT
,\end{align} 
mit $a$ dem Binnendruck, hervorgerufen durch den Druck der einzelnen Moleküle, und $b$ dem Kovolumen, dem Eigenvolumen der Moleküle.

\subsection{Wärme}
Wärme ist der Teil der Energie, der von einem thermodynamischen System aufgenommen oder abgegeben wird.
Es gibt verschiedene Arten des Wärmeaustausch.
\begin{enumerate}[label=--]
        \item Wärmestrahlung: Wärme wird in der Form von elektromagnetischen Wellen aufgrund von Molekülschwingungen abgestrahlt.
        \item Konduktion: Wärme wird an der Grenze zwischen zwei Oberflächen in Richtung der kälteren Oberfläche abgegeben.
        \item Konvektion: Wärme wird von einem kälteren strömenden Material aufgenommen, indem dieses durch seinen Bewegung immer wieder mit seiner kalten Oberfläche Wärme abtransportieren kann.
\end{enumerate}

\subsection{\textsc{Maxwell}--\textsc{Boltzmann}--Verteilung}
Die \textsc{Maxwell--Boltzmann}--Verteilung beschreibt die statistische Geschwindigkeitsverteilung eines idealen Gases. 
Wichtig dabei ist, dass die Verteilung eine lang auslaufende Kurve besitzt.
Es können also Gasmoleküle existieren, die noch kinetische Energie haben, obwohl der peak der Kurve bei einer Geschwindigkeit von null liegt.\\\indent
Die kinetische Energie von idealen Gasen ist über ihre Freiheitsgrade gegeben
\begin{align} 
        E_\text{kin}&=\dfrac{f}{2}kT
.\end{align} 

\subsection{1.\ Hauptsatz der Wärmelehre}
Der erste Hauptsatz der Thermodynamik besagt, dass die Energie in einem abgeschlossenen System konstant bleibt.
\begin{align} 
        \td U=\td Q-p\td V
.\end{align} 
Es existiert kein Perpetuum Mobile erster Art, welches mehr Energie produziert, als mechanisch oder über Wärmezuführ hineingegeben.

\subsection{2.\ Hauptsatz der Wärmelehre}
Der zweite Hauptsatz der Thermodynamik besagt, dass Wärme stets von einem wärmeren Objekt zu einem kälteren Objekt und nie von sich selbst auch umgekehrt, fließt.\\\indent
Es existiert kein Perpetuum Mobile zweiter Art, welches Wärmeenergie aus einer kälteren Umgebung gewinnen kann.

\subsection{Kreisprozesse / Zustandsänderung}
Kreisprozesse bezeichnen eine Abfolge von Zustandsänderungen eines Arbeitsmediums (Gas, Dampf, Fluid) im Zustandsraum (Druck, Temperatur, Volumen).
Das Medium durchläuft ausschließlich Zustände, die im thermodynamischen Gleichgewicht liegen.

\begin{enumerate}[label=--]
        \item isobar: Konstanter Druck.
        \item isochor: Konstantes Volumen.
        \item isotherm: Konstante Temperatur.
        \item adiabatisch: Kein Wärmeaustausch mit der Umgebung. Ein adiabatisch reversibler Prozess ist immer isentrop, die Umkehrung allerdings nicht.
        \item Isentrop: Konstante Entropie.
        \item Isenthalp: Konstante Enthalpie.
\end{enumerate}
Der Wirkungsgrad eines System ist definiert als
\begin{align} 
        \eta:=\dfrac{E_\text{nutz}}{E_\text{zugef}}
.\end{align} 
Er bezeichnet die Effizienz eines Systems.

\subsubsection{\textsc{Carnot}--Prozess}
Der \textsc{Carnot}--Prozess ist ein Kreisprozess, der einen reversiblen Prozess zur Umwandlung von Wärme in Arbeit darstellt (\ref{fig:carnotprozess_im_pv_diagramm}). 
Das $T$--$S$--Diagramm ist ein Rechteck, da der adiabatische Prozess isentrop verläuft.\\\indent
Der Wirkungsgrad des \textsc{Carnot}--Prozess ist definiert durch die höchste ($T_h$) und niedrigste ($T_n$) im Prozess auftretende Temperatur
\begin{align} 
        \eta:=\dfrac{T_h-T_n}{T_h}
.\end{align} 

\subsection{Entropie, Enthalpie}
Die Entropie beschreibt das Maß an Chaos in einem System.
Sie steigt durch verschiedene thermodynamische Prozesse (Wärmeleitung, Diffusion, Erzeugung von Reibnungswärme, chemische Reaktionen, $\hdots$) immer mit der Zeit.
Der Gleichgewichtszustand eines Systems ist dann erreicht, wenn die Entropie am größten ist; dann bleibt die Entropie konstant.
Die Entropie kann nie vernichtet werden.
Ein Prozess bei dem Entropie entsteht, kann nicht rückgängig gemacht werden, ohne, dass die entstandene Entropie an die Umgebung abgegeben wird.\\\indent
Die Entropie kann definiert werden über
\begin{align} 
        S&:=k_B\ln \Omega &\delta S&:=\dfrac{\delta Q}{T}
,\end{align} 
mit $\Omega $ dem Phasenraumvolumen ($\td \Omega =\td^{3N}q\td^{3N}p$) wobei ($\vv{p},\vv{q}$) den Mikrozustand eines Teilchens im $6N$--dimensionalen Raum angibt und $\delta Q$ der dem System bei einer Temperatur $T$ zugeführten Wärme.
Wird an dem System nur mechanische Arbeit durch Volumenänderung verrichtet, ändert sich die Entropie nicht.\\\indent 
Die Enthalpie ist eine systemspezifische Rechengröße, dargestellt als Summe aus innerer Energie und der Volumenarbeit
\begin{align} 
        H:=U+pV
.\end{align} 
Die \glqq Bruttoenergie\grqq{}, die einem System zugeführt werden muss, ist gegeben als die Änderung der Energie $\td U$ und die durch die isobare Volumenarbeit $p\td V$ verbrauchte Energie
\begin{align} 
        \td H=\td U+p\td V
.\end{align} 
