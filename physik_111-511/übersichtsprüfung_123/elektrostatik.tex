\newpage
\section{Elektrostatik}
\subsection{Makroskopische Elektrostatik}

\subsection{Bändermodell / Leiter}
Das Bändermodell wird verwendet, um die quantenmechanischen Energieniveaus der Elektronen in einem Kristall darzustellen.
Dabei existiert das Valenz-- und Leitungsband. 
Im Valenzband sind die Elektronen an die Atomrümpfe gebunden.
Im Leitungsband können sich die Elektronen frei durch den Kristall bewegen.\\\indent
Betrachtet man einen Leiter, so befindet sich das Leitungsband unmittelbar über dem Valenzband; es muss also nur eine verschwindent geringe Energie aufgebracht werden, um die Elektronen vom Valenzband in das Leitungsband zu befördern. 
Leiter sind im Allgemeinen kaltleiter, da sich bei zu hohen Temperaturen zu viele Elektronen im Leitungsband befinden.\\\indent
In Halbleitern muss eine gewisse Energie aufgebracht werden, damit Elektronen vom Valenzband in das Leitungsband gelangen.
Halbleiter sind im Allgemeinen heißleiter, da sich bei hohen Temperaturen ausreichend viele Elektronen im Leitungsband befinden.\\\indent
Bei Nichtleitern ist die Energie zwischen Valenz-- und Leitungsband so hoch, dass es für die Elektronen im Allgemeinen nicht möglich ist in das Leitungsband zu gelangen.

\subsection{\textsc{Maxwell}--Gleichungen}
Die \textsc{Maxwell}--Gleichungen sind
\begin{align} 
        \text{div}\vv{E}&=-\dfrac{\rho }{\varepsilon _0}&\text{div}\vv{B}&=0\\
        \vec{\text{rot}}\vv{E}&=-\partial_t \vv{B}&\vec{\text{rot}}\vv{B}&=\mu _0\vv{j}-\dfrac{1}{c^2}\partial_t \vv{E}
.\end{align} 

\subsection{\textsc{Hertz}'scher Dipol}
Der \textsc{Hertz}'sche Dipol ist ein langer gerader Draht, in dem Ladung zwischen den Enden schwingen.
Er beruht darauf, dass die Enden des Drahtes einen Kondensator bilden, dessen Feldlinien ähnlich den Feldlinien eines Dipols sind und, dass der Draht selbst ein konzentrisches Magnetfeld aufbaut.
Durch die oszillierende entstehung des elektrischen und magnetischen Feldes sendet der Dipol elektromagnetische Strahlung.\\\indent
Seine Ursprungsform ist ein elektrischer Schwingkreis mit Spule und Plattenkondensator.

\subsection{Kondensator}
Ein Kondensator besteht aus zwei unterschiedlich geladenen Polen, welcher Energie eines Gleichstromkreises zwischen den Platten mit Hilfe eines elektrischen Feldes speichert.
Die Capazität eines Kondensators ist gegeben durch
\begin{align} 
        C=\dfrac{Q}{U}
.\end{align} 
Die Energie
\begin{align} 
        E=\dfrac{1}{2}CU^2
.\end{align} 
Für einen Plattenkondensator gelten folgende Formeln
\begin{align} 
        C&=\varepsilon \dfrac{A}{d}&E&=\dfrac{Q}{\varepsilon A}
,\end{align} 
mit $d$ dem Plattenabstand und $A$ der Fläche, die die Platten gegenseitig durchsetzen.
