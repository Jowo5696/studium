\newpage
\section{Optik}
\subsection{Polarisation}
Die Polarisation einer Welle beschreibt die Richtung der Schwingung.
Für elektromagnetische Wellen bilden $\vv{E}$--, $\vv{B}$--Feld und Wellenvektor $\vv{k}$ ein orthogonales System, also $\vv{E}\perp\vv{B}\perp\vv{k}$.
\begin{enumerate}[label=--]
        \item Linearpolarisierung: Eine Welle, bei der die Richtung der Schwingung konstant ist.
        \item zirkulare Polarisierung: Eine Welle, bei der der Betrag der Schwingung konstant ist, die Richtung sich aber zirkular ändert.
        \item elliptische Polarisierung: Eine Wellen, bei der sich der Betrag und und Richtung der Schwingung elliptisch ändert.
\end{enumerate}

\subsection{Interferometer}
\subsection{Linsen}
\subsection{Resonator}
\subsection{Phasen-- und Gruppengeschwindigkeit}
Die Phasen-- und Gruppengeschwindigkeit einer Welle ist gegeben durch
\begin{align} 
        v_\text{Ph}&=\dfrac{\omega }{k}&v_\text{Gr}=\diffp[]{\omega }{k}
.\end{align} 

\subsection{Dispersion}
Dispersion beschreibt die wellenlängenabhängige Änderung des Brechungsindex, also
\begin{align} 
        A=\diff[]{n}{\lambda }
.\end{align} 
Dispersion kann z.B.\ für Spektroskopie verwendet werden:
Ein paralleler Lichtstrahl trifft in einem Winkel auf eine Seite eines Prismas.
In diesem Prisma tritt Dispersion auf, also wird Licht mit einer geringen Wellenlänge nicht so stark abgelenkt, wie Licht mit einer größeren Wellenlänge.
Dadurch lassen sich z.B.\ einzelne Emissionslinien aufteilen und beobachten.

\subsection{\textsc{Snellius}'sche Brechnungsgesetz}
Das \textsc{Snellius}'sche Brechungsgesetz gibt an, unter welchem Winkel Licht an der Grenzschicht zwischen zwei Medien gebrochen wird
\begin{align} 
        n_1\sin \alpha =n_2\sin \beta 
,\end{align} 
mit $\alpha $ dem Eintritts-- und $\beta $ dem Austrittwinkel.\\\indent
Die Herleitung dieses Gesetzes folgt aus dem \textsc{Fermat}'schen Prinzip, welches besagt, dass Licht immer den kürzesten Weg durch ein Medium wählt. 
Allgemeiner das Prinzip der stationären Wirkung.\\\indent
Quantenmechanisch betrachtet ist das Licht gleichzeitig auf allen Wegen zu finden, allerdings löschen sich die Wege, die nicht dem kürzesten entsprechen, aus.

\subsection{Kohärenz}
Kohärenz beschreibt die Fähigkeit einer Welle mit einer anderen Welle destruktiv oder konstruktiv zu Überlagern.\\\indent
Eine Welle ist zeitlich Kohärent, wenn sie über ein Zeitinterval auf vorhersagbare Weise schwingt.
Eine höhere Kohärenzzeit entspricht einer längeren Zeit, in der das Licht interferieren kann.
Zeitliche Kohärenz ist dann relevant, wenn Licht zu einer zeitlich verschobenen Kopie ihrer selbst kohärent sein soll.
Der Wellenzug muss dann so lange auf vorhersagbare Weise schwingen, dass das kohärente Licht noch die andere Quelle erreichen kann.\\\indent
Eine Welle ist räumlich Kohärent, wenn sie mit einer räumlich verschobenen Quelle noch interferieren kann.
Wie weit diese andere Quelle weg sein darf, beschreibt die Größe des räumlichen Kohärenzgebietes.

\subsection{Doppelspalt}
Ein Doppelspalt kann verwendet werden, um die Wellenlänge eines Lichtbündels zu berechnen.
Es gilt die Gleichung für konstruktive Interferenz
\begin{align} 
        d\sin \alpha &=m\lambda=g
,\end{align} 
mit $d$ dem Spaltabstand / Gitterkonstante, $\alpha $ dem Winkel des Interferenzmaximums, $m \in \mathbb{N}$ der Zahl des Maximums und $g$ dem Gangunterschied.
Für destruktive Interferenz werden die korrespondierenden Größen für $m \in \mathbb{N}+\tfrac{1}{2}$ verwendet. 
Bei der Beugung ist die Verteilung der Extremstellen genau andersherum, also ist für $m \in \mathbb{N}$ ein Minimum und für $m \in \mathbb{N}+\tfrac{1}{2}$ ein Maximum.\\\indent
Die Herleitung dieser Formel folgt aus der Geometrie des Doppelspaltes und aus der Bedingung, dass für konstruktive bzw.\ desktruktive Interferenz die Wellenberge und --täler überlagern, bzw.\ sich auslöschen müssen.

\subsection{Totalreflexion}

\subsection{\textsc{Brewster}--Winkel}
