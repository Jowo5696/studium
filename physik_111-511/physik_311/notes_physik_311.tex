%:LLPStartPreview
%:VimtexCompile(SS)

%{{{ Formatierung

\documentclass[a4paper,12pt]{article}

\usepackage{physics_notetaking}

%%% dark red
%\definecolor{bg}{RGB}{60,47,47}
%\definecolor{fg}{RGB}{255,244,230}
%%% space grey
%\definecolor{bg}{RGB}{46,52,64}
%\definecolor{fg}{RGB}{216,222,233}
%%% purple
%\definecolor{bg}{RGB}{69,0,128}
%\definecolor{fg}{RGB}{237,237,222}
%\pagecolor{bg}
%\color{fg}

\newcommand{\td}{\,\text{d}}
\newcommand{\RN}[1]{\uppercase\expandafter{\romannumeral#1}}
\newcommand{\zz}{\mathrm{Z\kern-.3em\raise-0.5ex\hbox{Z} }}

\newcommand\inlineeqno{\stepcounter{equation}\ {(\theequation)}}
\newcommand\inlineeqnoa{(\theequation.\text{a})}
\newcommand\inlineeqnob{(\theequation.\text{b})}
\newcommand\inlineeqnoc{(\theequation.\text{c})}

\newcommand\inlineeqnowo{\stepcounter{equation}\ {(\theequation)}}
\newcommand\inlineeqnowoa{\theequation.\text{a}}
\newcommand\inlineeqnowob{\theequation.\text{b}}
\newcommand\inlineeqnowoc{\theequation.\text{c}}

\renewcommand{\refname}{Source}
\renewcommand{\sfdefault}{phv}
%\renewcommand*\contentsname{Contents}

\pagestyle{fancy}

\sloppy

\numberwithin{equation}{section}

%}}}

\begin{document}

%{{{ Titelseite

\title{physik311 $|$ Notizen}
\author{Jonas Wortmann}
\maketitle
\pagenumbering{gobble}

%}}}

\newpage

%{{{ Inhaltsverzeichnis

\fancyhead[L]{\thepage}
\fancyfoot[C]{}
\pagenumbering{arabic}

\tableofcontents

%}}}

\newpage

%{{{

\fancyhead[R]{\leftmark\\\rightmark}

%{{{ Einführung
\section{Einführung}
Licht ist eine elektromagnetische Welle. Die Wellenlänge ist im Bereich von $\SI{400}{nm}$ bis $\SI{800}{nm}$, das entspricht einer Frequenz von $\SI{750}{THz}$ bis $\SI{375}{THz}$. Ein Lichtpuls kann nie kürzer als ein Zyklus sein.

\subsection{Lichtquellen}
\begin{enumerate}[label=--]
        \item Lampe: Inkoherentes (\glqq ungeordnetes\grqq{}) Licht
        \item Laser: Koherentes (\glqq geordnetes\grqq{}, auch Wellen in \glqq gleichschritt\grqq{}) Licht
\end{enumerate}

\newpage
\section{Die elektromagnetische Theorie des Lichts}
Für diesen Fall betrachtet man nur die Lichtausbreitung in großer Entfernung von allen Quellen. Also ist $\rho =0$ und $\vv{j}=0$. Die Maxwell--Gleichungen sind dann
\begin{align}
        \,\text{div}\,\vv{D}&=0\\
        \,\text{div}\,\vv{B}&=0\\
        \,\text{rot}\,\vv{E}&=-\diffp[]{\vv{B}}{t}\\
        \,\text{rot}\,\vv{H}&=\diffp[]{\vv{D}}{t}
.\end{align}
In Materialien gilt dann
\begin{align} 
        \vv{D}&=\varepsilon \varepsilon _0\vv{E}\\
        \vv{B}&=\mu \mu _0\vv{H}
.\end{align} 
Hier ist $\varepsilon $ die Dielektrizitätskonstante und $\mu $ die relative Permeabilität (in der Optik ist sie üblicherweise 1). \\\indent
\begin{align} 
        \,\text{rot}\,\left(\,\text{rot}\,\vv{E}\right)&=\underbrace{\,\text{div}\,\left(\,\text{div}\,\vv{E}\right)}_{=0}-\left(\vv{\nabla }\cdot \vv{\nabla }\right)\vv{E}&&|\left(\vv{\nabla }\cdot \vv{\nabla }\right)=\,\text{div}\,\,\text{grad}\,=\triangle\\
                                                       &=\,\text{rot}\,\left(-\diffp[]{\vv{B}}{t}\right)
.\end{align} 
Mit $\,\text{rot}\,\vv{B}=\varepsilon\mu \varepsilon _0 \mu _0\diffp[]{\vv{E}}{t}$ folgt
\begin{align} 
        \triangle \vv{E}&=\varepsilon \mu \varepsilon _0\mu _0\diffp[2]{\vv{E}}{t}
.\end{align} 
Dies ist die \textbf{Wellengleichung} für das elektrische Feld. Man erwartet eine Ausbreitungsgeschwindigkeit mit
\begin{align} 
        v_{\,\text{ph}\,}&=\dfrac{1}{\,\sqrt[]{\varepsilon \mu \varepsilon _0\mu _0}}\equiv \dfrac{c}{n}
,\end{align} 
wobei $c=\,\sqrt[]{\varepsilon _0\mu _0}^{-1}$ die Vakuumslichtgeschwindigkeit und $n=\,\sqrt[]{\varepsilon \mu }$ der Brechungsindex ist. Dann lässt sich die Wellengleichung wie folgt schreiben
\begin{align} 
        \triangle \vv{E}&=\dfrac{1}{v_{\,\text{ph}\,}^2}\diffp[2]{\vv{E}}{t}\stackrel{\,\text{Vakuum}\,}{=}\dfrac{1}{c^2}\diffp[2]{\vv{E}}{t}
.\end{align} 
Eine analoge Rechnung kann auch für das $\vv{B}$--Feld verwendet werden.\\\indent

\subsection{Einfachste Lösung der Wellengleichung: Ebene elektromag.\ Welle}
Hier wird die Lichtausbreitung nur entlang einer Koordinate (z.B.\ $z$) betrachtet. Also ist $\vv{E}\left(\vv{r},t\right)=\vv{E}\left(z,t\right)$, bzw.\ $\diffp[]{\vv{E}}{x}=\diffp[]{\vv{E}}{y}=0$. Die Wellengleichung vereinfacht sich dann zu
\begin{align} 
        \diffp[2]{\vv{E}}{z}=\dfrac{1}{c^2}\diffp[2]{\vv{E}}{t}
.\end{align} 
Mit $\,\text{div}\,\vv{E}=0$ folgt für ebene Wellen $\diffp[]{E}{z}=0$, also ist $E_z=\,\text{const.}\,$.\\\indent
Jetzt wählt man die Randbedingungen, dass $E_z=0$. Also ist $\vv{E}=\begin{pmatrix}
        E_x\left(z\right)\\E_y\left(z\right)\\0
\end{pmatrix}$. Die Lösung der Wellengleichung ist dann
\begin{align} 
        \vv{E}\left(z,t\right)&=\vv{E}_0\cos \left(kz-\omega t\right)\\
                              &=\vv{E}_0\cos \left(k\left(z-ct\right)\right)
,\end{align} 
wobei $\tfrac{\omega }{k}=c$, mit $k=\tfrac{2\pi }{\lambda }$ der \textbf{Wellenzahl} ($\lambda $ der Wellenlänge) und $E_0$ der Amplitude. 



%}}}

%}}}

\end{document}
