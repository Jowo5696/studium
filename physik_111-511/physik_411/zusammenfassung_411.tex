%{{{ Formatierung

\documentclass[a4paper,12pt]{article}

\usepackage{physics_notetaking}

%%% dark red
%\definecolor{bg}{RGB}{60,47,47}
%\definecolor{fg}{RGB}{255,244,230}
%%% space grey
%\definecolor{bg}{RGB}{46,52,64}
%\definecolor{fg}{RGB}{216,222,233}
%%% purple
%\definecolor{bg}{RGB}{69,0,128}
%\definecolor{fg}{RGB}{237,237,222}
%\pagecolor{bg}
%\color{fg}

\newcommand{\td}{\,\text{d}}
\newcommand{\RN}[1]{\uppercase\expandafter{\romannumeral#1}}
\newcommand{\zz}{\mathrm{Z\kern-.3em\raise-0.5ex\hbox{Z} }}

\newcommand\inlineeqno{\stepcounter{equation}\ {(\theequation)}}
\newcommand\inlineeqnoa{(\theequation.\text{a})}
\newcommand\inlineeqnob{(\theequation.\text{b})}
\newcommand\inlineeqnoc{(\theequation.\text{c})}

\newcommand\inlineeqnowo{\stepcounter{equation}\ {(\theequation)}}
\newcommand\inlineeqnowoa{\theequation.\text{a}}
\newcommand\inlineeqnowob{\theequation.\text{b}}
\newcommand\inlineeqnowoc{\theequation.\text{c}}

\renewcommand{\refname}{Source}
\renewcommand{\sfdefault}{phv}
%\renewcommand*\contentsname{Contents}

\pagestyle{fancy}

\sloppy

\numberwithin{equation}{section}

%}}}

\begin{document}

%{{{ Titelseite

\title{physik411 $|$ Zusammenfassung}
\author{Jonas Wortmann}
\maketitle
\pagenumbering{gobble}

%}}}

\newpage

%{{{ Inhaltsverzeichnis

\fancyhead[L]{\thepage}
\fancyfoot[C]{}
\pagenumbering{arabic}

\tableofcontents

%}}}

\newpage

%{{{

\fancyhead[R]{\leftmark\\\rightmark}

\newpage
\section{Quantenmechanik}
\subsection{Quantenzahlen}
\begin{enumerate}[label=--]
        \item $n  \in \mathbb{N}$ Hauptquantenzahl (Energie: $E_n=-\tfrac{13.6}{n^2}\,\text{eV}$)
        \item $l  \in \left[0,n-1\right]$ Drehimpulsquantenzahl () (Summe: $L$)
        \item $m_l  \in \left[-l,l\right]$ magnetische Quantenzahl (Energie: Magnetfeld)
        \item $s = \tfrac{1}{2}$ Spin Elektron (Energie) (Summe: $S$)
        \item $m_s = \pm \tfrac{1}{2}$ magnetische Spinquantenzahl ()
        \item $j = l \pm s$ Gesamtdrehimpulsquantenzahl (Energie: Feinstruktur)
        \item $J  \in \left[|L-S|,L+S\right]$ Summe der Gesamtdrehimpulsquantenzahlen ()
        \item $m_j  \in \left[-j,j\right]$ magnetische Quantenzahl (Energie)
        \item $I = 1\lor \tfrac{1}{2}$ Kernspin für ganzzahlige oder halbzahlige Nukleonen ()
        \item $F  \in \left[|J-I|,J+I\right]$ Gesamtdrehimpulsquantenzahl des Atoms (Energie: Hyperfeinstruktur)
\end{enumerate}
\subsection{Gyromagnetische Konstante}
\subsection{Entartung}
Ein Energieeigenwert ist entartet, wenn mehrere Zustände existieren, die den selben Energieeigenwert besitzen.

\subsection{Normaler Zeeman--Effekt}
Der Normale \textsc{Zeeman}--Effekt beschreibt die Aufspaltung der $n$ Energieniveaus durch ein externes Magnetfeld für ein System mit Gesamtspin gleich null.

\subsection{Spin--Orbit--Kopplung / Feinstruktur}
Durch die Spin--Orbit--Kopplung entsteht ein magnetisches Moment, welches die $n$ Energieniveaus weiter in $m_l$ Energieniveaus aufteilt.

\subsection{Hyperfeinstruktur}
Durch die WW der Elektronen mit dem elektrischen Dipol-- und magnetischen Quadrupolmoment des Kerns werden die $m_l$ Energieniveaus weiter in $F$ Energieniveaus aufgeteilt.

\subsection{Hybridisierung}

\newpage
\section{Experimente}
\subsection{Stern--Gerlach}
\subsection{Zeeman}

\newpage
\section{Festkörperphysik}
\subsection{Miller--Indizes}
\subsection{Bloch--Funktion}
\subsection{Elementarzellen}

\newpage
\section{Formelsammlung}
\begin{align} 
        \lambda =\dfrac{h}{p}
\end{align} 

\subsection{Operatoren}
\begin{align} 
        \hat{p}&=-\text{i}\hbar \partial\\
        \hat{L}=\hat{x}\times \hat{p}&=-\text{i}\hbar x\times \partial\\
        \hat{L}_z&=-\text{i}\hbar \partial_\varphi \left(\text{Kugelkoordinaten}\right)\\
        \left[\hat{L}_i,\hat{L}_j\right]&=\sum_{k}^{}\text{i}\hbar \varepsilon _{ijk}\hat{L}_k\\
        \hat{H}&=\dfrac{\hat{p}}{2m}+\hat{V}\left(\hat{x}\right)\\
        \Delta \hat{A}\Delta \hat{B}&=\dfrac{1}{2}\left\langle \left[\hat{A},\hat{B}\right]\right\rangle \\
        \hat{L}_z\left|\psi \right\rangle &= m\hbar \left.\psi \right\rangle 
\end{align} 





%}}}

\end{document}
