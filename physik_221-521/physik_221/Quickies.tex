%:LLPStartPreview
%:VimtexCompile(SS)

%{{{ Formatierung

\documentclass[a4paper,12pt]{article}

\usepackage{physics_notetaking}

%%% dark red
%\definecolor{bg}{RGB}{60,47,47}
%\definecolor{fg}{RGB}{255,244,230}
%%% space grey
%\definecolor{bg}{RGB}{46,52,64}
%\definecolor{fg}{RGB}{216,222,233}
%%% purple
%\definecolor{bg}{RGB}{69,0,128}
%\definecolor{fg}{RGB}{237,237,222}
%\pagecolor{bg}
%\color{fg}

\newcommand{\td}{\,\text{d}}
\newcommand{\RN}[1]{\uppercase\expandafter{\romannumeral#1}}
\newcommand{\zz}{\mathrm{Z\kern-.3em\raise-0.5ex\hbox{Z} }}

\newcommand\inlineeqno{\stepcounter{equation}\ {(\theequation)}}
\newcommand\inlineeqnoa{(\theequation.\text{a})}
\newcommand\inlineeqnob{(\theequation.\text{b})}
\newcommand\inlineeqnoc{(\theequation.\text{c})}

\newcommand\inlineeqnowo{\stepcounter{equation}\ {(\theequation)}}
\newcommand\inlineeqnowoa{\theequation.\text{a}}
\newcommand\inlineeqnowob{\theequation.\text{b}}
\newcommand\inlineeqnowoc{\theequation.\text{c}}

\renewcommand{\refname}{Source}
\renewcommand{\sfdefault}{phv}
%\renewcommand*\contentsname{Contents}

\pagestyle{fancy}

\sloppy

\numberwithin{equation}{section}

%}}}

\begin{document}

%{{{ Titelseite

\title{Klausurvorbereitung $|$ physik221}
\author{Stella Hoffmann, Jonas Wortmann}
\maketitle
\pagenumbering{gobble}

%}}}

\newpage

%{{{ Inhaltsverzeichnis

\fancyhead[L]{\thepage}
\fancyfoot[C]{}
\pagenumbering{arabic}

\tableofcontents

%}}}

\newpage

%{{{

\fancyhead[R]{\leftmark\\\rightmark}

\section{Quickies}
\subsection{Übungszettel}
\begin{enumerate}[label=\arabic*.]
        \item Wie ist die Lagrange--Funktion definiert?
        \item Wie wird die Wirkung aus der Lagrange--Funktion definiert?
        \item Wie lauten die Euler--Lagrange--Gleichungen?
        \item Welche Eigenschaften muss die Lagrange--Funktion erfüllen, damit die Gesamtenergie erhalten ist?
        \item Welche Zwangsbedingungen gelten in Kartesischen Koordinaten für die folgenden Beispiele?
                \begin{enumerate}[label=\alph*)]
                        \item Ein Massenpunkt hängt an einem nicht dehnbaren Faden.
                        \item Ein Gummiball springt auf dem Boden.
                \end{enumerate}
        \item Wie viele Erhaltungsgrößen hat das Kepler--Problem?
        \item Welche sind dies?
        \item Wie lautet die Hamilton--Funktion?
        \item Wie lauten die Bewegungsgleichungen im Hamilton--Formalismus?
        \item Wie lauten die Kepler'schen Gesetze?
        \item Wann ist der Drehimpuls erhalten?
        \item Wie ist der Trägheitstensor eines Systems aus $N$ Massenpunkten definiert?
        \item Was sind Hauptträgheitsmomente?
        \item Wie lautet der Satz von Steiner?
        \item Wie viele Koordinaten werden benötigt, um die Bewegung eines starren Körpers zu beschreiben?
        \item Nenne 3 Scheinkräfte.
        \item Wie lautet Newtons Theorem?
        \item Wie lauten die Euler--Gleichungen?
        \item Wie lautet Newtons Theorem?
        \item Was ist der Virialsatz?
\end{enumerate}

\newpage
\subsubsection{Lösungen Übungszettel}
\begin{enumerate}[label=\arabic*.]
        \item $$L=T-V$$
        \item $$S=\int_{t_1}^{t_2}L\left(q, \dot{q}, t\right)\td t$$
        \item $$\diff*{\frac{\partial L}{\partial \dot{q_i} }}{t}-\frac{\partial L}{\partial {q_i}}=0 $$
        \item Wenn $L$ nicht explizit von der Zeit abhängt, also: $$\frac{\partial L}{\partial t} = 0$$
        \item Zwangsbedingung
        \begin{enumerate}[label=\alph*)]
                \item $$l^2 = x^2+y^2+z^2=\text{const.}$$
                \item $$ z \geq 0$$
        \end{enumerate}
        \item 5
        \item Kepler-Erhaltungsgrößen
        \begin{enumerate}[label=\alph*)]
                \item Runge-Lenz-Vektor
                \item Gesamtdrehimpuls
                \item Gesamtimpuls
                \item Energie
                \item Schwerpunkt
        \end{enumerate}
        \item $$H=\sum_{i}^{}\dot{q}_i p_i - L$$
        \item $$\dot{p}_i = - \diffp{H}{q_i} \qquad \land \qquad \dot{q}_i = \diffp{H}{{p}_i}  $$
        \item Kepler'schen Gesetze
        \begin{enumerate}[label=\alph*)]
                \item Die Planeten kreisen auf elliptischen Bahnen um ihr Zentralgestirn.
                \item $$\frac{a^3}{T^2} = \text{const.}$$
                \item Die Ortsvektoren der Planeten überstreichen in gleichen Zeiten, die gleichen Flächen.
        \end{enumerate}
        \item Der Drehimpuls ist erhalten, wenn das System invariant unter einer Rotation aller Koordinaten $\vv{x}_i$ ist.
        \item $$I_{ij}=\sum_{n}^{N} m_n \left( x_{ij}^2\cdot\delta_{ij}-x_{ni} \cdot x_{nj}\right)$$
        \item Die Hauptträgheitsmomente sind die Diagonalelemente/Eigenwerte des diagonalisierten Trägheitstensors.
        \item $$I_{\text{ges}}=I_{\text{SP}}+ma^2$$
        \item 6: Dazu gehören 3 Ortskoordinaten und 3 Eulerinkel.
        \item Scheinkräfte
        \begin{enumerate}[label=\alph*)]
                \item Azimutalkraft $$\vv{F} = -m\dot{\vv{\omega}} \times \vv{r}$$
                \item Fliehkraft $$\vv{F}= -m{\vv{\omega}} \times \left(\vv{\omega} \times \vv{r}\right) $$
                \item Corioliskraft $$\vv{F}= - 2m \left(\vv{\omega} \times \vv{v}\right)$$
                \item Translationskraft $$\vv{F}= -m \diff*[2]{\vv{R}}{t}$$
        \end{enumerate}
        \item Sphärisch symmetrische Körper erzeugen die gleiche Kraft wie ein Massenpunkt im Mittelpunkt. Innerhalb dieser sphärisch symmetrischen Körper mit homogener Massenverteilung (z.B.\ Hohlkugel) wirkt keine Kraft.
        \item Euler-Gleichungen
        \begin{align*}
                \vv{M} &= I \dot{\vv{\omega}} + \vv{\omega} \times (I \vv{\omega})\\
                M_1 &= I_{11} \dot{\omega_1} + (I_{33}-I_{22})\cdot \omega_3 \omega_2\\
                M_2 &= I_{22} \dot{\omega_2} + (I_{11}-I_{33})\cdot \omega_1 \omega_3\\
                M_3 &= I_{33} \dot{\omega_3} + (I_{22}-I_{11})\cdot \omega_2 \omega_1
        \end{align*}
        \item Außerhalb einer kugelsymmetrischen Massenverteilung wirkt die gleiche Gravitationskraft wie eine Punktmasse in dessen Mitte. Innerhalb einer z.B.\ Hohlkugel wirkt keine Gravitationskraft.
        \item $\left\langle E_{\,\text{kin}\,}\right\rangle =\tfrac{k}{2}\left\langle E_{\,\text{pot}\,}\right\rangle $, mit $\left\langle .\right\rangle $ als Mittelwert. $E_{\,\text{pot}\,}$ ist ein Potential $k$--ter Ordnung.
\end{enumerate}

\newpage
\subsection{Zettel\_14}
\begin{enumerate}[label=\arabic*.]
        \item Wann ist eine Kraft $\vv{F}$ konservativ?
        \item Wie lautet die Lagrangefunktion?
        \item Wie lauten die Bewegungsgleichungen zur Lagrangefunktion?
        \item In einer Dimension sei das Potential $V=kx^2$. Wie lautet die Lösung der Bewegungsgleichung?
        \item Wie lautet die kinetische Energie in Kugelkoordinaten?
        \item Wie lautet die Galilei-Transformation?
        \item Wie ist der kanonisch konjugierte Impuls definiert?
        \item Wie lautet die Beziehung zwischen Lagrange- und Hamilton-funktion?
        \item Wie lauten die Bewegungsgleichungen im Hamiltonformalismus?
        \item Leite die Bewegungsgleichung des harmonischen Oszillators im Hamiltonformalismus her.
        \item Wie ist ein starrer Körper definiert?
        \item Wie ist der Trägheitstensor $I_{ij}$ definiert?
        \item Bestimme die Komponente des Trägheitstensors eines Stabes für die Rotationsachse orthogonal zum Stab durch den Schwerpunkt.
        \item Wie lautet das Trägheitsmoment, wenn die Rotationsachse nicht durch den Schwerpunkt sondern durch das Ende des Stabes geht? Welchen Satz kann man hier benutzen?
        \item Wie lauten die Bewegungsgleichungen eines starren Körpers?
        \item Was sind Hauptträgheitsmomente?
        \item Wie lautet Newtons Theorem?
        \item Wie lauten die Keplerschen Gesetze?
        \item Wann ist der Drehimpuls erhalten?
        \item Was sind zyklische Koordinaten?
\end{enumerate}

\newpage
\subsubsection{Lösungen Zettel\_14}
\begin{enumerate}[label=\arabic*.]
        \item Wenn eine der drei folgenden Bedingungen erfüllt ist
                \begin{enumerate}[label=\alph*)]
                        \item Wenn das geschlossene Integral über den Weg gleich null ist, oder die Arbeit wegunabhängig ist,
                                \begin{align*} 
                                        \oint_{}^{}\vv{F}\td \vv{x}&=0\\
                                        \int_{1}^{2}\vv{F}\td \vv{x}&=W_{12}
                                .\end{align*} 
                        \item Es existiert ein Potential, sodass 
                                \begin{align*} 
                                        \vv{F}&=-\vv{\nabla }V
                                .\end{align*} 
                        \item Wenn die Rotation der Kraft gleich null ist, also
                                \begin{align*} 
                                        \vv{\nabla }\times \vv{F}=0
                                .\end{align*} 
                \end{enumerate}
        \item $L=T-V$
        \item $\diff*[]{\diffp[]{\mathcal{L}}{\dot{q}_i}}{t}-\diffp[]{\mathcal{L}}{q_i}=0$
        \item Für ein Teilchen gilt
                \begin{align*} 
                        \mathcal{L}&=\dfrac{1}{2}m\dot{x}^2-kx^2
                .\end{align*} 
                Mit \begin{align*} 
                        \diff*[]{\diffp[]{\mathcal{L}}{\dot{q}_i}}{t}-\diffp[]{\mathcal{L}}{q_i}&=0
                ,\end{align*} 
                folgt
                \begin{align*} 
                        m\ddot{x}+2kx&=0\\
                        \ddot{x}&=-\dfrac{2k}{m}x
                .\end{align*} 
                Der Ansatz ist $A\cos \left(\omega t\right)+B\sin \left(\omega t\right)$, da es sich um einen harmonischen Oszillator handelt. Das ergibt
                \begin{align*} 
                        \omega ^2&=\dfrac{2k}{m}
                \end{align*} 
        \item Kugelkoordinaten sind
                \begin{align*} 
                        \vv{r}&=\begin{pmatrix}
                                r\cos \varphi \sin \theta \\
                                r\sin \varphi \sin \theta  \\
                                r\cos \theta 
                        \end{pmatrix}
                .\end{align*} 
                Die kinetische Energie ist
                \begin{align*} 
                        T&=\dfrac{1}{2}m\dot{r}^2\\
                         &=\dfrac{1}{2}m\left(\diff[]{r}{t}\right)^2\\
                         &=\dfrac{1}{2}m\left(\dot{r}^2+r^2\dot{\theta }^2+r^2\dot{\varphi }^2\sin ^2\theta \right)
                \end{align*} 
        \item \begin{align*} 
                        \vv{x}\mapsto \vv{x}'&=\mathcal{R}\vv{x}+\vv{v}t+\vv{a}\qquad t\mapsto t'=t+t_0\\
                                             &\mathcal{R} \in \mathbb{R}^{3\times 3}\,\text{mit}\,\mathcal{R}^T=\mathcal{R}^{-1}
        ,\end{align*} 
        mit $\mathcal{R}$ unabhängig von $t$, $\vv{v}$ einer konstanten Geschwindigkeit und $\vv{a}$ einer konstanten Raumverschiebung.
        \item $p_i=\diffp[]{\mathcal{L}}{\dot{q}_i}$ 
        \item $\mathcal{H}=\sum_{i}^{}p_i\dot{q}_i-\mathcal{L}$ 
        \item \begin{align*} 
                        \dot{p}_i&=-\diffp[]{\mathcal{H}}{q_i}\\
                        \dot{q}_i&=\diffp[]{\mathcal{H}}{p_i}
        \end{align*} 
        \item Man nimmt eine Dimension an, mit
                \begin{align*} 
                        T=\dfrac{1}{2}m\dot{x}^2\qquad V=\dfrac{1}{2}kx^2
                .\end{align*} 
                Also ist die Lagrange--Funktion
                \begin{align*} 
                        \mathcal{L}&=\dfrac{1}{2}m\dot{x}^2-\dfrac{1}{2}kx^2
                .\end{align*} 
                Der kanonische Impuls ist
                \begin{align*} 
                        p&=\diffp[]{\mathcal{L}}{\dot{x}}\\
                         &=m\dot{x}
                .\end{align*} 
                Die Hamilton--Funktion ist dann
                \begin{align*} 
                        \mathcal{H}&=p\dot{x}-\mathcal{L}\\
                                   &=p\dot{x}-\dfrac{1}{2}m\dot{x}^2+\dfrac{1}{2}kx^2
                .\end{align*} 
                $\mathcal{H}=\mathcal{H}\left(x,p,t\right)$ ist nur eine Funktion von $x,p$ und $t$, also muss $\dot{x}$ mit $\dot{x}=\tfrac{p}{m}$ ersetzt werden
                \begin{align*} 
                        \mathcal{H}&=\dfrac{p^2}{m}-\dfrac{1}{2}\dfrac{p^2}{m}+\dfrac{1}{2}kx^2\\
                                   &=\dfrac{p^2}{2m}+\dfrac{1}{2}kx^2
                .\end{align*} 
                Die Bewegungsgleichungen sind dann
                \begin{align*} 
                        \dot{p}&=-\diffp[]{\mathcal{H}}{x}&\dot{x}&=\diffp[]{\mathcal{H}}{p}\\
                               &=-kx&&=\dfrac{p}{m}\\
                               &&\ddot{x}&=\dfrac{\dot{p}}{m}
                .\end{align*} 
                Daraus folgt $m\ddot{x}=-kx$.
        \item Der starre Körper ist der idealisierte Festkörper, dessen Massenelemente einen festen Abstand zueinander haben. Verformungen werden vernachlässigt. (Für diskrete Massenverteilung gilt $\,\forall \vv{x}_i,\vv{x}_j:|\vv{x}_i-\vv{x}_j|=\,\text{const.}\,$.)
        \item Für diskrete Massenverteilungen
                \begin{align*} 
                        I_{ij}&=\sum_{n=1}^{N}m_n\left(|\vv{x}_n|^2\delta _{ij}-x_{ni}x_{nj}\right)
                .\end{align*} 
                Für kontinuierliche Massenverteilungen
                \begin{align*} 
                        I_{ij}&=\int_{}^{}\td ^3x\rho \left(|\vv{x}|\right)\left(\vv{x}^2\delta _{ij}-x_ix_j\right)
                .\end{align*} 
        \item Der Stab wird entlang der $x$--Achse gelegt. Das Trägheitsmoment ist dann
                \begin{align*} 
                        I_{yy}&=\int_{}^{}\td V\rho \left(\vv{x}\right)\left(|\vv{x}|^2\delta _{yy}-yy\right)\\
                              &\equiv \rho \int_{}^{}\td x\left(x^2+y^2+z^2-y^2\right)\\
                              &=\rho \int_{}^{}\td x\left(x^2+z^2\right)&&z=0\\
                              &=\rho \int_{-l}^{l}x^2\td x\\
                              &=\left.\rho \dfrac{1}{3}x^3\right|_{-l}^l\\
                              &=\dfrac{m}{2l}\dfrac{2}{3}l^3&&\rho =\dfrac{m}{2l}\\
                              &=\dfrac{1}{3}ml^2
                \end{align*} 
        \item (Mit Satz von Steiner)
        \item (Wahrscheinlich) Euler--Gleichungen
                \begin{align*} 
                        \vv{M}&=I\dot{\vv{\omega }}+\vv{\omega }\times \left(I\vv{\omega }\right)
                \end{align*} 
        \item Die HTM sind die Diagonalelemente der Trägheitsmatrix, im Falle dass diese diagonalisiert wurde. Um diese zu diagonalisieren muss über Rotationsmatrizen $\mathcal{R} \in \mathbb{R}^{3\times 3}$ in das HTS transformiert werden. Die HTM beziehen sich auf Drehungen bzgl.\ der HTA.
        \item (Newtons Theorem)
        \item (Keplers Gesetze)
        \item Der Drehimpuls ist erhalten, wenn das System invariant unter einer Rotation aller Koordinaten $\vv{x}_i$ ist.
        \item Eine zyklische Koordinate ist eine Koordinate $q_i$ von der die Lagrange--Funktion nicht abhängt. Sei $q_i$ zyklisch, dann gilt
                \begin{align*} 
                        \mathcal{L}&=\mathcal{L}\left(\hdots ,q_{i-1},q_{i+1},\hdots ;\hdots \dot{q}_{i-1},\dot{q}_i,\dot{q}_{i+1},\hdots \right)\\
                        \diffp[]{\mathcal{L}}{q_i}&=0
                .\end{align*} 
\end{enumerate}
        
\newpage
\subsection{2022 Klausur I}
\begin{enumerate}[label=\arabic*.]
        \item Wie viele und welche Erhaltungsgrößen gibt es im Zentralkraftproblem im Allgemeinen? Woraus folgen diese Erhaltungsgrößen?
        \item Welche weitere Erhaltungsgröße findet man beim Keplerproblem und welche Funktion mit zentraler Bedeutung für das Keplerproblem kann hieraus leicht berechnet werden?
        \item Wie ist die Lagrange--Funktion definiert und wie lauten die Euler--Lagrange--Gleichungen?
        \item Was ist das Hamilton'sche Prinzip? Wie ist die Wirkung definiert?
        \item Betrachten Sie eine Lagrange--Funktion mit generalisierten Koordinaten $q:\mathcal{L}=\mathcal{L}\left(q,\dot{q},t\right)$. Was sind zyklische Koordinaten und wie stehen diese in Zusammenhang mit den Euler--Lagrange--Gleichungen?
        \item Was ist eine Legendre--Transformation? Berechnen Sie aus $f\left(x\right)=ax^2$ die zugehörige Legendre--Transformierte $g\left(y\right)$ mit $y=\diffp[]{f}{x}$.
        \item Wie lautet das Neother--Theorem, wenn die Lagrange--Funktion unter einer kontinuierlichen, stetig differenzierbaren Koordinatentransformation bis auf eine Eichtransformation invariant ist?
        \item Wie sind Poisson--Klammern definiert? Wie lauten die Hamilton'schen Bewegungsgleichungen, ausgedrückt durch Poisson--Klammern?
        \item Zeige, für welche $a,b$ die Transformation $Q=q^a\cos \left(bp\right),P=q^a\sin \left(bp\right)$ kanonisch ist.
\end{enumerate}

\newpage
\subsubsection{Lösungen 2022 Klausur I}
\begin{enumerate}[label=\arabic*.]
        \item 5:
                \begin{enumerate}[label=\alph*)]
                        \item Energie, da es sich bei Gravitation um ein konservatives Kraftfeld handelt.
                        \item Gesamtdrehmipuls, da kein Drehmoment ausgeübt wird.
                        \item Gesamtimpuls, da keine äußere Kraft ausgeübt wird.
                        \item Schwerpunkt, da
                        \item Runge--Lenz--Vektor, da
                \end{enumerate}
        \item (wahrscheinlich Runge--Lenz--Vektor)
        \item Lagrange--Funktion: $L=T-V$. \\
                Euler--Lagrange: $\diff*[]{\diffp[]{L}{\dot{q}} }{t}=\diffp[]{L}{q}$.
        \item Das Hamilton--Prinzip ist ein Extremalprinzip. Es besagt, dass Felder und Teilchen für eine bestimmte Größe einen extremalen Wert annehmen. Dies wird auch Wirkung genannt. Sie ist
                \begin{align*} 
                        S=\int_{t_1}^{t_2}\mathcal{L}\td t
                .\end{align*} 
        \item Man bezeichnet die Koordinate $q_j$, für die $\diffp[]{\mathcal{L}}{q_j}=0$ gilt, als zyklische Koordinate. Man nennt das System translationsinvariant in der zyklischen Koordinate; diese ist widerum äquivalent zur Erhaltung des Impulses, welcher der zyklischen Koordinate zugeordnet ist.
        \item Die Legendre--Transformation beschreibt den Übergang der Variablen $x$ in einer Funktion $f\left(x\right)$ zu den neuen Variablen $u:=\diffp[]{f}{x}$ in einer neuen Funktion $g\left(u\right)=ux\left(u\right)-f$. Also ist
                \begin{align*} 
                        f\left(x\right)&=ax^2\\
                        g\left(u\right)&=ux\left(u\right)-f&&\left|u=\diffp[]{f}{x}=2ax\Leftrightarrow x\left(u\right)=\dfrac{u}{2a}\right.\\
                                       &=u\cdot \dfrac{u}{2a}-a\cdot \dfrac{u^2}{4a^2}\\
                                       &=\dfrac{u^2}{4a}
                \end{align*} 
        \item \begin{align*} 
                I\left(q,\dot{q},t\right):=\left.\sum_{i=1}^{3N-k}\diffp[]{\mathcal{L}}{\dot{q}}\cdot \diffp[]{q_i\left(q',t,\alpha \right)}{\alpha }\right|_{\alpha =0}
        \end{align*} 
        \item (nicht behandelt, stand: 2023 Dreeeeeeeeeeeeeeeeeeeeees)
        \item s.o.\
\end{enumerate}

\newpage
\subsection{2022 Klausur II}
\begin{enumerate}[label=\arabic*.]
        \item Erläutern Sie kurz die drei Newton'schen Axiome.
        \item Was sagt das Noether--Theorem im Lagrange--Formalismus aus?
        \item Wie lautet die totale zeitliche Ableitung $\diff[]{f}{t}$ einer Funktion $f\left(q,p,t\right)$? Drücken Sie $\diff[]{f}{t}$ durch die Hamilton--Funktion aus und zeigen Sie, dass sich das Ergebnis kompakt mit Hilfe der Poisson--Klammern schreiben lässt. Wann ist $f$ eine Erhaltungsgröße wenn sie nicht explizit zeitabhängig ist?
        \item Berechne aus $f\left(x\right)=ax^2$ die zugehörige Legendre--Transformation $g\left(y\right)$ mit $y=\diffp[]{f}{x}$.
        \item Wie ist der Trägheitstensor eines starren Körpers bestehend aus $N$ Punktmassen definiert?
        \item Untersuche ob folgende Kraftfelder konservativ sind ($c_i$ sind Konstanten, $x,y,z$ sind Ortskoordinaten, $t$ ist die Zeit)
                \begin{enumerate}[label=\roman*.]
                        \item $\vv{F}_1=c_1\left(x^2z,xy,xz\right)^T$ 
                        \item $\vv{F}_2=c_2\left(y^3z,3xy^2z,xy^3\right)^T$ 
                        \item $\vv{F}_3=c_3t^2\left(x^2\,\sqrt[]{z}\tan \left(xyz\right),\,\sqrt[]{y}z^5,x^{3/2}\cos y\right)^T$ 
                \end{enumerate}
        \item Wann ist die Anwendung der Störungstheorie sinnvoll?
        \item Betrachten Sie die Bewegungsgleichung $\ddot{x}\left(t\right)+\omega ^2_0x\left(t\right)=\varepsilon \zeta \,\text{i}\,x^2\left(t\right)+O\left(\varepsilon ^2\right)$ des anharmonischen Oszillators, die für alle Werte $\varepsilon $ eine Lösung besitzen soll. Skizzieren Sie (ohne explizite Rechnung), wie man vorgehen muss, um für diese inhomogene Differenzialgleichung mit beliebig fixierten Anfangsbedingungen perturbativ eine explizite Lösung für $x\left(t\right)$ bis zur ersten Ordnung in $\varepsilon $ zu finden.
        \item Zeigen Sie, für welche $a,b$ die Transformation $Q=q ^{a/2}\sin \left(bp\right),P= q ^{a/2}\cos \left(bp\right)$ kanonisch ist.
\end{enumerate}

\newpage
\subsubsection{Lösungen 2022 Klausur II}
\begin{enumerate}[label=\arabic*.]
        \item \begin{enumerate}[label=\alph*)]
                \item Ein kräftefreier Körper bleibt in Ruhe oder bewegt sich geradelinig mit konstanter Geschwindigkeit.
                \item $\vv{F}=m\cdot \vv{a}$.
                \item $\vv{F}_{A\rightarrow B}=-\vv{F}_{B\rightarrow A}$.
        \end{enumerate}
        \item Wenn die Lagrange--Funktion unter der kontinuierlichen, stetig differenzierbaren Koordinatentransformation, dann ist die Funktion eine Erhaltungsgröße. Das bedeutet insbesondere, dass zu jeder Transformation die die Lagrange--Funktion nicht ändert, eine Erhaltungsgröße gehört. Sie wird mit
                \begin{align*} 
                I\left(q,\dot{q},t\right):=\left.\sum_{i=1}^{3N-k}\diffp[]{\mathcal{L}}{\dot{q}}\cdot \diffp[]{q_i\left(q',t,\alpha \right)}{\alpha }\right|_{\alpha =0}
                \end{align*} 
                berechnet.
        \item \begin{align*} 
        \diff[]{f}{t}=\diffp[]{f}{q}\diffp[]{q}{t}+\diffp[]{f}{p}\diffp[]{p}{t}+\diffp[]{f}{t}
        \end{align*} 
        \begin{align*} 
                H=\sum_{i}^{}p_i\dot{q}_i-f\qquad ?
        \end{align*} 
        \item Die Legendre--Transformation beschreibt den Übergang der Variablen $x$ in einer Funktion $f\left(x\right)$ zu den neuen Variablen $u:=\diffp[]{f}{x}$ in einer neuen Funktion $g\left(u\right)=ux\left(u\right)-f$. Also ist
                \begin{align*} 
                        f\left(x\right)&=ax^2\\
                        g\left(u\right)&=ux\left(u\right)-f&&\left|u=\diffp[]{f}{x}=2ax\Leftrightarrow x\left(u\right)=\dfrac{u}{2a}\right.\\
                                       &=u\cdot \dfrac{u}{2a}-a\cdot \dfrac{u^2}{4a^2}\\
                                       &=\dfrac{u^2}{4a}
                \end{align*} 
        \item \begin{align*} 
                I_{ij}:=\sum_{n}^{N}m_n\left(|\vv{x}|^2\delta _{ij}-x_{ni}\cdot x_{nj}\right)
        \end{align*} 
        \item konservativ, wenn $\vv{\nabla }\times \vv{F}=0$ oder $\oint_{C}^{}\vv{F}\td s=0$.
        \item (gibts nicht)
        \item s.o.\
        \item s.o.\
\end{enumerate}

\newpage
\subsection{2021 Klausur I}
\begin{enumerate}[label=\arabic*.]
        \item Geben Sie die Euler--Lagrange Gleichungen zu einer Lagrangefunktion $\mathcal{L}\left(q,\dot{q},t\right)$ an.
        \item Zeigen Sie die Energieerhaltung bei der eindimensionalen Bewegung in einem Potential $U\left(x\right)$ ausgehend von der Newton'schen Bewegungsgleichung $m\ddot{x}=-\diff[]{U}{x}$.
        \item Es sei die Zwangsbedingung $\tfrac{x^2}{a}+\tfrac{y^2}{b}=1$ für eine Bewegung gegeben. Was bedeutet diese Bedingung geometrisch? Berechnen Sie die zugehörige Zwangskraft $Z$.
        \item Zeigen Sie, dass für eine Lagrange--Funktion $\mathcal{L}\left(q,\dot{q}\right)$, die nicht explizit von der Zeit abhängt, auf der physikalischen Bahnkurve $q\left(t\right),\dot{q}\left(t\right)$ (welche die Bewegungsgleichung erfüllt) gilt $\diff*[]{\left(\mathcal{L}-\diffp[]{\mathcal{L}}{\dot{q}}\dot{q}\right)}{t}=0$.
        \item Was bedeutet es, wenn eine Variable $q$ zyklisch ist? Was gilt dann für den kanonischen Impuls $p$ bezüglich der Variablen $q$?
        \item Welche Erhaltungsgrößen hat das Keplerproblem?
        \item Wie lauten die Hamilton'schen Bewegungsgleichungen allgemein?
        \item Wie sind die Poissonklammern definiert?
        \item Wie ist das Skalarprodukt im Minkowski--Raum definiert? Geben Sie den metrischen Tensor im Minkowski--Raum in Matrixform an.
        \item Wie lautet die relativistische Energie--Impuls--Beziehung?
        \item Wie hängt die Rotationsenergie mit dem Trägheitstensor zusammen?
        \item Was besagt der Satz von Steiner? (Formel angeben).
\end{enumerate}

\newpage
\subsection{2019 Klausur I}
\begin{enumerate}[label=\arabic*.]
        \item Wie lauten die Newton'schen Axiome?
        \item Zeigen Sie, dass für den Drehimpuls $\vv{L}$ im Zentralpotential $\diff[]{\vv{L}}{t}=0$ gilt.
        \item Was ist das Hamilton'sche Prinzip? Wie ist die Wirkung definiert?
        \item Was besagt das Neother--Theorem im Lagrange--Formalismus?
        \item Wann ist eine Verschiebung der Frequenz bei der Störungstheorie eines anharmonischen Oszillators notwendig? Was passiert wenn man diese nicht berücksichtigt?
        \item Wie lautet der Satz von Steiner?
        \item Wie lauten die Hamilton'schen Bewegungsgleichungen?
        \item Wie lautet die totale zeitliche Ableitung einer Funktion $f\left(q,p,t\right)$ ausgedrückt durch Poisson--Klammern? Wann ist $f$ eine Erhaltungsgröße, wenn sie nicht explizit zeitabhängig ist?
\end{enumerate}

\newpage

\begin{align} 
        \ddot{\theta }&=-\dfrac{g\theta }{l\left(1-\dfrac{m}{M+m}\right)}
\end{align} 


%}}}

\end{document}
