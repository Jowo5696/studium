%:LLPStartPreview
%:VimtexCompile(SS)

%{{{ Formatierung

\documentclass[a4paper,12pt]{article}

\usepackage{physics_notetaking}

%%% dark red
%\definecolor{bg}{RGB}{60,47,47}
%\definecolor{fg}{RGB}{255,244,230}
%%% space grey
%\definecolor{bg}{RGB}{46,52,64}
%\definecolor{fg}{RGB}{216,222,233}
%%% purple
%\definecolor{bg}{RGB}{69,0,128}
%\definecolor{fg}{RGB}{237,237,222}
%\pagecolor{bg}
%\color{fg}

\newcommand{\td}{\,\text{d}}
\newcommand{\RN}[1]{\uppercase\expandafter{\romannumeral#1}}
\newcommand{\zz}{\mathrm{Z\kern-.3em\raise-0.5ex\hbox{Z} }}

\newcommand\inlineeqno{\stepcounter{equation}\ {(\theequation)}}
\newcommand\inlineeqnoa{(\theequation.\text{a})}
\newcommand\inlineeqnob{(\theequation.\text{b})}
\newcommand\inlineeqnoc{(\theequation.\text{c})}

\newcommand\inlineeqnowo{\stepcounter{equation}\ {(\theequation)}}
\newcommand\inlineeqnowoa{\theequation.\text{a}}
\newcommand\inlineeqnowob{\theequation.\text{b}}
\newcommand\inlineeqnowoc{\theequation.\text{c}}

\renewcommand{\refname}{Source}
\renewcommand{\sfdefault}{phv}
%\renewcommand*\contentsname{Contents}

\pagestyle{fancy}

\sloppy

\numberwithin{equation}{section}

%}}}

\begin{document}

%{{{ Titelseite

\title{Klausurvorbereitung $|$ physik221}
\author{Stella Hoffmann, Jonas Wortmann}
\maketitle
\pagenumbering{gobble}

%}}}

\newpage

%{{{ Inhaltsverzeichnis

\fancyhead[L]{\thepage}
\fancyfoot[C]{}
\pagenumbering{arabic}

\tableofcontents

%}}}

\newpage

%{{{

\fancyhead[R]{\leftmark\\\rightmark}

\section{Quickies}
\subsection{Übungszettel}
\begin{enumerate}[label=\arabic*.]
        \item Wie ist die Lagrange--Funktion definiert?
        \item Wie wird die Wirkung aus der Lagrange--Funktion definiert?
        \item Wie lauten die Euler--Lagrange--Gleichungen?
        \item Welche Eigenschaften muss die Lagrange--Funktion erfüllen, damit die Gesamtenergie erhalten ist?
        \item Welche Zwangsbedingungen gelten in Kartesischen Koordinaten für die folgenden Beispiele?
                \begin{enumerate}[label=\alph*)]
                        \item Ein Massenpunkt hängt an einem nicht dehnbaren Faden.
                        \item Ein Gummiball springt auf dem Boden.
                \end{enumerate}
        \item Wie viele Erhaltungsgrößen hat das Kepler--Problem?
        \item Welche sind dies?
        \item Wie lautet die Hamilton--Funktion?
        \item Wie lauten die Bewegungsgleichungen im Hamilton--Formalismus?
        \item Wie lauten die Kepler'schen Gesetze?
        \item Wann ist der Drehimpuls erhalten?
        \item Wie ist der Trägheitstensor eines Systems aus $N$ Massenpunkten definiert?
        \item Was sind Hauptträgheitsmomente?
        \item Wie lautet der Satz von Steiner?
        \item Wie viele Koordinaten werden benötigt, um die Bewegung eines starren Körpers zu beschreiben?
        \item Nenne 3 Scheinkräfte.
        \item Wie lautet Newtons Theorem?
        \item Wie lauten die Euler--Gleichungen?
\end{enumerate}

\newpage
\subsection{2022 Klausur I}
\begin{enumerate}[label=\arabic*.]
        \item Wie viele und welche Erhaltungsgrößen gibt es im Zentralkraftproblem im Allgemeinen? Woraus folgen diese Erhaltungsgrößen?
        \item Welche weitere Erhaltungsgröße findet man beim Keplerproblem und welche Funktion mit zentraler Bedeutung für das Keplerproblem kann hieraus leicht berechnet werden?
        \item Wie ist die Lagrange--Funktion definiert und wie lauten die Euler--Lagrange--Gleichungen?
        \item Was ist das Hamilton'sche Prinzip? Wie ist die Wirkung definiert?
        \item Betrachten Sie eine Lagrange--Funktion mit generalisierten Koordinaten $q:\mathcal{L}=\mathcal{L}\left(q,\dot{q},t\right)$. Was sind zyklische Koordinaten und wie stehen diese in Zusammenhang mit den Euler--Lagrange--Gleichungen?
        \item Was ist eine Legendre--Transformation? Berechnen Sie aus $f\left(x\right)=ax^2$ die zugehörige Legendre--Transformierte $g\left(y\right)$ mit $y=\diffp[]{f}{x}$.
        \item Wie lautet das Neother--Theorem, wenn die Lagrange--Funktion unter einer kontinuierlichen, stetig differenzierbaren Koordinatentransformation bis auf eine Eichtransformation invariant ist?
        \item Wie sind Poisson--Klammern definiert? Wie lauten die Hamilton'schen Bewegungsgleichungen, ausgedrückt durch Poisson--Klammern?
        \item Zeige, für welche $a,b$ die Transformation $Q=q^a\cos \left(bp\right),P=q^a\sin \left(bp\right)$ kanonisch ist.
\end{enumerate}

\newpage
\subsection{2022 Klausur II}
\begin{enumerate}[label=\arabic*.]
        \item Erläutern Sie kurz die drei Newton'schen Axiome.
        \item Was sagt das Noether--Theorem im Lagrange--Formalismus aus?
        \item Wie lautet die totale zeitliche Ableitung $\diff[]{f}{t}$ einer Funktion $f\left(q,p,t\right)$? Drücken Sie $\diff[]{f}{t}$ durch die Hamilton--Funktion aus und zeigen Sie, dass sich das Ergebnis kompakt mit Hilfe der Poisson--Klammern schreiben lässt. Wann ist $f$ eine Erhaltungsgröße wenn sie nicht explizit zeitabhängig ist?
        \item Berechne aus $f\left(x\right)=ax^2$ die zugehörige Legendre--Transformation $g\left(y\right)$ mit $y=\diffp[]{f}{x}$.
        \item Wie ist der Trägheitstensor eines starren Körpers bestehend aus $N$ Punktmassen definiert?
        \item Untersuche ob folgende Kraftfelder konservativ sind ($c_i$ sind Konstanten, $x,y,z$ sind Ortskoordinaten, $t$ ist die Zeit)
                \begin{enumerate}[label=\roman*.]
                        \item $\vv{F}_1=c_1\left(x^2z,xy,xz\right)^T$ 
                        \item $\vv{F}_2=c_2\left(y^3z,3xy^2z,xy^3\right)^T$ 
                        \item $\vv{F}_3=c_3t^2\left(x^2\,\sqrt[]{z}\tan \left(xyz\right),\,\sqrt[]{y}z^5,x^{3/2}\cos y\right)^T$ 
                \end{enumerate}
        \item Wann ist die Anwendung der Störungstheorie sinnvoll?
        \item Betrachten Sie die Bewegungsgleichung $\ddot{x}\left(t\right)+\omega ^2_0x\left(t\right)=\varepsilon \zeta \,\text{i}\,x^2\left(t\right)+O\left(\varepsilon ^2\right)$ des anharmonischen Oszillators, die für alle Werte $\varepsilon $ eine Lösung besitzen soll. Skizzieren Sie (ohne explizite Rechnung), wie man vorgehen muss, um für diese inhomogene Differenzialgleichung mit beliebig fixierten Anfangsbedingungen perturbativ eine explizite Lösung für $x\left(t\right)$ bis zur ersten Ordnung in $\varepsilon $ zu finden.
        \item Zeigen Sie, für welche $a,b$ die Transformation $Q=q ^{a/2}\sin \left(bp\right),P= q ^{a/2}\cos \left(bp\right)$ kanonisch ist.
\end{enumerate}

\newpage
\subsection{2021 Klausur I}
\begin{enumerate}[label=\arabic*.]
        \item Geben Sie die Euler--Lagrange Gleichungen zu einer Lagrangefunktion $\mathcal{L}\left(q,\dot{q},t\right)$ an.
        \item Zeigen Sie die Energieerhaltung bei der eindimensionalen Bewegung in einem Potential $U\left(x\right)$ ausgehend von der Newton'schen Bewegungsgleichung $m\ddot{x}=-\diff[]{U}{x}$.
        \item Es sei die Zwangsbedingung $\tfrac{x^2}{a}+\tfrac{y^2}{b}=1$ für eine Bewegung gegeben. Was bedeutet diese Bedingung geometrisch? Berechnen Sie die zugehörige Zwangskraft $Z$.
        \item Zeigen Sie, dass für eine Lagrange--Funktion $\mathcal{L}\left(q,\dot{q}\right)$, die nicht explizit von der Zeit abhängt, auf der physikalischen Bahnkurve $q\left(t\right),\dot{q}\left(t\right)$ (welche die Bewegungsgleichung erfüllt) gilt $\diff*[]{\left(\mathcal{L}-\diffp[]{\mathcal{L}}{\dot{q}}\dot{q}\right)}{t}=0$.
        \item Was bedeutet es, wenn eine Variable $q$ zyklisch ist? Was gilt dann für den kanonischen Impuls $p$ bezüglich der Variablen $q$?
        \item Welche Erhaltungsgrößen hat das Keplerproblem?
        \item Wie lauten die Hamilton'schen Bewegungsgleichungen allgemein?
        \item Wie sind die Poissonklammern definiert?
        \item Wie ist das Skalarprodukt im Minkowski--Raum definiert? Geben Sie den metrischen Tensor im Minkowski--Raum in Matrixform an.
        \item Wie lautet die relativistische Energie--Impuls--Beziehung?
        \item Wie hängt die Rotationsenergie mit dem Trägheitstensor zusammen?
        \item Was besagt der Satz von Steiner? (Formel angeben).
\end{enumerate}

\newpage
\subsection{2019 Klausur I}
\begin{enumerate}[label=\arabic*.]
        \item Wie lauten die Newton'schen Axiome?
        \item Zeigen Sie, dass für den Drehimpuls $\vv{L}$ im Zentralpotential $\diff[]{\vv{L}}{t}=0$ gilt.
        \item Was ist das Hamilton'sche Prinzip? Wie ist die Wirkung definiert?
        \item Was besagt das Neother--Theorem im Lagrange--Formalismus?
        \item Wann ist eine Verschiebung der Frequenz bei der Störungstheorie eines anharmonischen Oszillators notwendig? Was passiert wenn man diese nicht berücksichtigt?
        \item Wie lautet der Satz von Steiner?
        \item Wie lauten die Hamilton'schen Bewegungsgleichungen?
        \item Wie lautet die totale zeitliche Ableitung einer Funktion $f\left(q,p,t\right)$ ausgedrückt durch Poisson--Klammern? Wann ist $f$ eine Erhaltungsgröße, wenn sie nicht explizit zeitabhängig ist?
\end{enumerate}

%}}}

\end{document}
