%:LLPStartPreview
%:VimtexCompile(SS)

%{{{ Formatierung

\documentclass[a4paper,12pt]{article}

\usepackage{physics_notetaking}

%%% dark red
%\definecolor{bg}{RGB}{60,47,47}
%\definecolor{fg}{RGB}{255,244,230}
%%% space grey
%\definecolor{bg}{RGB}{46,52,64}
%\definecolor{fg}{RGB}{216,222,233}
%%% purple
%\definecolor{bg}{RGB}{69,0,128}
%\definecolor{fg}{RGB}{237,237,222}
%\pagecolor{bg}
%\color{fg}

\newcommand{\td}{\,\text{d}}
\newcommand{\RN}[1]{\uppercase\expandafter{\romannumeral#1}}
\newcommand{\zz}{\mathrm{Z\kern-.3em\raise-0.5ex\hbox{Z} }}

\newcommand\inlineeqno{\stepcounter{equation}\ {(\theequation)}}
\newcommand\inlineeqnoa{(\theequation.\text{a})}
\newcommand\inlineeqnob{(\theequation.\text{b})}
\newcommand\inlineeqnoc{(\theequation.\text{c})}

\newcommand\inlineeqnowo{\stepcounter{equation}\ {(\theequation)}}
\newcommand\inlineeqnowoa{\theequation.\text{a}}
\newcommand\inlineeqnowob{\theequation.\text{b}}
\newcommand\inlineeqnowoc{\theequation.\text{c}}

\renewcommand{\refname}{Source}
\renewcommand{\sfdefault}{phv}
%\renewcommand*\contentsname{Contents}

\pagestyle{fancy}

\sloppy

\numberwithin{equation}{section}

%}}}

\begin{document}

%{{{ Titelseite

\title{physik221 $|$ Ferientutorium}
\author{Jonas Wortmann}
\maketitle
\pagenumbering{gobble}

%}}}

\newpage

%{{{ Inhaltsverzeichnis

\fancyhead[L]{\thepage}
\fancyfoot[C]{}
\pagenumbering{arabic}

\tableofcontents

%}}}

\newpage

%{{{

\fancyhead[R]{\leftmark\\\rightmark}

\newpage
\section{01.09.}
\subsection{Gedämpfter harmonischer Oszillator}
\begin{align} 
        F\left(x,\dot{x}\right)&=kx+2\gamma m\dot{x}&&\\
        \ddot{x}+2\gamma \dot{x}+\omega _0^2x&=0&\omega _0&=\,\sqrt[]{\dfrac{k}{m}}
\end{align} 
Der Lösungsansatz ist
\begin{align} 
        x\left(t\right)&=Ae^{\lambda _1t}+Be^{\lambda _2t}
.\end{align} 
$A$ und $B$ sind komplex; die Summe muss allerdings reell sein, da $x$ reell ist.

\subsection{Konservative Kräfte}
$\,\forall \gamma  \in \left[a,b\right]$ mit $\gamma \left(a\right)=\gamma \left(b\right)$ gilt
\begin{align} 
        \int_{\gamma }^{}\vv{F}\td x=\,\text{const.}\,
.\end{align} 
$\exists V:\vv{F}=-\vv{\nabla }V\land V\left(t\right)=V\left(0\right)$.

\subsection{Moden}
Moden bezeichnen die Eigenfrequenzen eines schwingfähigen Systems.

\subsection{Hamilton--Formalismus}
Der kanonische Impuls ist
\begin{align} 
        p_q=\diffp[]{\mathcal{L}}{\dot{q}}
.\end{align} 
Die Hamilton--Funktion ist die Legendre--Transformation der Lagrange--Funktion
\begin{align} 
        H=\sum_{i}^{}p_i\dot{q}_i-\mathcal{L}
.\end{align} 
Die Bewegungsgleichungen sind
\begin{align} 
        \dot{p}=-\diffp[]{\mathcal{H}}{q}\qquad \dot{q}=\diffp[]{\mathcal{H}}{p}
.\end{align} 

\subsection{Neother--Theorem}
{
        \centering
        \begin{tabulary}{\textwidth}{c|c}
                kontinuierliche Symmetrie&Erhaltungsgröße\\
                \hline
                Zeit&Energie\\
                Ort&Impuls\\
                Rotation&Drehimpuls\\
                Richtung des Perihels&Laplace--Runge--Lenz--Vektor
        \end{tabulary}
\par}

%}}}

\end{document}
