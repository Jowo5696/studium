\documentclass[a4paper,10pt]{article}

%{{{ packages
\usepackage{blindtext}
\usepackage{lipsum} % lorem ipsum
\usepackage[ngerman]{babel}
\usepackage[tiny]{titlesec} % section title size
\usepackage{index}
\usepackage[onehalfspacing]{setspace} % line spacing
\usepackage{fullpage} % makes document wider
\usepackage{pdfpages}
\usepackage{fancyhdr} % fancy headers
\usepackage{authblk} % author formatting
\usepackage{hyperref} % "clickable" references in text
\hypersetup{colorlinks=true,allcolors=blue}

\usepackage{booktabs} % toprule,midrule,bottomrule
\usepackage{multirow} % multiple rows in tabular
\usepackage{multicol} % multiple columns in tabular
\usepackage{longtable}
\usepackage{graphicx}
\usepackage{svg} % \includesvg{}
\usepackage{tikz}
\usepackage[european,siunitx]{circuitikz}
\usepackage{import} % can import files from other directories
\usepackage{pgfplots}
\usepackage{gnuplottex}
\usepackage{wrapfig} % let text wrap around figure

\usepackage{amsmath,amsfonts,amssymb}
\usepackage{tensor} % for right index order 
\usepackage{dsfont} % double stroke font
\usepackage{cancel} % to cancel in fraction
\usepackage{bm} % bold math font (if error is produced use \bm{{}})
\usepackage{mathtools}
\usepackage[ISO]{diffcoeff} % differentiation
\usepackage[locale=DE]{siunitx}
\usepackage[official]{eurosym}
\usepackage{mathrsfs} % calligraphy
\usepackage{physics}
\usepackage[a]{esvect}
\usepackage{bigints} % big integrals
%\usepackage[frak=esstix]{mathalpha} % disable if LaTeX uses too many alphabets
\usepackage[ruled,vlined,linesnumbered]{algorithm2e}
\usepackage{listings}

\usepackage{hyphsubst}
\usepackage{caption}
\usepackage{xr} % crossreferencing between documents \externaldocument{}
\usepackage{enumitem} % for enumerate environment
\usepackage{lineno}
\usepackage{makecell}
\usepackage{xcolor}
\usepackage{color}

\allowdisplaybreaks % allows equations to be broken (e.g. by multicols)

\usetikzlibrary{arrows}
\pgfplotsset{compat=1.15}

\newcommand{\td}{\,\text{d}}
\newcommand{\RN}[1]{\uppercase\expandafter{\romannumeral#1}}
\newcommand{\zz}{\mathrm{Z\kern-.3em\raise-0.5ex\hbox{Z} }}
\newcommand{\id}{1\kern-.258em1}

\newcommand\inlineeqno{\stepcounter{equation}\ {(\theequation)}}
\newcommand\inlineeqnoa{(\theequation.\text{a})}
\newcommand\inlineeqnob{(\theequation.\text{b})}
\newcommand\inlineeqnoc{(\theequation.\text{c})}

\newcommand\inlineeqnowo{\stepcounter{equation}\ {(\theequation)}}
\newcommand\inlineeqnowoa{\theequation.\text{a}}
\newcommand\inlineeqnowob{\theequation.\text{b}}
\newcommand\inlineeqnowoc{\theequation.\text{c}}

\renewcommand{\refname}{Source}
\renewcommand{\sfdefault}{phv}

\iffalse\newenvironment{Figure}
  {\par\medskip\noindent\minipage{\linewidth}}
  {\endminipage\par\medskip}\fi % for multicols figures

\pagestyle{fancy}

\sloppy % block text

\numberwithin{equation}{section}

\titleformat{\subsection}{}{\thesubsection}{1em}{\itshape}
\titleformat{\subsubsection}{}{\thesubsubsection}{1em}{\itshape}

%}}}

\begin{document}

%{{{ Titelseite

\begin{titlepage}
  \title{Übersichtsprüfung Theorie 1 \& 4\\\small Note 1.0\normalsize}
  \author{Stefan Förste}
\end{titlepage}

\maketitle
\pagenumbering{gobble}

%\renewcommand\abstractname{Abstract}
%\abstract{\noindent}

%}}}

%\clearpage

\iffalse
%{{{ Inhaltsverzeichnis
\fancyhead[R]{\leftmark}
%\fancyhead[R]{\leftmark\\rightmark}
\fancyhead[L]{\thepage}
\fancyfoot[C]{}

%\renewcommand*\contentsname{Contents}
\tableofcontents

%}}}
\fi

%\clearpage

\pagenumbering{arabic}

%{{{

%\begin{multicols}{2}
Die Prüfungsatmosphäre ist sehr angenehm und die gestellten Fragen sind klar formuliert.
Oft sind keine expliziten Herleitungen nötig, es reicht diese zu mündlich zu erklären.
Die Folgenden Themen wurden abgefragt, jeweils in der Tiefe, dass man auf eine Prüfungszeit von ca.\ 35 Minuten kommt.
\begin{enumerate}[label=--]
  \item Hamilton'sches Prinzip
  \item Euler-Lagrange Gleichung
  \item Noether Theorem, Noether Ladung
  \item Impulserhaltung und Transformation (berechnen durch geeignete Transformation)
  \item Energieerhaltung durch Zeitinvarianz / Hamilton Funktion
  \item Hamilton Funktion und Bewegungsgleichungen
  \item Newton Inertialsystem, beschleunigte Bezugssysteme, Scheinkräfte
  \item Ergodenhypothese
  \item Gibbs-Paradoxon und Besetzungszahldarstellung
  \item Unterschied Mikro- und Makrozustand
  \item Mögliche Mikrozustände / Sind alle Mikrozustände gleich dem Makrozustand? 
    Antwort: Es besteht keine eins zu eins Korrespondenz zwischen den Mikro- und Makrozuständen. 
    Die Mikrozustände sind beschränkt auf die zum Ensemble passende Hyperebene in Phasenraum. 
    \footnote{Hier die Antwort, da die Frage nicht präzise formuliert ist. In der Prüfung hat es ein wenig länger gedauert bis wir uns verständigen konnten.}
  \item mikrokanonisch, kanonisch, großkanonisches Ensemble: Zustandsvariablen, Wahrscheinlichkeitsverteilung, Zustandssumme, Entropie, Potential
  \item erster und zweiter Hauptsatz der Thermodynamik
  \item Gibbs-Duhem Relation
  \item Carnot in $(p,V)$ und $(T,S)$ Diagram und Wirkungsgrad herleiten
\end{enumerate}
Die Vorbereitungszeit betrug 114.5 Stunden und zwei gehaltene Theorie 1 Tutorien.
Verwendetes Material ist für Theorie 1: Hanhart \& Kubis Skript SS25, Kuypers; für Theorie 4: Pathria, Borghini Skript (uni Bielefeld), Kroha Skript WS24/25.
Zum Üben empfiehlt es sich von einem LLM Fragen zum Thema gestellt zu bekommen.
Der Vorbereitungsfortschritt/Wissenstand sollte allerdings so weit sein, dass Fehler des LLMs erkannt werden können.
%\end{multicols}

%\clearpage
%\listoffigures
%\listoftables
%\bibliographystyle{plain}
%\bibliography{refs}

%}}}

\end{document}
