%{{{ Formatierung

\documentclass[a4paper,12pt]{article}
%twocolumn

\usepackage{physics_notetaking}

%%% dark red
%\definecolor{bg}{RGB}{60,47,47}
%\definecolor{fg}{RGB}{255,244,230}
%%% space grey
%\definecolor{bg}{RGB}{46,52,64}
%\definecolor{fg}{RGB}{216,222,233}
%%% purple
%\definecolor{bg}{RGB}{69,0,128}
%\definecolor{fg}{RGB}{237,237,222}
%\pagecolor{bg}
%\color{fg}

\newcommand{\td}{\,\text{d}}
\newcommand{\RN}[1]{\uppercase\expandafter{\romannumeral#1}}
\newcommand{\zz}{\mathrm{Z\kern-.3em\raise-0.5ex\hbox{Z} }}

\newcommand\inlineeqno{\stepcounter{equation}\ {(\theequation)}}
\newcommand\inlineeqnoa{(\theequation.\text{a})}
\newcommand\inlineeqnob{(\theequation.\text{b})}
\newcommand\inlineeqnoc{(\theequation.\text{c})}

\newcommand\inlineeqnowo{\stepcounter{equation}\ {(\theequation)}}
\newcommand\inlineeqnowoa{\theequation.\text{a}}
\newcommand\inlineeqnowob{\theequation.\text{b}}
\newcommand\inlineeqnowoc{\theequation.\text{c}}

\renewcommand{\refname}{Source}
\renewcommand{\sfdefault}{phv}
%\renewcommand*\contentsname{Contents}

\pagestyle{fancy}

\sloppy

\numberwithin{equation}{section}

%}}}

\begin{document}

%{{{ Titelseite

\title{physik321 $|$ Notizen}
\author{Jonas Wortmann}
\maketitle
\pagenumbering{gobble}

%}}}

\newpage

%{{{ Inhaltsverzeichnis

\fancyhead[L]{\thepage}
\fancyfoot[C]{}
\pagenumbering{arabic}

\tableofcontents

%}}}

\newpage

%{{{

\fancyhead[R]{\leftmark\\\rightmark}

\section{Überblick Elektrodynamik}
Das Ziel ist die Untersuchung der Ursache und Wirkung von elektrischen ($\vv{E}$) und magnetischen ($\vv{B}$) Feldern auf elektrische Ladungen ($q$).\\\indent
Aus der experimentellen Beobachtung ist bekannt, dass auf elektrisch geladene Körper eine elektromagnetische Kraft 
\begin{align} 
        \vv{F}=q\left(\vv{E}+\vv{v}\times \vv{B}\right)
,\end{align} 
mit $\vv{v}$ der Geschwindigkeit des geladenen Teilchens, wirkt. Diese Kraft führt zu einer Bewegungsänderung
\begin{align} 
        \vv{F}=\diff*[]{\left[\dfrac{m\vv{v}}{\,\sqrt[]{1-\tfrac{\vv{v}^2}{c^2}} }\right]}{t}
.\end{align} 
Der Zusammenhang zwischen elektrischen und magnetischen Feldern und (bewegten) Ladungen sind die \textbf{Maxwell--Gleichungen}
\begin{align} 
        \,\text{div}\,\vv{E}&=\dfrac{\rho }{\varepsilon _0}&\,\text{rot}\,\vv{E}&=-\diffp[]{\vv{B}}{t}\\
        \,\text{div}\,\vv{B}&=0&c^2\,\text{rot}\,\vv{B}&=\diffp[]{\vv{E}}{t}+\dfrac{\vv{j}}{\varepsilon _0}
.\end{align} 
Zusammen mit Randbedingungen an \textbf{Grenzflächen} bestimmen sie alle Effekte der Elektrodynamik.

\newpage
\section{Statische Felder}
Die \textsc{Maxwell}--Gleichungen für statische Felder sind
\begin{align} 
        \,\text{div}\,\vv{E}&=\dfrac{\rho }{\varepsilon _0}&\,\text{rot}\,\vv{E}&=0\\
        \,\text{div}\,\vv{B}&=0&c^2\,\text{rot}\,\vv{B}&=\dfrac{\vv{j}}{\varepsilon _0}
.\end{align} 
Man kann sehen, dass die Gleichungen für statische Felder entkoppeln und sie sich in \textbf{Elektrostatik} und \textbf{Magnetostatik} aufteilen.

\subsection{Eletrostatik}
Die Grundgrößen der klassischen Mechanik sind die Masse, Länge und Zeit.
Eine wichtige Grundgröße in der Elektrostatik ist die elektrische Ladung. 
Aus der experimentellen Beobachtung ist bekannt, dass Körper in einen elektrischen Zustand versetzt werden können (z.B.\ können geladene Körper andere geladene Körper anziehen).
Dieses Phänomen ist mechanisch nicht erklärbar.
Dieser Zustand ist auch auf andere Körper übertragbar, woraus folgt, dass es sich um eine substanzartige Größe handeln muss.
\\\indent Diese Größe ist die \textbf{elektrische Ladung} $q$.
Bei der Übertragung fließt ein elektrischer Strom $I$.
Die Ladung des Elektrons ist negativ, also $q<0$.
Zudem ist sie additiv, es existiert also die Gesamtladung $Q=\sum_{i}^{}q_i$.
In abgeschlossenen Systemen ist die Summe aus positiven und negativen Ladungen konstant.
Die Ladungen sind \textbf{gequantelt}, es existiert also eine nicht teilbare Elementarladung $e$, also gilt immer, dass $q=n\cdot e,n  \in \mathbb{Z}$
\begin{align*} 
        \,\text{Elektron}\,&:n=-1\\
        \,\text{Proton}\,&:n=+1\\
        \,\text{Neutron}\,&:n=0\\
        \,\text{Atomkern}\,&:n=Z
.\end{align*} 
In der Elektrodynamik wird dieses Prinzip allerdings verallgemeinert.
Man führt die Ladungsdichte $\rho \left(\vv{r}\right)$ ein, also $Q=\int_{V}^{}\rho \left(\vv{r}\right)\td ^3r$.
Für Punktladungen gilt dann $\rho \left(\vv{r}\right)=q\delta \left(\vv{r}-\vv{r}_0\right)$.\\\indent
Befinden sich zwei Ladungen $q_1$ und $q_2$ im Abstand von einem Meter im Vakuum, dann wirkt eine Kraft von
\begin{align} 
        F&=\dfrac{10^{12}}{4\pi \cdot 8,854}\,\unit{N}
.\end{align} 
Dann haben $q_1$ und $q_2$ eine Ladung von $|q_1|=|q_2|=\SI{1}{C}$.
Die Elementarladung ist $e=\SI{1,602e-19}{C}$.\\\indent
\\\hfill\\\textbf{Stromdichte}\\ 
Für bewegte Ladungen existiert die Stromdichte $\vv{j}$.
Sie gibt die Ladung pro Zeiteinheit durch eine Flächeneinheit senkrecht zur Stromrichtung an.
Betrachte als Beispiel eine homogene Ladungsverteilung von $N$ Teilchen mit Ladung $q$ und Geschwindigkeit $\vv{v}$, dann ist die Ladungsdichte $\vv{j}=\tfrac{N}{V}q\vv{v}$. 
Die Stromstärke $I$ ist dann $I=\int_{\mathcal{F}}^{}\vv{j}\left(\vv{r}\right)\td \vv{f}=\int_{\mathcal{F}}^{}\vv{j}\left(\vv{r}\right)\vv{n}\left(\vv{r}\right)\td f$.
Die Einheit ist $\SI{1}{A}$, was einem Ladungstransport von $\SI{1}{C}$ in einer Sekunde entspricht.
\\\hfill\\\textbf{Ladungserhaltung}\\ 
Die Ladungserhaltung kann mit Hilfe der \textbf{Kontinuitätsgleichung} beschrieben werden
\begin{align} 
        \diffp[]{\rho }{t}=\,\text{div}\,\vv{j}
.\end{align} 

\subsubsection{Coulomb'sche Gesetz}
Zwei Ladungen $q_1$ und $q_2$ befinden sich im Abstand $\vv{r}_1$ und $\vv{r}_2$ zum Ursprung.
Der Abstand zwischen diesen Ladungen ist $\vv{r}_{12}$.
Dieser Abstand soll viel größer sein also die Ausdehnung von $q_1$ und $q_2$.
Die Kraft zwischen diesen Ladungen ist
\begin{align} 
        \vv{F}_{12}=kq_1q_2\dfrac{\vv{r}_1-\vv{r}_2}{|\vv{r}_1-\vv{r}_2|^3}=-\vv{F}_{21}
.\end{align} 
Sie ist also direkt proportional zu $q_1$ und $q_2$, $|\vv{F}_{12}|\propto |\vv{r}_1-\vv{r}_2|^{-2}$, sie wirkt entlang von $F_{12}$.
Diese Kraft gilt nur für \textbf{ruhende} Ladungen.
\\\hfill\\\textbf{Experimentelle Tatsachen}\\ 
Die Proportionalitätskonstante ist $k=\tfrac{1}{4\pi \varepsilon _0}$.\\\indent
Das Superpositionsprinzip erlaubt es die Kraft auf mehrere Ladungen zu berechnen
\begin{align} 
        \vv{F}_1=kq_1\sum_{j=2}^{n}q_j\dfrac{\vv{r}_1-\vv{r}_j}{|\vv{r}_1-\vv{r}_j|^3}
.\end{align} 

\subsubsection{Das elektrische Feld}
Das elektrische Feld, bzw.\ die Feldlinien, erlauben eine Abstraktion der Kraftwirkung. Für Punktladungen ist das Feld
\begin{align} 
        \vv{E}=\dfrac{1}{4\pi \varepsilon _0}\sum_{j=1}^{n}q_j\dfrac{\vv{r}_1-\vv{r}_j}{|\vv{r}_1-\vv{r}_j|^3}
.\end{align} 
Für kontinuierliche Ladungsverteilungen gilt
\begin{align} 
        \vv{E}\left(\vv{r}\right)=\dfrac{1}{4\pi \varepsilon _0}\int_{}^{}\td ^3r'\rho \left(r'\right)\dfrac{\vv{r}-\vv{r}'}{|\vv{r}-\vv{r}'|^3}
.\end{align} 
Der Integrand kann umgeschrieben werden als 
\begin{align} 
        \dfrac{\vv{r}-\vv{r}'}{|\vv{r}-\vv{r}'|^3}=-\vv{\nabla }\dfrac{1}{|\vv{r}-\vv{r}'|}
,\end{align} 
also ist $\vv{E}$ ein Gradientenfeld des skalaren Potentials
\begin{align} 
        \varphi \left(\vv{r}\right)=\dfrac{1}{4\pi \varepsilon _0}\int_{}^{}\dfrac{\rho \left(\vv{r}'\right)}{|\vv{r}-\vv{r}'|}\td ^3r\qquad \vv{E}\left(\vv{r}\right)=-\vv{\nabla }\varphi \left(\vv{r}\right)
.\end{align} 
Die \textsc{Poisson}--Gleichung ist
\begin{align} 
        \,\text{div}\,\vv{E}=-\,\text{div}\,\,\text{grad}\,\varphi =\dfrac{\rho }{\varepsilon _0}
.\end{align} 
Die Lösung dieser Gleichung ist das Grundproblem der Elektrostatik.\\\indent
Die Äquipotentialflächen sind konstant, also $\varphi \left(\vv{r}\right)=\,\text{const.}\,$. Die Coulomb--Kraft ist konservativ $\,\text{rot}\,q\vv{E}=0$. Da die Kraft $\vv{F}=-\vv{\nabla }V$ ist das Potential $V\left(\vv{r}\right)=q\varphi \left(\vv{r}\right)$.\\\indent
Das Linienintegral über $\vv{E}$ ist wegabhängig
\begin{align} 
        \varphi \left(\vv{r}\right)-\varphi \left(\vv{r}_0\right)&=-\int_{\vv{r}_0}^{\vv{r}}\vv{E}\left(\vv{r}'\right)\td \vv{r}'=U\left(\vv{r},\vv{r}_0\right)
.\end{align} 
Für $n$ Punktladungen gilt 
\begin{align} 
        \varphi \left(\vv{r}\right)=\dfrac{1}{4\pi \varepsilon _0}\int_{}^{}\td ^3r'\dfrac{\sum_{j=1}^{n}q_j\delta \left(\vv{r}'-\vv{r}_j\right)}{|\vv{r}-\vv{r}'|}\qquad \rho \left(\vv{r}\right)=\sum_{j=1}^{n}q_j\delta \left(\vv{r}-\vv{r}_j\right)
.\end{align} 

\subsubsection{Beispiel: Homogen geladene Kugel}
Der Ursprung wird in das Zentrum der geladenen Kugel mit Radius $R$ gelegt. Eine Probeladung befindet sich im Abstand $\vv{r}$ zum Zentrum. Die Ladungsdichte der Kugel ist
\begin{align} 
        \rho \left(\vv{r}'\right)=\begin{cases}
                \rho _0&\,\text{für}\,|\vv{r}'|=r'\leq R\\
                0&\,\text{sonst}\,
        \end{cases}
.\end{align} 
Das Potential ist dann
\begin{align} 
        \varphi \left(\vv{r}\right)&=\dfrac{\rho _0}{4\pi \varepsilon _0}\int_{\mathcal{K}\left(R\right)}^{}\dfrac{1}{|\vv{r}-\vv{r}'|}\td ^3r'\\
                                   &=\dfrac{\rho _0}{4\pi \varepsilon _0}\int_{0}^{R}r'^2\td r'\int_{0}^{2\pi }\td \varphi '\int_{0}^{\pi }\sin \theta '\td \theta '\dfrac{1}{\,\sqrt[]{r^2+r'^2-2rr'\cos \theta '}}\nonumber \\
                                   &=\dfrac{\rho _0}{4\pi \varepsilon _0}2\pi \int_{0}^{R}r'^2\td r'\left.\dfrac{1}{rr'}\,\sqrt[]{r^2+r'^2-2rr'\cos \theta '}\right|_0^\pi \nonumber \\
                                   &=\dfrac{\rho _0}{4\pi \varepsilon _0}\dfrac{2\pi }{r}\int_{0}^{R}r'\td r'\left[|r+r'|-|r-r'|\right]\nonumber \\
                                   &=\dfrac{\rho _0}{4\pi \varepsilon _0}\dfrac{2\pi }{r}\int_{0}^{R}\td r'\begin{cases}
                                           2rr'&,r\leq r'\\
                                           2r'^2&,r>r'
                                   \end{cases}
.\end{align} 
Für $\vv{r}$ außerhalb von $\mathcal{K}\left(R\right)$, also $r>R$ 
\begin{align} 
        \varphi \left(\vv{r}\right)=\dfrac{\rho _0}{4\pi \varepsilon _0}\dfrac{2\pi }{r}\int_{0}^{R}2r'^2\td r'=\dfrac{\rho _0}{4\pi \varepsilon _0}\dfrac{4\pi R^3}{3}\dfrac{1}{r}=\dfrac{Q}{4\pi \varepsilon _0}\dfrac{1}{r}
.\end{align} 
Für $\vv{r}$ innerhalb von $\mathcal{K}\left(R\right)$, also $r<R$ 
\begin{align} 
        \varphi \left(\vv{r}\right)&=\dfrac{\rho _0}{4\pi \varepsilon _0}\dfrac{2\pi }{r}\left[\int_{0}^{r}2r'^2\td r'+\int_{r}^{R}2rr'\td r'\right]=\dfrac{\rho _0}{4\pi \varepsilon _0}\dfrac{2\pi }{r}\left[\dfrac{2r^3}{3}+r\left(R^2-r^2\right)\right]\\
                                   &=\dfrac{Q}{4\pi \varepsilon _0}\left(3R^2-r^2\right)\dfrac{1}{2R^3}
.\end{align} 
Das elektrische Feld außerhalb ($r>R$) bzw.\ innerhalb ($r<R$) ist
\begin{align} 
        \vv{E}\left(\vv{r}\right)&=-\dfrac{Q}{4\pi \varepsilon _0}\vv{\nabla }\dfrac{1}{r}=\dfrac{Q}{4\pi \varepsilon _0}\dfrac{\vv{r}}{r^3}& \vv{E}\left(\vv{r}\right)&=-\dfrac{Q}{4\pi \varepsilon _0}\dfrac{1}{2R^3}\vv{\nabla }\left(3R^2-r^2\right)\\
                                 &&&=\dfrac{Q}{4\pi \varepsilon _0}\dfrac{r}{R^3}\dfrac{\vv{r}}{|\vv{r}|}\nonumber \\
                                 &&&=\dfrac{Q\left(r\right)}{4\pi \varepsilon _0}\dfrac{r^3}{R^3}\dfrac{\vv{r}}{r^3}
.\end{align} 

\subsection{Mittlere Quelldichte -- Satz von \textsc{Gauss}}
Sei ein Volumen mit den Kanten $\Delta x,\Delta y$ und $\Delta z$. Der Mittelpunkt ist $\vv{r}_0\left(\vv{x}_0,\vv{y}_0,\vv{z}_0\right)$. Die Flächennormalen sind $\Delta \vv{f}_1,\hdots ,\Delta \vv{f}_6$ ($\vv{f}_1$ zeigt entlang der $\Delta x$--Achse)
\begin{align} 
        \Delta \vv{f}_1&=\Delta y\Delta x\vv{e}_x=-\Delta \vv{f}_2\\
        \Delta \vv{f}_3&=\Delta x\Delta z\vv{e}_y=-\Delta \vv{f}_4\\
        \Delta \vv{f}_5&=\Delta x\Delta y\vv{e}_z=-\Delta \vv{f}_6
.\end{align} 
Das elektrische Feld, welches den Quader durchsetzt ist
\begin{align} 
        \oint_{}^{}\vv{E}\left(\vv{r}\right)\td \vv{f}&=\iint_{}^{}\td y\td z\left[E_x\left(x_0+\dfrac{\Delta x}{2},y,z\right)-E_x\left(x_0-\dfrac{\Delta x}{2},y,z\right)\right]\\
                                                      &+\iint_{}^{}\td x\td z\left[E_y\left(x,y_0+\dfrac{\Delta y}{2},z\right)-E_y\left(x,y_0-\dfrac{\Delta y}{2},z\right)\right]\\
                                                      &+\iint_{}^{}\td x\td y\left[E_z\left(x,y,z_0+\dfrac{\Delta z}{2}\right)-E_z\left(x,y,z_0-\dfrac{\Delta z}{2}\right)\right]
.\end{align} 
Mit Hilfe einer Taylorentwicklung (da $\Delta x,\Delta y$ und $\Delta z$ klein sind),
\begin{align} 
        &=\int_{}^{}\td y\td z\left(\diffp[]{E_x}{x}\left(x_0,y,z\right)\Delta x+\mathcal{O}\left(\Delta x^3\right)\right)\\
        &+\int_{}^{}\td x\td z\left(\diffp[]{E_y}{y}\left(x,y_0,z\right)\Delta y+\mathcal{O}\left(\Delta y^3\right)\right)\\
        &+\int_{}^{}\td x\td y\left(\diffp[]{E_z}{z}\left(x,y,z_0\right)\Delta z+\mathcal{O}\left(\Delta z^3\right)\right)
.\end{align} 
Der Mittelwertsatz der Integralrechnung besagt
\begin{align} 
        \dfrac{1}{\Delta V}\oint_{S\left(V\right)}^{}\vv{E}\td \vv{f}&=\partial x E_x\left(x_0,y_1,z_1\right)+\mathcal{O}\left(\Delta x^2\right)\\
                                                                     &+\partial y E_y\left(x_2,y_0,z_2\right)+\mathcal{O}\left(\Delta y^2\right)\\
                                                                     &+\partial z E_z\left(x_3,y_3,z_0\right)+\mathcal{O}\left(\Delta z^2\right)
.\end{align} 
Der Limes von $\Delta V\rightarrow 0$ gibt dann
\begin{align} 
        \lim_{\Delta V\rightarrow 0}\dfrac{1}{\Delta V}\oint_{S\left(V\right)}^{}\vv{E}\td \vv{f}=\,\text{div}\,\vv{E}\left(\vv{r}_0\right)
.\end{align} 
Beliebige Volumina können mit Quadern ausgeschöpft werden. Flächenintegrale über gemeinsame Grenzflächen heben sich auf. Am Ende bleibt nur die äußere Grenzfläche übrig. Damit ist der Satz von \textsc{Gauss}
\begin{align} 
        \oint_{S\left(V\right)}^{}\vv{E}\left(\vv{r}\right)\td \vv{f}=\int_{V}^{}\,\text{div}\,\vv{E}\left(\vv{r}\right)\td V
.\end{align} 
Wendet man diesen Satz auf die Elektrostatik an, folgt
\begin{align} 
        \oint_{S\left(V\right)}^{}\vv{E}\left(\vv{r}\right)\td \vv{f}&=\dfrac{1}{4\pi \varepsilon _0}\int_{}^{}\rho \left(\vv{r}'\right)\td ^3r'\int_{S\left(V\right)}^{}\td \vv{f}\dfrac{\vv{r}-\vv{r}'}{|\vv{r}-\vv{r}'|^3}\\
                                                                     &=\dfrac{1}{4\pi \varepsilon _0}\int_{}^{}\rho \left(\vv{r}'\right)\td ^3r'\int_{S\left(V\right)}^{}\td \vv{f}\left(-\,\text{grad}\,\dfrac{1}{|\vv{r}-\vv{r}'|}\right)\\
                                                                     &=-\dfrac{1}{4\pi \varepsilon _0}\int_{}^{}\rho \left(\vv{r}'\right)\td ^3r'\int_{V}^{}\td ^3r\,\text{div}\,\,\text{grad}\,\dfrac{1}{|\vv{r}-\vv{r}'|}
.\end{align}
Es gilt $\Delta \tfrac{1}{|\vv{r}-\vv{r}'|}=-4\pi \delta \left(\vv{r}-\vv{r}'\right)$. Damit kann geschrieben werden
\begin{align} 
        &=\dfrac{1}{\varepsilon _0}\int_{}^{}\rho \left(\vv{r}'\right)\td ^3r'\int_{V}^{}\delta \left(\vv{r}-\vv{r}'\right)\td V\\
        &=\dfrac{1}{\varepsilon _0}\int_{}^{}\rho \left(\vv{r}\right)\td ^3r=\dfrac{1}{\varepsilon _0}Q\left(V\right)\\
        &=\int_{V}^{}\,\text{div}\,\vv{E}\left(\vv{r}\right)\td ^3r
.\end{align} 
Aus diesem Ausdruck lässt sich für beliebige Volumina $V$ sagen
\begin{align} 
        \int_{V}^{}\left(\,\text{div}\,\vv{E}\left(\vv{r}\right)-\dfrac{\rho \left(\vv{r}\right)}{\varepsilon _0}\right)\td ^3r&=0
.\end{align}
Daraus folgen die \textsc{Maxwell}'schen Gleichungen
\begin{align} 
        \,\text{div}\,\vv{E}\left(\vv{r}\right)&=\dfrac{\rho \left(\vv{r}\right)}{\varepsilon _0}&\oint_{S\left(V\right)}^{}\vv{E}\left(\vv{r}\right)\td \vv{f}&=\dfrac{Q\left(V\right)}{\varepsilon _0}
.\end{align} 
Da $\vv{E}=-\,\text{grad}\,\varphi $ ein Gradientenfeld ist, gilt
\begin{align} 
        \,\text{rot}\,\,\text{grad}\,\varphi =0\Rightarrow \,\text{rot}\,\vv{E}=0
.\end{align} 

\subsubsection{Beispiel: Bestimmung von $E$ mit \textsc{Gauss}'schem Satz}
Sei eine homogen geladene Kugel mit Radius $R$ und einem elektrischem Feld von
\begin{align} 
        \vv{E}\left(\vv{r}\right)&=E_r\left(r,\varphi ,\theta \right)\vv{e}_r+E_\varphi \left(r,\varphi ,\theta \right)\vv{e}_\varphi +E_\theta \left(r,\varphi ,\theta \right)\vv{e}_\theta 
.\end{align} 
Die \textsc{Maxwell}--Gleichung besagt,
\begin{align} 
        \oint_{S\left(V\right)}^{}\vv{E}\left(\vv{r}\right)\td \vv{f}&=\dfrac{q\left(V\right)}{\varepsilon _0}
\end{align} 
Die Kugel ist rotationssymmetrisch um die $x$--,$y$-- und $z$--Achse, also unabhängig von $\theta $ und $\varphi $. Das Feld ist also nur noch von Radius $r$ abhängig.\\\indent
Wird an der $x$--$y$--Ebene gespiegelt, wird $\theta $ zu $\pi -\theta $, also $E_z=E_r\left(r\right)\cos \theta -E_\theta \left(r\right)\sin \theta $. Zudem gilt $\cos \left(\pi -\theta \right)=-\cos \left(\theta \right)$ und $\sin \left(\pi -\theta \right)=\sin \left(\theta \right)$. Aus dieser Spiegelung folgt, dass $E_z\rightarrow -E_z$ und $E_\theta =0$. Aus den Spiegelungen an der $x$--$z$-- und $y$--$z$--Ebene kann man analog sagen, dass $E_\varphi =0$.\\\indent
Für das gesamte Feld folgt dann
\begin{align} 
        \vv{E}\left(\vv{r}\right)=E_r\left(r\right)\vv{e}_r
.\end{align} 
Jetzt kann der Satz von \textsc{Gauss} angewendet werden
\begin{align} 
        \int_{S\left(V_r\right)}^{}\vv{E}\left(\vv{r}\right)\td \vv{f}=E_r4\pi r^2=\dfrac{q\left(V_r\right)}{\varepsilon _0}=\begin{cases}
                \dfrac{Q}{\varepsilon _0}&,r>R\\
                \dfrac{Q}{\varepsilon _0}\dfrac{r^3}{R^3}&,r<R
        \end{cases}
.\end{align} 
Das Feld ist also
\begin{align} 
        \vv{E}\left(\vv{r}\right)&=\dfrac{Q}{4\pi \varepsilon _0}\left\{\begin{matrix}
                        \dfrac{1}{r^2}&,r>R\\
                        \dfrac{r}{R^3}&,r<R
        \end{matrix}\right\}\vv{e}_r
.\end{align} 

\subsection{Integraldarstellung -- Satz von \textsc{Stokes}}
Für die Integraldarstellung der \textsc{Maxwell}'schen Gleichungen wird die Zirkulation eines Vektorfeldes $\vv{a}\left(\vv{r}\right)$ entlang einer geschlossenen Kurve $C$ verwendet
\begin{align} 
        Z&=\oint_{C}^{}\vv{a}\left(\vv{r}\right)\td \vv{r}
.\end{align} 
Sei ein geschlossener Weg in der $xy$--Ebene mit Mittelpunkt $\vv{r}_0$. Die Fläche ist $\Delta F=\Delta x\Delta y$ und die Normale $\vv{n}=\vv{e}_z$ 
\begin{align} 
        Z&=\int_{x_0-\tfrac{\Delta x}{2}}^{x_0+\tfrac{\Delta x}{2}}\left[a_x\left(x,y_0-\dfrac{\Delta y}{2},z_0\right)-a_x\left(x,y_0+\dfrac{\Delta y}{2},z_0\right)\right]\td x\\
         &+\int_{y_0-\tfrac{\Delta y}{2}}^{y_0+\tfrac{\Delta y}{2}}\left[a_y\left(x_0-\dfrac{\Delta x}{2},y,z_0\right)-a_y\left(y+\dfrac{\Delta y}{2},y,z_0\right)\right]\td y\\
         &=\int_{x_0-\tfrac{\Delta x}{2}}^{x_0+\tfrac{\Delta x}{2}}\left[-\partial _ya_x\left(x,y_0,z_0\right)\Delta y+\mathcal{O}\left(\Delta y^3\right)\right]\td \\
         &=\int_{y_0-\tfrac{\Delta y}{2}}^{y_0+\tfrac{\Delta y}{2}}\left[-\partial _xa_y\left(x_0,y,z_0\right)\Delta x+\mathcal{O}\left(\Delta x^3\right)\right]\td 
.\end{align} 
Betrachtet man dann die Fläche
\begin{align} 
        \lim_{\Delta x,\Delta y\rightarrow 0}\dfrac{Z}{\Delta F}&=\left(\partial _xa_y\left(\vv{r}_0\right)-\partial _ya_x\left(\vv{r}_0\right)\right)\\
                                                         &=\left(\,\text{rot}\,\vv{a}\left(\vv{r}_0\right)\right)_z
.\end{align} 
Daraus folgt der \textbf{Satz von} \textsc{Stokes}
\begin{align} 
        \oint_{\partial F}^{}\vv{a}\left(\vv{r}\right)\td \vv{r}=\int_{F}^{}\,\text{rot}\,\left(\vv{a}\left(\vv{r}\right)\right)\td \vv{f}
.\end{align} 
Mit diesem Satz sind die \textsc{Maxwell}'schen Gleichungen dann
\begin{align} 
        \,\text{rot}\,\vv{E}&=0&\oint_{C}^{}\vv{E}\left(\vv{r}\right)\td \vv{r}=0
.\end{align} 

\subsection{Punktladungen}
Die Ladungsdichte einer Punktladung kann mit Hilfe der Delta--Distribution dargestellt werden (kartesische oder Kugelkoordinaten)
\begin{align} 
        \rho \left(\vv{r}\right)=q\delta \left(\vv{r}-\vv{r}_0\right)\qquad \rho \left(\vv{r}\right)=q\dfrac{1}{r_0^2\sin \left(\theta _0\right)}\delta \left(r-r_0\right)\delta \left(\varphi -\varphi _0\right)\delta \left(\theta -\theta _0\right)
.\end{align} 
Das Potential lässt sich schreiben als
\begin{align} 
        \varphi \left(\vv{r}\right)&=\int_{\mathbb{R}^3}^{}\dfrac{1}{4\pi \varepsilon _0}\dfrac{q}{|\vv{r}-\vv{r}'|}\delta \left(\vv{r}_0-\vv{r}'\right)\td ^3r'\\
                                   &=\dfrac{q}{4\pi \varepsilon _0}\dfrac{1}{|\vv{r}-\vv{r}_0|}
.\end{align} 

\subsubsection{Beispiel: Kugeloberfläche}
Sei eine Kugeloberfläche mit Ladungsdichte $\rho \left(\vv{r}\right)=\sigma$. Die Gesamtladung in Kugelkoordinaten im Abstand $R$ vom Zentrum ist
\begin{align} 
        Q=\int_{\mathbb{R}^3}^{}\delta \left(r-R\right)\sigma r^2\sin \theta \td \theta \td \varphi \td r=2\pi 2\sigma R^2=4\pi R^2\sigma 
.\end{align} 
Die Ladung einer Punktladung in Kugelkoordinaten ist
\begin{align} 
        \int_{\mathbb{R}^3}^{}q\dfrac{1}{r_0^2\sin \theta _0}\delta \left(r-r_0\right)\delta \left(\varphi -\varphi _0\right)\delta \left(\theta -\theta _0\right)r^2\sin \theta \td \theta \td \varphi \td r&=q\dfrac{1}{r_0^2\sin \theta _0}r_0^2\sin \theta _0\\
                                                                                    &=q
.\end{align} 

\subsection{Feldverhalten an Grenzflächen}
\subsubsection{\textsc{Gauss}'sche Fläche}
Man betrachtet ein \textsc{Gauss}'sches Kästchen mit der Höhe von $\Delta x$ auf einer Grenzfläche mit Normale $\vv{n}$ eines elektrischen Feldes. Die Normalen $\td \vv{f}$ des Kästchens sind orthogonal zu der Grenzfläche. Es existiert ein Feld außerhalb der Fläche $\vv{E}_a$ und innerhalb der Fläche $\vv{E}_i$. Auf der Grenzfläche befindet sich die Flächenladungsdichte $\sigma $. Hier wird der Satz von \textsc{Gauss} verwendet
\begin{align} 
        \int_{\Delta V}^{}\,\text{div}\,\vv{E}\left(\vv{r}\right)\td ^3r&=\int_{S\left(\Delta V\right)}^{}\vv{E}\td \vv{f}\rightarrow \lim_{\Delta x\rightarrow 0}\Delta F\vv{n}\left(\vv{E}_a-\vv{E}_i\right)
.\end{align} 
Wird $\Delta x$ gegen 0, dann bleibt nur noch die Fläche $\Delta F\vv{n}\left(\vv{E}_a-\vv{E}_i\right)$ übrig.
\begin{align} 
        \int_{\Delta V}^{}\,\text{div}\,\vv{E}\left(\vv{r}\right)\td ^3r=\dfrac{1}{\varepsilon _0}\int_{\Delta V}^{}\rho \left(\vv{r}\right)\td ^3r=\dfrac{\sigma }{\varepsilon _0}\Delta F
.\end{align} 
Insgesamt ist also
\begin{align} 
        E_a^n-E_i^n=\dfrac{\sigma }{\varepsilon _0}
.\end{align} 
Die Normalkomponente an der Grenzfläche ist unstetig, macht also einen Sprung, wenn $\sigma \neq 0$.

\subsubsection{\textsc{Stokes}'sche Fläche}
Man betrachtet jetzt eine \textsc{Stokes}'sche Fläche auf der selben Grenzfläche. Die Höhe der Fläche ist $\Delta x$. Die Breite der Fläche ist $\Delta l_i$ und $\Delta l_a$. Die Normale $\Delta \vv{F}=\vv{t}\Delta F$ der Fläche liegt in der Grenzfläche. Es gilt $\Delta \vv{l}_a=\Delta l\left(\vv{t}\times \vv{n}\right)=-\Delta \vv{l}_i$. Der \textsc{Stokes}'sche Satz besagt
\begin{align} 
        0=\int_{\Delta F}^{}\,\text{rot}\,\vv{E}\left(\vv{r}\right)\td \vv{f}=\int_{\partial \Delta F}^{}\vv{E}\left(\vv{r}\right)\td \vv{r}\rightarrow \lim_{\Delta x\rightarrow 0}\Delta l\left(\vv{t}\times \vv{n}\right)\left(\vv{E}_a-\vv{E}_i\right)=0
.\end{align} 
Daraus folgt, dass die Tangentialkomponenten des $\vv{E}$--Feldes stetig durch die Grenzflächen gehen. Eine Oberflächenladungsdichte spielt für das $\vv{E}$--Feld keine Rolle.

\subsection{Elektrostatische Feldenergie}
Die Krafteinwirkung auf eine Punktladung ist $\vv{F}\left(\vv{r}\right)=q\vv{E}\left(\vv{r}\right)$. Verschiebt man diese Punktladung von $A$ nach $B$ muss eine Arbeit verrichtet werden
\begin{align} 
        W_{AB}&=-\int_{B}^{A}\vv{F}\left(\vv{r}\right)\td \vv{r}=q\left(\varphi \left(A\right)-\varphi \left(B\right)\right)
.\end{align} 
$W_{AB}>0$, wenn Arbeit am System verrichtet wird. Die Energie einer Ladungsverteilung wird als die Arbeit bezeichnet, um die Ladungen aus dem Unendlichen zusammenzuziehen.

\subsubsection{Energie von $N$ Punktladungen}
Punktladungen $q_j$ befinden sich an den Punkten $\vv{r}_j$. Da mit jeder weiteren \glqq zusammengezogenen\grqq{} Ladung das Feld verändert wird, wird nur die $i-1$te Ladung betrachtet. Die Arbeit für diese Ladung ist die Summe aus allen Ladung bis zur $i-1$ten Ladung
\begin{align} 
        \varphi \left(\vv{r}_i\right)=\dfrac{1}{4\pi \varepsilon _0}\sum_{j=1}^{i-1}\dfrac{q_j}{|\vv{r}_i-\vv{r}_j|}\qquad W_i=q_i\varphi \left(\vv{r}_i\right)
.\end{align} 
Die Summation über alle Ladungen $i=1,\hdots ,N$ ist
\begin{align} 
        W&=\dfrac{1}{4\pi \varepsilon _0}\sum_{i=2}^{N}\sum_{j=1}^{i-1}\dfrac{q_iq_j}{|\vv{r}_i-\vv{r}_j|}\\
         &=\dfrac{1}{2}\dfrac{1}{4\pi \varepsilon _0}\sum_{i,j=1;i \neq j}^{N}\dfrac{q_iq_j}{|\vv{r}_i-\vv{r}_j|}
.\end{align} 
Dieser Ausdruck kann für eine kontinuierliche Ladungsverteilung $\rho $ verallgemeinert werden
\begin{align} 
        W_c&=\dfrac{1}{2}\dfrac{1}{4\pi \varepsilon _0}\int_{}^{}\td ^3r\int_{}^{}\td ^3r'\dfrac{\rho \left(\vv{r}\right)\rho \left(\vv{r}'\right)}{|\vv{r}-\vv{r}'|}\\
           &=\dfrac{1}{2}\int_{}^{}\td ^3r\rho \left(\vv{r}\right)\varphi \left(\vv{r}\right)\qquad \left|-\Delta \varphi =\dfrac{\rho }{\varepsilon _0}\right.\\
           &=-\dfrac{\varepsilon _0}{2}\int_{}^{}\td ^3r\left(\Delta \varphi \left(\vv{r}\right)\right)\varphi \left(\vv{r}\right)\\
           &=-\dfrac{\varepsilon _0}{2}\int_{}^{}\td ^3r\vv{\nabla }\cdot \left(\vv{\nabla }\varphi \left(\vv{r}\right)\varphi \left(\vv{r}\right)\right)+\dfrac{\varepsilon _0}{2}\int_{}^{}\td ^3r\overbrace{\left(-\vv{\nabla }\varphi \left(\vv{r}\right)\right)}^{\vv{E}}\left(-\vv{\nabla }\varphi \left(\vv{r}\right)\right)\\
           &=-\dfrac{\varepsilon _0}{2}\oint_{}^{}\td \vv{f}\left(\vv{\nabla }\varphi \left(\vv{r}\right)\right)\varphi \left(\vv{r}\right)+\dfrac{\varepsilon _0}{2}\int_{}^{}\td ^3r|\vv{E}\left(\vv{r}\right)|^2
.\end{align} 
Für Ladungen im Unendlichen wird das erste Integral gleich null, da $\varphi \left(\vv{r}\right)\propto\tfrac{1}{r}$ und $\varphi \vv{\nabla }\varphi \propto \tfrac{1}{r^3}$. Das Flächenintegral im Unendlichen ist damit
\begin{align} 
        \lim_{r\rightarrow \infty}\oint_{\mathcal{K}_r}^{}\td \vv{f}\left(\vv{\nabla }\varphi \left(\vv{r}\right)\right)\varphi \left(\vv{r}\right)\propto \lim_{r\rightarrow \infty}\int_{}^{}\td \Omega \dfrac{1}{r^3}r^2=0
.\end{align} 
Für die Energiedichte bleibt dann
\begin{align} 
        w_c&=\dfrac{\varepsilon _0}{2}|\vv{E}\left(\vv{r}\right)|^2\geq 0\,\text{positiv--semidefinit}\,
.\end{align} 
Das Problem bei der kontinuierlichen Ladungsdichte ist, dass sie die Selbstenergie der Ladung enthält.

\subsubsection{Beispiel: 2 Punktladungen}
Das elektrische Feld für zwei Punktladungen ist
\begin{align} 
        \vv{E}\left(\vv{r}\right)&=\dfrac{1}{4\pi \varepsilon _0}\left(\dfrac{q_1\left(\vv{r}-\vv{r}_1\right)}{|\vv{r}-\vv{r}_1|^3}+\dfrac{q_2\left(\vv{r}-\vv{r}_2\right)}{|\vv{r}-\vv{r}_2|^3}\right)
.\end{align} 
Die Energiedichte berechnet sich dann zu
\begin{align} 
        w_{2\,\text{pt.}\,}&=\dfrac{1}{32\pi ^2\varepsilon _0}\left(\dfrac{q_1^2}{|\vv{r}-\vv{r}_1|^4}+\dfrac{q_2^2}{|\vv{r}-\vv{r}_2|^4}\right)+\dfrac{1}{16\pi ^2\varepsilon _0}\left(\dfrac{q_1q_2\left(\vv{r}-\vv{r}_1\right)\left(\vv{r}-\vv{r}_2\right)}{|\vv{r}-\vv{r}_1|^3|\vv{r}-\vv{r}_2|^3}\right)
.\end{align} 
Sie jetzt eine unendlich große unendlich verdünnte Ladungswolke. Diese Ladungswolke kann man als homogen geladene Kugel mit Radius $R$ und Gesamtladung $q$ darstellen. Für die Arbeit, um alle Ladungen auf den Punkt in der Mitte zusammenzuziehen, gilt
\begin{align} 
        W_{\,\text{Kugel}\,}&=\dfrac{q^2}{4\pi \varepsilon _0}\dfrac{3}{5}\dfrac{1}{R}
.\end{align} 
Um eine Punktladung darzustellen wird $R\rightarrow 0$. Dabei wird aber die Arbeit unendlich. Physikalisch ist dies aber kein Problem, da die Selbstenergie selbst keinen Effekt hat. Nur die Energiedifferenz bzw.\ der wechselwirkungsanteil ist von Relevanz.

\subsection{Multipolentwicklung}
Die Annahme ist eine räumlich begrenzte (Radius $R$) Ladungsverteilung $\rho $ und keine Randbedingungen im Endlichen. Von Interesse ist hier der Effekt im Unendlichen, also die \textbf{Fernzone}. Das allgemeine Potential ist $\varphi \left(\vv{r}\right)=\tfrac{1}{4\pi \varepsilon _0}\int_{}^{}\td ^3r'\tfrac{\rho \left(\vv{r}'\right)}{|\vv{r}-\vv{r}'|}$. Man sucht das Potential für $r\gg R$. Dafür wird eine \textsc{Taylor}--Entwicklung um $\tfrac{r'}{r}\ll 1$ angesetzt.
\begin{align} 
        \dfrac{1}{|\vv{r}-\vv{r}'|}&=\text{e}^{-\vv{r}'\cdot \vv{\nabla }}\dfrac{1}{r}=\dfrac{1}{r}+\dfrac{\vv{r}'\vv{r}}{r^3}+\dfrac{3\left(\vv{r}\vv{r}'\right)^2-r^2r'^2}{r^5}+\hdots 
.\end{align} 
Damit ist das Potential dann
\begin{align} 
        \varphi \left(\vv{r}\right)&=\dfrac{1}{4\pi \varepsilon _0}\dfrac{1}{r}\int_{}^{}\td ^3r'\rho \left(\vv{r}'\right)+\dfrac{1}{4\pi \varepsilon _0}\dfrac{\vv{r}}{r^3}\int_{}^{}\td ^3r'\vv{r}'\rho \left(\vv{r}'\right)+\dfrac{1}{4\pi \varepsilon _0}\int_{}^{}\td ^3r'\dfrac{3\left(\vv{r}\vv{r}'\right)^2-r^2r'^2}{r^5}\rho \left(\vv{r}'\right)+\hdots 
.\end{align}
Wenn die Terme ab inklusive $\tfrac{1}{r^3}$ wegfallen (da sie so klein sind), vereinfacht sich das Potential zu $\varphi \left(\vv{r}\right)=\tfrac{Q}{4\pi \varepsilon _0}\tfrac{1}{r}$. Die Momente der Ladungsverteilungen sind dann
\begin{align} 
        \,\text{Monopol}\,&:\vv{p}_M=\int_{}^{}\td ^3r'\rho \left(\vv{r}'\right)\propto \dfrac{1}{r}\\
        \,\text{Dipol}\,&:\vv{p}_D=\int_{}^{}\td ^3r'\vv{r}'\rho \left(\vv{r}'\right)\propto \dfrac{1}{r^2}\\
        \,\text{Quadrupol}\,&:\vv{p}_Q=\int_{}^{}\td ^3r'\left(3r_i'r_j'-\delta _{ij}r'^2\right)\rho \left(\vv{r}'\right)\propto \dfrac{1}{r^3}
.\end{align} 
Wenn $Q\neq 0$, dann ist der Monopolterm dominant in der Fernzone. Wenn $Q=0$, dann ist der Dipolmoment dominant. Für zwei entgegengerichtete gleichgroße Ladungen ist das Dipolmoment $\vv{p}=q\vv{d}$. Der Dipol in einem Punkt ist $\vv{p}\left(\vv{r}\right)=\lim_{d\rightarrow 0,q\rightarrow \infty}q\vv{d}$, sodass $\vv{p}$ endlich wird.
Das Potential des Dipols lässt sich dann mit dem Dipolmoment schreiben: $\varphi _D\left(\vv{r}\right)=\tfrac{1}{4\pi \varepsilon _0}\tfrac{\vv{p}_D\vv{r}}{r^3}$.\\\indent
Wenn $Q=0$ und $\vv{p}=0$, dann dominiert der Quadropolmoment. Das Potential ist $\varphi _Q\left(\vv{r}\right)=\tfrac{1}{4\pi \varepsilon _0}\sum_{i,j}^{}Q_{ij}\tfrac{r_ir_j}{r^5}$. Ein Quadrupol kann durch zwei antiparallele Dipole realisiert werden.

\subsection{Randwertproblem in der $E$--Statik}
Falls es keine Randbedingungen gibt, ist die Lösung der \textsc{Poisson}--Gleichung das \textsc{Poisson}--Integral
\begin{align} 
        \dfrac{1}{4\pi \varepsilon _0}\int_{}^{}\td ^3r'\dfrac{\rho \left(\vv{r}-\vv{r}'\right)}{|\vv{r}-\vv{r'}|}=\varphi \left(\vv{r}\right)
.\end{align} 
Oft ist aber $\rho \left(\vv{r}\right)$ in einem Raumgebiet $V$ und $\varphi $, oder $\diffp[]{\varphi }{n}=\left(\vv{n}\cdot \vv{\nabla }\right)\varphi =-E^{\left(n\right)}$ auf $S\left(V\right)$ gegeben.

\newpage
\newpage
\section{\textsc{Green}'sche Funktion}
\textit{Vorlesung 6 (26.10.) fehlt}
\newpage

\newpage
\section{Kugelflächenfunktion}
Der \textsc{Laplace}--Operator $\triangle$ kann als eine Funktion verstanden werden, die
\begin{align} 
        \triangle :f\mapsto \triangle f\left(\vv{r}\right)=\sum_{i=1}^{3}\partial _{r_i}^2 f\left(\vv{r}\right)
.\end{align} 
Es ist also möglich, dass $\triangle$ Eigenwerte besitzt, für die $\triangle f\left(\vv{r}\right)=\lambda f\left(\vv{r}\right)$.\par
Man sucht nun nach Eigenfunktionen auf einer Kugeloberlfäche. Es gilt bereits
\begin{align} 
        \triangle f&=\dfrac{1}{r^2}\partial_r \left(r^2\partial_r f\right)+\dfrac{1}{r^2}\triangle _{\theta ,\varphi }f
,\end{align} 
mit dem Winkelanteil 
\begin{align} 
        \triangle _{\theta ,\varphi }f&=\dfrac{1}{\sin \theta }\partial _\theta \left(\sin \theta \partial_\theta  f\right)+\dfrac{1}{\sin ^2\theta }\partial_\theta ^2 f
.\end{align} 
Die Eigenfunktionen von $\triangle _{\theta ,\varphi }$ sind die \textbf{Kugelflächenfunktionen}
\begin{align} 
        y_{lm}\left(\theta ,\varphi \right)&:=\,\sqrt[]{\dfrac{2l+1}{4\pi }\dfrac{\left(l-m\right)!}{\left(l+m\right)!}}P_l^m \left(\cos \theta \right)\text{e}^{\text{i}m\varphi }
,\end{align} 
mit $l=0,1,2,\hdots $ und $m=-l,-l+1,-l+2,\hdots ,l-2,l-1,l$. Beispiele sind
\begin{align} 
        y_{00}\left(\theta ,\varphi \right)&=\dfrac{1}{\,\sqrt[]{4\pi }}&y_{11}\left(\theta ,\varphi \right)&=-\,\sqrt[]{\dfrac{3}{8\pi }}\sin \theta \text{e}^{\text{i}\varphi }&y_{10}\left(\theta ,\varphi \right)&=\,\sqrt[]{\dfrac{3}{4\pi }}\cos \theta 
.\end{align}
Die Funktion $P_l^m$ sind die zugeordneten \textsc{Legendre}--Polynome. Sie sind Lösungen der verallgemeinerten \textsc{Legendre}--Gleichung
\begin{align} 
        \diff*[]{\left(\left(1-z^2\right)\diff[]{P_l^m}{z}\right)}{z}+\left(l\left(l+1\right)-\dfrac{m^2}{1-z^2}\right)P_l^m\left(z\right)&=0&z&=\cos \theta 
.\end{align} 
Wendet man den \textsc{Laplace}--Operator auf die Kugelflächenfunktion an erhält man die Eigenfunktionen auf der Kugeloberfläche (ohne Radialteil)
\begin{align} 
        \triangle_{\theta ,\varphi }y_{l m}\left(\theta ,\varphi \right)=-l\left(l+1\right)y_{l m}\left(\theta ,\varphi \right)
.\end{align} 
Diese Funktion ist orthogonal. Es gilt
\begin{align} 
        \int_{0}^{2\pi }\td \varphi \int_{0}^{\pi }\sin \theta \td \theta y_{l'm'}^*\left(\theta ,\varphi \right)y_{l m}\left(\theta ,\varphi \right)&=\delta _{ll'}\delta _{mm'}
.\end{align} 
Zudem ist sie vollständig, also
\begin{align} 
        \sum_{l=0}^{\infty}\sum_{m=-l}^{l}y_{l m}^*\left(\theta ',\varphi '\right)y_{l m}\left(\theta ,\varphi \right)&=\underbrace{\delta \left(\cos \theta '-\cos \theta \right)}_{\delta \left(z'-z\right)}\delta \left(\varphi -\varphi '\right)
.\end{align} 

\subsection{Entwicklung von Funktionen}
Eine Funktion kann entwickelt werden, indem
\begin{align} 
        f\left(r,\theta ,\varphi \right)&=\sum_{l=0}^{\infty}\sum_{m=-l}^{l}R_{l m}\left(r\right)y_{l m}\left(\theta ,\varphi \right)
,\end{align} 
mit $R_{l m}\left(r\right)$ dem Radialanteil,
\begin{align} 
        R_{l m}\left(r\right)&=\int_{0}^{2\pi }\td \varphi \int_{0}^{\pi }\sin \theta \td \theta f\left(r,\theta ,\varphi \right)y_{l m}^*\left(\theta ,\varphi \right)
.\end{align} 

\subsection{Lösung der \textsc{Laplace}--Gleichung}
Der Ansatz für die Lösung der Gleichung $\triangle \varphi =0$ ist
\begin{align} 
        \varphi \left(\vv{r}\right)&=\varphi \left(r,\theta ,\varphi \right)=\sum_{l,m}^{}R_{l m}\left(r\right)y_{l m}\left(\theta ,\varphi \right)
.\end{align} 
Damit folgt
\begin{align} 
        0=\triangle \varphi &=\sum_{l,m}^{}\left[\dfrac{1}{r^2}\diff*[]{r^2\diff[]{R_{l m}}{r}}{r}+\dfrac{R_{l m}}{r^2}\triangle _{\theta ,\varphi }\right]y_{l m}\left(\theta ,\varphi \right)\\
                            &=\sum_{l,m}^{}\left[\dfrac{1}{r^2}\diff*[]{\left(r^2\diff[]{R_{l m}\left(r\right)}{r}-\dfrac{R_{l m}\left(r\right)l\left(l+1\right)}{r^2}\right)}{r}\right]y_{l m}
.\end{align} 
Der Term 
\begin{align} 
        \dfrac{1}{r^2}\diff*[]{\left(r^2\diff[]{R}{r}\right)}{r}-\dfrac{l\left(l+1\right)}{r^2}R=0
\end{align} 
ist der \textbf{Radialanteil}. Mit dem Ansatz $R\left(r\right)=\tfrac{1}{r}u\left(r\right)$ 
\begin{align} 
        \left(\dfrac{\td ^2}{\td r^2}-\dfrac{l\left(l+1\right)}{r^2}\right)u\left(r\right)&=0\\
        Ar^{l+1}+Br^{-l}&=u\left(r\right)\\
        Ar^l+Br^{-\left(l+1\right)}&=R\left(r\right)
.\end{align} 
Die allgemeine Lösung der \textsc{Laplace}--Gleichung ist dann
\begin{align} 
        \varphi \left(\vv{r}\right)&=\sum_{l,m}^{}\left(A_{l m}r^l+B_{l m}r^{-\left(l+1\right)}\right)y_{l m}\left(\theta ,\varphi \right)
,\end{align} 
mit $A_{l m}$ und $B_{l m}$ aus den Randbedingungen.

\subsection{Beispiel: Ladung einer Hohlkugel}
Sie eine Kugel mit Radius $R$ und Ladungsverteilung $\sigma \left(\theta ,\varphi \right)$ auf der Hülle. Im Inneren und im Äußeren ist die Ladung $\rho _i$ und $\rho _a$ gleich null. Das Potential lässt sich dann mit der allgemeinen Lösung für die Kugelflächenfunktion lösen
\begin{align} 
        \phi _i\left(\vv{r}\right)&=\sum_{l,m}^{}\left(A^i_{l m}r^l+B^i_{l m}r^{-\left(l+1\right)}\right)y_{l m}\left(\theta ,\varphi \right)\\
        \phi _a\left(\vv{r}\right)&=\sum_{l,m}^{}\left(A^a_{l m}r^l+B^a_{l m}r^{-\left(l+1\right)}\right)y_{l m}\left(\theta ,\varphi \right)
.\end{align} 
Die Randbedingungen sind,
\begin{enumerate}[label=\roman*)]
        \item $\phi _a\left(\vv{r}\right)=0$ für $r\rightarrow \infty$. $\Rightarrow A_{l m}^a=0\,\forall l,m$.
        \item Bei $r=0$ soll $\phi _i$ eine reguläre Funktion sein (sie soll nicht divergieren).$\Rightarrow B_{l m}^i=0\,\forall l,m$.
\end{enumerate}
Damit ist die Lösung vorerst
\begin{align} 
        \phi \left(\vv{r}\right)&=\begin{cases}
                \sum_{l m}^{}A_{l m}^ir^ly_{l m}\left(\theta ,\varphi \right)&,r\leq R\\
                \sum_{l,m}^{}B_{l m}^ar^{-\left(l+1\right)}y_{l m}\left(\theta ,\varphi \right)&,r>R
        \end{cases}
.\end{align} 
\begin{enumerate}[label=\roman*)]
        \item[iii)] $\phi _i\left(R,\theta ,\varphi \right)=\phi _a\left(R,\theta ,\varphi \right)\,\forall \theta ,\varphi $. Es muss dann $\,\forall l,m$ gelten, dass $A^i_{l m}R^l=B_{l m}^aR^{-\left(l+1\right)}$. Damit ist $B_{l m}^a=A^i_{l m}R^{-(2l+1)}$.
        \item[iv)] $\sigma \left(\theta ,\varphi \right)$ existerit auf $S\left(K_R\right)$. Damit ist das $\vv{E}$--Feld nicht stetig, also $\partial_r \phi _i-\partial_r \varphi _a=\tfrac{\sigma }{\varepsilon _0}$. Entwickelt man $\sigma \left(\theta ,\varphi \right)$ in $y_{l m}$ folgt, $\sigma \left(\theta ,\varphi \right)=\sum_{l,m}^{}\sigma _{l m}y_{l m}\left(\theta ,\varphi \right)$. Daraus folgt $A_{l m}^i=\dfrac{\sigma _{l m}}{2\left(l+1\right)\varepsilon _0}\dfrac{1}{R^{l-1}}$.
\end{enumerate}

\newpage
\section{Dielektrika}
\subsection{Makroskopische Feldgrößen}
Im Wesentlichen werden für makroskopische Feldgrößen mikroskopische Felder $f\left(\vv{r}\right)$ geglättet, also $\left\langle f\left(\vv{r}\right)\right\rangle =\dfrac{1}{v\left(\vv{r}\right)}\int_{v\left(\vv{r}\right)}^{}f\left(\vv{r}'\right)\td ^3r'$. $\left\langle f\left(\vv{r}\right)\right\rangle $ ist ein kontinuierliches Feld. Man macht die Annahme, dass $\left\langle \,\text{grad}\,f\left(\vv{r}\right)\right\rangle =\,\text{grad}\,\left\langle f\left(\vv{r}\right)\right\rangle $. Analog betrachtet man das $\vv{E}$--Feld als $\vv{E}\left(\vv{r}\right)=\left\langle \vv{E}_{\,\text{mikroskop.}\,}\left(\vv{r}\right)\right\rangle $. Damit sind die \textsc{Maxwell}--Gleichungen
\begin{align} 
        \,\text{div}\,\vv{E}&=\dfrac{\left\langle \rho _m\right\rangle }{\varepsilon _0}&\,\text{rot}\,\vv{E}&=0
.\end{align} 
Weiterhin ist $\left\langle \vv{E}_m\right\rangle =-\left\langle \,\text{grad}\,\varphi _m\right\rangle =-\,\text{grad}\,\left\langle \varphi _m\right\rangle $.\par
Sei ein $j$--tes Teilchen an Punkt $\vv{R}_j$, insgesamt neutral geladen mit \textbf{Überschussladungen} am Rand. 
Man betrachtet sein Potential am Punkt $\vv{r}$.
Die Gesamtladung ist $q_j=\sum_{n}^{\left(j\right)}q ^{(j)}_n $, mit der Ladungsdichte $\rho _j\left(\vv{r}\right)=\sum_{n}^{(j)}q ^{(j)}_n\delta \left(\vv{r}-\vv{r}_n\right)$.
$j$ ist hierbei das Teilchen mit $n$ Ladungsträgern.\par
Wenn ein $\vv{E}$--Feld anliegt, werden Ladungen aus ihrer Gleichgewichtslage ausgelenkt, wodurch Multipole induziert werden.
Das Dipolmoment ist also $\vv{p}_j=\int_{}^{}\td ^3r\rho _j\left(\vv{r}\right)\left(\vv{r}-\vv{R}_j\right)$.
Man nimmt an, dass $|\vv{r}-\vv{R}_j|\gg$ als die Abstände im Raum des Teilchens $j$ sind.
Für die Multipolentwicklung befindet man sich also in der Fernzone.
Das Potential des $j$--ten Teilchens mit Dipolmoment ist dann
\begin{align} 
        \varphi _j\left(\vv{r}\right)&=\dfrac{1}{4\pi \varepsilon _0}\left[\dfrac{q_j}{|\vv{r}-\vv{R}_j|}+\dfrac{\vv{p}_j\left(\vv{r}-\vv{R}_j\right)}{|\vv{r}-\vv{R}_j|^3}\right]
.\end{align} 
Für $N$ Teilchen kann eine \textbf{effektive Ladungsdichte} bzw.\ \textbf{effektive Dipoldichte} verwendet werden,
\begin{align} 
        \rho _e\left(\vv{r}\right)&=\sum_{j=1}^{N}q_j\delta \left(\vv{r}-\vv{R}_j\right)&\vv{\Pi }_e\left(\vv{r}\right)&=\sum_{j=1}^{N}\vv{p}_j\delta \left(\vv{r}-\vv{R}_j\right)
.\end{align} 
Damit lässt sich das \textbf{effektive Potential} schreiben,
\begin{align} 
        \varphi _e\left(\vv{r}\right)&=\dfrac{1}{4\pi \varepsilon _0}\int_{}^{}\td ^3r'\left[\dfrac{\rho _e\left(\vv{r}'\right)}{|\vv{r}-\vv{r}'|}+\vv{\Pi }_e\left(\vv{r}'\right)\dfrac{\vv{r}-\vv{r}'}{|\vv{r}-\vv{r}'|^3}\right]
.\end{align} 
Die Mittelung für makroskopische Ladungsdichte ist $\rho \left(\vv{r}\right)=\left\langle \rho _e\left(\vv{r}\right)\right\rangle $.
Sie stellt die \textbf{Überschussladungen} dar, da alle anderen Ladung im Mittel neutral sind.\par
Aus der Dipoldichte lässt sich die \textbf{makroskopische Polarisation} herleiten
\begin{align} 
        \vv{P}\left(\vv{r}\right)&=\left\langle \vv{\Pi }_e\left(\vv{r}\right)\right\rangle =\dfrac{1}{v\left(\vv{r}\right)}\sum\limits_{j \in v\left(\vv{r}\right)}^{}\vv{p}_j
.\end{align} 
Damit lässt sich das gemittelte Potential schreiben als
\begin{align} 
        \varphi \left(\vv{r}\right)=\left\langle \varphi _m\left(\vv{r}\right)\right\rangle &=\dfrac{1}{4\pi \varepsilon _0}\int_{}^{}\td ^3r'\left[\dfrac{\rho \left(\vv{r}\right)}{|\vv{r}-\vv{r}'|}+\vv{P}\left(\vv{r}'\right)\cdot \,\text{grad}\,_{\vv{r}}\dfrac{1}{|\vv{r}-\vv{r}'|}\right]
,\end{align} 
bzw.\ das elektrische Feld
\begin{align} 
        \,\text{div}\,\vv{E}&=\dfrac{1}{\varepsilon _0}\left(\rho \left(\vv{r}\right)-\,\text{div}\,\vv{P}\left(\vv{r}\right)\right)=\dfrac{1}{\varepsilon _0}\left(\rho +\rho _P\right)
,\end{align} 
mit $\rho _P=\,\text{div}\,\vv{P}\left(\vv{r}\right)$, der Polarisationsladungsdichte.\par
Man definiert die \textbf{dielektrische Verschiebung} und erhält die \textsc{Maxwell}--Gleichungen in einem Medium
\begin{align} 
        \vv{D}&=\varepsilon _0\vv{E}+\vv{P}&\,\text{div}\,\vv{D}&=\rho &\,\text{rot}\,\vv{E}&=0
.\end{align} 
Daraus folgt, dass die Quellen von $\vv{D}$ Überschussladungen sind.
Das elektrische Feld hängt von der Materie ab.\par
Das Polarisationfeld induziert eine Oberflächenladung $\sigma _P=\vv{P}\vv{n}$.\par
Es gibt verschiedene Arten von Dielektrika, z.B.\ eigentliche Dielektrika, Paraelektrika oder Ferroelektrika.
\\\hfill\\\textbf{Typen von Polarisation}\\ 
Man unterscheidet zwei Typen von Polarisation, $\vv{P}=\vv{P}\left(\vv{E}\right)$ und $\vv{P}\left(0\right)=0$.
\begin{align} 
        \vv{P}\left(\vv{E}\right)_i&=\sum_{k}^{}\gamma _{ik}E_k-\sum_{k,l}^{}\beta _{ikl}E_kE_l+\hdots 
.\end{align} 
Man vereinfacht diesen Ausdruck zu $P_i=\gamma E_i$.
Es handelt sich um ein isotropes Meidum, mit $\vv{P}=\chi_e \varepsilon _0\vv{E}$.
Die dielektrische Verschiebung ist dann
\begin{align} 
        \vv{D}=\varepsilon _0\vv{E}+\vv{P}=\varepsilon _0\left(\chi_e+1\right)\vv{E}&=\varepsilon _r\varepsilon _0\vv{E}
.\end{align} 

\subsection{Randwertprobleme}
Von Interesse sind die Bedingungen an Grenzflächen verschiedener Dielektrika, mit $\varepsilon ^1_r$ und $\varepsilon ^2_r$.
Mit Hilfe des \textsc{Gauss}--schen Satzes folgt dann für die Normalkomponente
\begin{align} 
        \vv{n}\cdot \left(\vv{D}_2-\vv{D}_1\right)=\sigma 
,\end{align} 
bzw.\ für die Tangentialkomponente
\begin{align} 
        \left(\vv{E}_2-\vv{E}_1\right)\times \vv{n}=0
.\end{align} 
Daraus folgt, dass
\begin{align} 
        D_1^n&=D_2^n&E_1^n&=\dfrac{\varepsilon _r^2}{\varepsilon _r^1}E_2^n&E_1^t&=E_2^t&D_1^t&=\dfrac{\varepsilon _r^1}{\varepsilon _r^2}D_2^t
.\end{align} 

\subsection{Beispiel: Grenzfläche}
Seien zwei Dielektrika mit $\varepsilon _1$ und $\varepsilon _2$. Auf der $z$--Achse sei eine Punktladung $q$ in Medium $\varepsilon _1$ im Abstand $d$ zu Medium $\varepsilon _2$. Auf der Grenzfläche ist $\sigma =0$.
Es gilt
\begin{align} 
        \varepsilon _1\,\text{div}\,\vv{E}&=\rho \left(z>0\right)=q\delta \left(\vv{r}-d\vv{e}_z\right)
.\end{align} 
Für das andere Medium gilt
\begin{align} 
        \varepsilon _2\,\text{div}\,\vv{E}&=0\qquad z<0
,\end{align} 
mit $\varepsilon _i\,\text{div}\,\vv{E}=\,\text{div}\,\vv{D}$.
$\,\forall z$ gilt $\,\text{rot}\,\vv{E}=0$.\par
Die Randbedingungen sind
\begin{align} 
        \lim_{z\rightarrow 0}D_z^1\left(-z\right)&=\lim_{z\rightarrow 0}D_z^1\left(z\right)
.\end{align} 
Diese Bedingung ist gleichbedeutend mit
\begin{align} 
        \lim_{z\rightarrow 0}\varepsilon _2E_z^2\left(-z\right)&=\lim_{z\rightarrow 0}\varepsilon _1E_z^1\left(z\right)
.\end{align} 
Zudem ist
\begin{align} 
        \lim_{z\rightarrow 0}E_{x,y}^2\left(-z\right)&=\lim_{z\rightarrow 0}E_{x,y}^1\left(z\right)
.\end{align} 
Der Ansatz ist, dass die Ladung $q$ eine Spiegelladung bzw.\ Bildladung $q'$ bei $-d$ hat. Weiter hat die Ladung $q'$ eine Bildladung $q''$ im Medium 1, welche die Randbedingungen an der Grenzfläche simulieren.\par
Die Potentiale sind
\begin{align} 
        z>0&:\varphi ^1\left(\vv{r}\right)=\dfrac{1}{4\pi \varepsilon _0\varepsilon _r^1}\left[\dfrac{q}{|\vv{r}-d\vv{e}_z|}+\dfrac{q'}{|\vv{r}+d\vv{e}_z|}\right]\\
        z<0&:\varphi ^2\left(\vv{r}\right)=\dfrac{1}{4\pi \varepsilon _0\varepsilon _r^2}\left[\dfrac{q''}{|\vv{r}-d\vv{e}_z|}\right]
.\end{align} 


%{{{ Notizen
\newpage
\section{Delta--Distribution}
\begin{align} 
        \delta \left(\vv{r}-\vv{r}_0\right)=0\,\forall \vv{r}\neq \vv{r}_0
.\end{align} 
\begin{align} 
        \int_{V}^{}\delta \left(\vv{r}-\vv{r}_0\right)\td ^3r=\begin{cases}
                1&,\vv{r}_0 \in V\\
                0&,\vv{r}_0 \notin V
        \end{cases}
\end{align} 
\begin{align} 
        \int_{V}^{}f\left(\vv{r}\right)\delta \left(\vv{r}-\vv{r}_0\right)\td ^3r=\begin{cases}
                f\left(\vv{r}_0\right)&,\vv{r}_0 \in V\\
                0&,\vv{r}_0 \notin V
        \end{cases}
\end{align} 


\newpage
\section{Flächenintegrale}
Sei die Fläche $\mathcal{F}:=\left\{\vv{r}\left(u,v\right)|\left(u,v\right) \in D\right\}$. Der Normalenvektor zur Fläche ist $\td \vv{f}=\td \vv{a}\times \td \vv{b}$, mit
\begin{align} 
        \td \vv{a}&=\vv{r}\left(u,v+\td v\right)-\vv{r}\left(u,v\right)\approx \partial_v \vv{r}\left(u,v\right)\td v\\
        \td \vv{b}&=\vv{r}\left(u+\td u,v\right)-\vv{r}\left(u,v\right)\approx \partial_u \vv{r}\left(u,v\right)\td u\\
        \td \vv{f}&=\partial_v \vv{r}\left(u,v\right)\times \partial_u\vv{r}\left(u,v\right)\td u\td v
.\end{align} 
Die Vektoren $\partial_v \vv{r}$ und $\partial_u \vv{r}$ spannen die Tangentialebene in $\vv{r}\left(u,v\right)$ auf.\\\indent
Der Fluss eines Vektorfeldes $\vv{a}\left(\vv{r}\right)$ durch eine Fläche $S$, bzw.\ eine geschlossene Fläche $S\left(V\right)$ mit $\td \vv{f}$ als Flächennormale ist gegeben durch
\begin{align} 
        \varphi _S\left(\vv{a}\right)&=\int_{S}^{}\vv{a}\left(\vv{r}\right)\td \vv{f}=\int_{S}^{}\vv{a}\left(\vv{r}\right)\cdot \vv{n}\left(\vv{r}\right)\td f\\
        \varphi _{S\left(v\right)}\left(\vv{a}\right)&=\oint_{S\left(V\right)}^{}\vv{a}\left(\vv{r}\right)\td \vv{f}=\oint_{S\left(V\right)}^{}\vv{a}\left(\vv{r}\right)\cdot \vv{n}\left(\vv{r}\right)\td f
.\end{align} 


\subsection{Beispiel: Kugeloberfläche}
Sei eine Kugel mit dem Radius $R$, dann sind die Kugelkoordinaten
\begin{align} 
        \begin{pmatrix}
                x\\y\\z
        \end{pmatrix}\mapsto \begin{pmatrix}
                R\cos \varphi \sin \theta \\R\sin \theta \sin \varphi \\R\cos \theta 
        \end{pmatrix}
.\end{align} 
\begin{align} 
        \diffp[]{\vv{r}}{\theta }&=R\vv{e}_\theta &\diffp[]{\vv{r}}{\varphi }&=R\sin \theta \vv{e}_\varphi 
.\end{align} 
Das Flächenelement ist dann
\begin{align} 
        \td \vv{f}=\diffp[]{\vv{r}}{\theta }\times \diffp[]{\vv{r}}{\varphi }=R^2\sin \theta \vv{e}_r
\end{align} 
%}}}

%}}}

\end{document}
