%:LLPStartPreview
%:VimtexCompile(SS)

%{{{ Formatierung

\documentclass[a4paper,12pt]{article}

\usepackage{physics_notetaking}

%%% dark red
%\definecolor{bg}{RGB}{60,47,47}
%\definecolor{fg}{RGB}{255,244,230}
%%% space grey
%\definecolor{bg}{RGB}{46,52,64}
%\definecolor{fg}{RGB}{216,222,233}
%%% purple
%\definecolor{bg}{RGB}{69,0,128}
%\definecolor{fg}{RGB}{237,237,222}
%\pagecolor{bg}
%\color{fg}

\newcommand{\td}{\,\text{d}}
\newcommand{\RN}[1]{\uppercase\expandafter{\romannumeral#1}}
\newcommand{\zz}{\mathrm{Z\kern-.3em\raise-0.5ex\hbox{Z} }}

\newcommand\inlineeqno{\stepcounter{equation}\ {(\theequation)}}
\newcommand\inlineeqnoa{(\theequation.\text{a})}
\newcommand\inlineeqnob{(\theequation.\text{b})}
\newcommand\inlineeqnoc{(\theequation.\text{c})}

\newcommand\inlineeqnowo{\stepcounter{equation}\ {(\theequation)}}
\newcommand\inlineeqnowoa{\theequation.\text{a}}
\newcommand\inlineeqnowob{\theequation.\text{b}}
\newcommand\inlineeqnowoc{\theequation.\text{c}}

\renewcommand{\refname}{Source}
\renewcommand{\sfdefault}{phv}
%\renewcommand*\contentsname{Contents}

\pagestyle{fancy}

\sloppy

\numberwithin{equation}{section}

%}}}

\begin{document}

%{{{ Titelseite

\title{physik321 $|$ Notizen}
\author{Jonas Wortmann}
\maketitle
\pagenumbering{gobble}

%}}}

\newpage

%{{{ Inhaltsverzeichnis

\fancyhead[L]{\thepage}
\fancyfoot[C]{}
\pagenumbering{arabic}

\tableofcontents

%}}}

\newpage

%{{{

\fancyhead[R]{\leftmark\\\rightmark}

\section{Überblick Elektrodynamik}
Das Ziel ist die Untersuchung der Ursache und Wirkung von elektrischen ($\vv{E}$) und magnetischen ($\vv{B}$) Feldern auf elektrische Ladungen ($q$).\\\indent
Aus der experimentellen Beobachtung ist bekannt, dass auf elektrisch geladene Körper eine elektromagnetische Kraft 
\begin{align} 
        \vv{F}=q\left(\vv{E}+\vv{v}\times \vv{B}\right)
,\end{align} 
mit $\vv{v}$ der Geschwindigkeit des geladenen Teilchens, wirkt. Diese Kraft führt zu einer Bewegungsänderung
\begin{align} 
        \vv{F}=\diff*[]{\left[\dfrac{m\vv{v}}{\,\sqrt[]{1-\tfrac{\vv{v}^2}{c^2}} }\right]}{t}
.\end{align} 
Der Zusammenhang zwischen elektrischen und magnetischen Feldern und (bewegten) Ladungen sind die \textbf{Maxwell--Gleichungen}
\begin{align} 
        \,\text{div}\,\vv{E}&=\dfrac{\rho }{\varepsilon _0}&\,\text{rot}\,\vv{E}&=-\diffp[]{\vv{B}}{t}\\
        \,\text{div}\,\vv{B}&=0&c^2\,\text{rot}\,\vv{B}&=\diffp[]{\vv{E}}{t}+\dfrac{\vv{j}}{\varepsilon _0}
.\end{align} 
Zusammen mit Randbedingungen an \textbf{Grenzflächen} bestimmen sie alle Effekte der Elektrodynamik.

\newpage
\section{Statische Felder}
Die Maxwell--Gleichungen für statische Felder sind
\begin{align} 
        \,\text{div}\,\vv{E}&=\dfrac{\rho }{\varepsilon _0}&\,\text{rot}\,\vv{E}&=0\\
        \,\text{div}\,\vv{B}&=0&c^2\,\text{rot}\,\vv{B}&=\dfrac{\vv{j}}{\varepsilon _0}
.\end{align} 
Man kann sehen, dass die Gleichungen für statische Felder entkoppeln und sie sich in \textbf{Elektrostatik} und \textbf{Magnetostatik} aufteilen.

\subsection{Eletrostatik}
Die Grundgrößen der klassischen Mechanik sind die Masse, Länge und Zeit.
Eine wichtige Grundgröße in der Elektrostatik ist die elektrische Ladung. 
Aus der experimentellen Beobachtung ist bekannt, dass Körper in einen elektrischen Zustand versetzt werden können (z.B.\ können geladene Körper andere geladene Körper anziehen).
Dieses Phänomen ist mechanisch nicht erklärbar.
Dieser Zustand ist auch auf andere Körper übertragbar, woraus folgt, dass es sich um eine substanzartige Größe handeln muss.
\\\indent Diese Größe ist die \textbf{elektrische Ladung} $q$.
Bei der Übertragung fließt ein elektrischer Strom $I$.
Die Ladung des Elektrons ist negativ, also $q<0$.
Zudem ist sie additiv, es existiert also die Gesamtladung $Q=\sum_{i}^{}q_i$.
In abgeschlossenen Systemen ist die Summe aus positiven und negativen Ladungen konstant.
Die Ladungen sind \textbf{gequantelt}, es existiert also eine nicht teilbare Elementarladung $e$, also gilt immer, dass $q=n\cdot e,n  \in \mathbb{Z}$
\begin{align*} 
        \,\text{Elektron}\,&:n=-1\\
        \,\text{Proton}\,&:n=+1\\
        \,\text{Neutron}\,&:n=0\\
        \,\text{Atomkern}\,&:n=Z
.\end{align*} 
In der Elektrodynamik wird dieses Prinzip allerdings verallgemeinert.
Man führt die Ladungsdichte $\rho \left(\vv{r}\right)$ ein, also $Q=\int_{V}^{}\rho \left(\vv{r}\right)\td ^3r$.
Für Punktladungen gilt dann $\rho \left(\vv{r}\right)=q\delta \left(\vv{r}-\vv{r}_0\right)$.\\\indent
Befinden sich zwei Ladungen $q_1$ und $q_2$ im Abstand von einem Meter im Vakuum, dann wirkt eine Kraft von
\begin{align} 
        F&=\dfrac{10^{12}}{4\pi \cdot 8,854}\,\unit{N}
.\end{align} 
Dann haben $q_1$ und $q_2$ eine Ladung von $|q_1|=|q_2|=\SI{1}{C}$.
Die Elementarladung ist $e=\SI{1,602e-19}{C}$.\\\indent
\\\hfill\\\textbf{Stromdichte}\\ 
Für bewegte Ladungen existiert die Stromdichte $\vv{j}$.
Sie gibt die Ladung pro Zeiteinheit durch eine Flächeneinheit senkrecht zur Stromrichtung an.
Betrachte als Beispiel eine homogene Ladungsverteilung von $N$ Teilchen mit Ladung $q$ und Geschwindigkeit $\vv{v}$, dann ist die Ladungsdichte $\vv{j}=\tfrac{N}{V}q\vv{v}$. 
Die Stromstärke $I$ ist dann $I=\int_{\mathcal{F}}^{}\vv{j}\left(\vv{r}\right)\td \vv{f}=\int_{\mathcal{F}}^{}\vv{j}\left(\vv{r}\right)\vv{n}\left(\vv{r}\right)\td f$.
Die Einheit ist $\SI{1}{A}$, was einem Ladungstransport von $\SI{1}{C}$ in einer Sekunde entspricht.
\\\hfill\\\textbf{Ladungserhaltung}\\ 
Die Ladungserhaltung kann mit Hilfe der \textbf{Kontinuitätsgleichung} beschrieben werden
\begin{align} 
        \diffp[]{\rho }{t}=\,\text{div}\,\vv{j}
.\end{align} 

\subsubsection{Coulomb'sche Gesetz}
Zwei Ladungen $q_1$ und $q_2$ befinden sich im Abstand $\vv{r}_1$ und $\vv{r}_2$ zum Ursprung.
Der Abstand zwischen diesen Ladungen ist $\vv{r}_{12}$.
Dieser Abstand soll viel größer sein also die Ausdehnung von $q_1$ und $q_2$.
Die Kraft zwischen diesen Ladungen ist
\begin{align} 
        \vv{F}_{12}=kq_1q_2\dfrac{\vv{r}_1-\vv{r}_2}{|\vv{r}_1-\vv{r}_2|^3}=-\vv{F}_{21}
.\end{align} 
Sie ist also direkt proportional zu $q_1$ und $q_2$, $|\vv{F}_{12}|\propto |\vv{r}_1-\vv{r}_2|^{-2}$, sie wirkt entlang von $F_{12}$.
Diese Kraft gilt nur für \textbf{ruhende} Ladungen.

\newpage
\section{Delta--Distribution}
\begin{align} 
        \delta \left(\vv{r}-\vv{r}_0\right)=0\,\forall \vv{r}\neq \vv{r}_0
.\end{align} 
\begin{align} 
        \int_{V}^{}\delta \left(\vv{r}-\vv{r}_0\right)\td ^3r=\begin{cases}
                0&,\vv{r}_0 \notin V\\1&,\vv{r}_0 \in V
        \end{cases}
\end{align} 


%}}}

\end{document}
