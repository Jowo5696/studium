\section{Schwingungen}
\subsection{Freie Schwingung ohne Dämpfung}
Drehmomentgleichung
\begin{align} 
        \Theta \ddot{\varphi }&=-D\varphi =-M
,\end{align} 
mit $\Theta $ dem Trägheitsmoment, $D$ der Richtkonstante und $M$ dem rücktreibenden Drehmoment.\\\\
Normalform
\begin{align} 
        \ddot{\varphi }+\omega _0^2\varphi =0\qquad \omega _0^2=\dfrac{D}{\Theta }
.\end{align} 
Der Lösungsansatz funktioniert mit $\varphi \left(t\right)=C\text{e}^{\lambda t}$.

\subsection{Freie Schwingung mit Dämpfung}
Bewegungsgleichung
\begin{align} 
        \Theta \ddot{\varphi }+r\dot{\varphi }+D\varphi &=0
,\end{align} 
mit dem Dämpfungsdrehmoment $r\dot{\varphi }$.\\\\
Normalform
\begin{align} 
        \ddot{\varphi }+2\beta \dot{\varphi }+\omega _0^2\varphi =0\qquad 2\beta =\dfrac{r}{\Theta }\qquad \omega _0^2=\dfrac{D}{\Theta }
.\end{align} 
Der Lösungsansatz $\varphi \left(t\right)=C\text{e}^{\lambda t}$ führt zu
\begin{align} 
        \lambda _{1,2}&=-\beta \pm\,\sqrt[]{\beta ^2-\omega _0^2}
.\end{align} 
Man unterschiedet drei Fälle
\begin{enumerate}[label=--]
        \item Kriechfall $\beta ^2>\omega _0^2$: Nach einmaliger Auslenkung schwingt das System nicht, sondern bewegt sich aperiodisch zur Ruhelage zurück.
        \item Grenzfall $\beta ^2=\omega _0^2$: Dieser Fall ist für die Messung interessant, da das System hier genau einmal schwingt und sich dann relativ schnell der Ruhelage annähert.
        \item Schwingfall $\beta a^2<\omega _0^2$: Durch den komplexen Exponenten schwingt die Amplitude periodisch.
\end{enumerate}
Das Dämpfungsverhältnis ist definiert durch
\begin{align} 
        K:=\dfrac{\varphi _n}{\varphi _{n+1}}=\text{e}^{\beta T}\qquad T=\dfrac{2\pi }{\hat{\omega }}\qquad \hat{\omega }=\,\sqrt[]{\omega _0^2-\beta ^2}
.\end{align} 
Die Güte ist
\begin{align} 
        Q:=\dfrac{\omega _0}{2\beta }=\dfrac{\pi }{\beta T}
.\end{align} 

\subsection{Erzwungene Schwingung mit Dämpfung}
Bewegungsgleichung
\begin{align} 
        \Theta \ddot{\varphi }+r\dot{\varphi }+D\varphi &=M_0\cos \left(\omega t\right)
,\end{align} 
mit $M_0$ dem Erregerdrehmoment und $\omega $ der Erregerfrequenz. 
Der Lösungsansatz funktioniert durch Zerlegung in homogene und partielle Lösungen.\\\\
Das Verhalten eines harmonischen Schwingsystems ist durch eine Eigenfrequenz und Güte vollständig beschrieben.
