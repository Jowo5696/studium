%{{{ Formatierung

\documentclass[a4paper,12pt]{article}

\usepackage{physics_notetaking}

%%% dark red
%\definecolor{bg}{RGB}{60,47,47}
%\definecolor{fg}{RGB}{255,244,230}
%%% space grey
%\definecolor{bg}{RGB}{46,52,64}
%\definecolor{fg}{RGB}{216,222,233}
%%% purple
%\definecolor{bg}{RGB}{69,0,128}
%\definecolor{fg}{RGB}{237,237,222}
%\pagecolor{bg}
%\color{fg}

\newcommand{\td}{\,\text{d}}
\newcommand{\RN}[1]{\uppercase\expandafter{\romannumeral#1}}
\newcommand{\zz}{\mathrm{Z\kern-.3em\raise-0.5ex\hbox{Z} }}

\newcommand\inlineeqno{\stepcounter{equation}\ {(\theequation)}}
\newcommand\inlineeqnoa{(\theequation.\text{a})}
\newcommand\inlineeqnob{(\theequation.\text{b})}
\newcommand\inlineeqnoc{(\theequation.\text{c})}

\newcommand\inlineeqnowo{\stepcounter{equation}\ {(\theequation)}}
\newcommand\inlineeqnowoa{\theequation.\text{a}}
\newcommand\inlineeqnowob{\theequation.\text{b}}
\newcommand\inlineeqnowoc{\theequation.\text{c}}

\renewcommand{\refname}{Source}
\renewcommand{\sfdefault}{phv}
%\renewcommand*\contentsname{Contents}

\pagestyle{fancy}

\sloppy

\numberwithin{equation}{section}

%}}}

\begin{document}

%{{{ Titelseite

\title{Praktikumsversuche}
\author{Jonas Wortmann}
\maketitle
\pagenumbering{gobble}

%}}}

\newpage

%{{{ Inhaltsverzeichnis

\fancyhead[L]{\thepage}
\fancyfoot[C]{}
\pagenumbering{arabic}

\tableofcontents

%}}}

\newpage

%{{{

\fancyhead[R]{\leftmark\\\rightmark}

%{{{ 102
\section{102: Freie und erzwungene Schwingung mit Dämpfung}
In diesem Versuch sollen die grundlegenen Gesetzmäßigkeiten der mechanischen Schwingung anhand des \textsc{Pohl}'schen Drehpendel untersucht werden.\\\\
\textbf{Theorie} 
\begin{enumerate}[label=--]
        \item Wirbelstrombremse \hspace{25pt}
                Ein leitender Drehkörper bewegt sich durch einen Luftspalt in einem Elektromagneten, dessen Feldstärke variabel ist. 
                Aufgrund der Bewegung des Drehkörpers erfahren die Elektronen im Drehkörper ein zeitlich veränderliches Induktionsfeld, wodurch ein elektrische Rotationsfeld aufgebaut wird. 
                Dieses Rotationsfeld erzeugt ein dem Elektromagneten entgegengerichtetes Magnetfeld (\textsc{Lenz}'sche Regel), welches die Pendelbewegung bremst.
        \item Resonanz \hspace{25pt} 
                Resonanz bezeichnet die Fähigkeit eines Systems, seine Amplitude in Abhängigkeit einer Erregerfrequenz zu erhöhen.
                Ist die Dämpfung des Systems bei $\delta =0$, so kommt es für ein Verhältnis von Erreger-- und Eigenfrequenz von 1 zur Resonanzkatastrophe.
\end{enumerate}
\textbf{Messungen} 
\begin{enumerate}[label=--]
        \item Eigenfrequenz \hspace{25pt} 
                Die Eigenfrequenz wird mit der Schwingungsdauer bei abgeschalteter Wirbelstrombremse und abgeschaltetem Motor bestimmt.
                \begin{align} 
                        \nu _0=\tfrac{1}{T}
                .\end{align} 
        \item Güte \hspace{25pt} 
                Das Verhältnis gespeicherter Energie zum thermischen Energieverlust während der folgenden Schwingung.
                Zur Bestimmung kann das Verhältnis der Frequenz der relativen Maximalamplitude, zum Abstand der beiden Frequenzen bei denen die relative Amplitude auf das $2^{-1/2}$--fache abgefallen ist verwendet werden.
                \begin{align} 
                        Q=\dfrac{\nu _{\text{max} }}{\Delta \nu }
                .\end{align} 
                Die Güte kann auch über eine Dämpfung bestimmt werden. Mit abklingenden Amplituden $\varphi _n\left(t\right)=\varphi _0\text{e}^{-\beta nT}$ folgt 
                \begin{align} 
                        \ln \varphi _n=\ln \varphi _0-\beta Tn
                .\end{align} 
                Trägt man $\varphi $ gegen $n$ auf folgt aus der Steigung
                \begin{align} 
                        \dfrac{\Delta \ln \varphi _n}{\Delta n}=-\beta T=:-\ln K
                ,\end{align} 
                folgt das Dämpfungsverhältnis $K$ und die Güte $Q=\tfrac{\pi }{\beta T}$.
        \item Resonanzkurve \hspace{25pt} 
                Die Resonanzkurve kann durch Auftragen der Amplitude des schwingfähigen Systems gegen die Erregerfrequenz bestimmt werden.
\end{enumerate}
%}}}

%{{{ 104
\newpage
\section{104: Physisches Pendel}
In diesem Versuch soll das Trägheitsmoment anhand eines physischen Pendels untersucht werden.\\\\
\textbf{Theorie} 
\begin{enumerate}[label=--]
        \item Trägheitsmoment \hspace{25pt} 
                Das Trägheitsmoment beschreibt die Trägheit einer Masse bei Rotation.
                Für eine Scheibe ist es
                \begin{align} 
                        \Theta =\dfrac{1}{2}mr^2
                .\end{align} 
                In diesem Versuch ist die Pendeldauer der Scheibe gegeben durch
                \begin{align} 
                        T^2=4\pi ^2\dfrac{\Theta }{D}=\left(\dfrac{2\pi }{\omega _0}\right)^2
                ,\end{align} 
                mit $D$ der Richtkonstante des durch die Erdanziehung verursachten Drehmoments. 
                Es gilt
                \begin{align} 
                        M=-D\varphi \approx -amg\varphi \approx -amg\sin \varphi 
                ,\end{align} 
                mit $a$ der Verschiebung der Rotationsachse. 
        \item \textsc{Steiner}'scher Satz \hspace{25pt} 
                Verschiebt sich die Rotationsachse parallel zur ursprünglichen Rotationsache um den Abstand $a$, dann gilt für das gesamte Drehmoment
                \begin{align} 
                        \Theta '=\Theta +ma^2
                .\end{align} 
\end{enumerate}
\textbf{Messung} 
\begin{enumerate}[label=--]
        \item Trägt man $aT^2$ gegen $a^2$ mit Hilfe von
                \begin{align} 
                        aT^2=\dfrac{4\pi ^2\Theta }{mg}+\dfrac{4\pi ^2}{g}a^2
                \end{align} 
                auf, folgt aus der Steigung $\tfrac{4\pi ^2}{g}$.
                Die Amplitude darf hier nur wenige Grad betragen, da sonst keine Kleinwinkelnäherung mehr gilt.
\end{enumerate}
%}}}

%{{{ 106
\newpage
\section{106: Trägheitsmoment}
In diesem Versuch soll das Trägheitsmoment eines Rades anhand der Erhaltungssätze der Mechanik bestimmt werden.\\\\
\textbf{Messung} 
\begin{enumerate}[label=--]
        \item Trägheitsmoment aus Energieerhaltung \hspace{25pt} 
                Aus der Energieerhaltung
                \begin{align} 
                        mgh=\dfrac{1}{2}\left[mv^2+\Theta \omega ^2\right]=\dfrac{1}{2}\left[mr^2+\Theta \right]\omega ^2
                ,\end{align} 
                mit $r$ dem Radius (der Stelle an der der Faden befestigt ist) und $\Theta $ dem Trägheitsmoment des sich drehenden Rades und $m$ der aus Höhe $h$ fallenden Masse, folgt unmittelbar das Trägheitsmoment.
                Für die Auswertung wird $h$ gegen $\omega ^2$ aufgetragen.
        \item Trägheitsmoment aus Impulserhaltung \hspace{25pt} 
                Aus
                \begin{align} 
                        M=\dot{L}
                \end{align} 
                folgt
                \begin{align} 
                        rmg\cdot t = \left[\Theta +mr^2\right]\omega 
                ,\end{align} 
                mit $\Theta $ dem Trägheitsmoment und $r$ dem Radius (der Stelle an der der Faden befestigt ist) des Rades und $m$ der fallenden Masse.
                Für die Auswertung wird $t$ gegen $\omega $ aufgetragen.
        \item Winkelgeschwindigkeit \hspace{25pt} 
                Die Winkelgeschwindigkeit berechnet sich aus
                \begin{align} 
                        \omega =\dfrac{2\pi }{T_1}
                ,\end{align} 
                wobei $T_1$ die Umlaufzeit für einen Umlauf ist. 
                Sie berechnet sich aus $T_1=T/n$ mit $n$ Umläufen. 
                Nachdem die Masse auf dem Boden angekommen ist wird Zeit $T$ und $n$ des noch rotierenden Rades gemessen.
\end{enumerate}
%}}}

%{{{ 108
\newpage
\section{108: Elastizitätskonstanten, Biegung und Knickung}
In diesem Versuch sollen Elastizitätsmodul und Knicklast verschiedener Metalle und Schubmodul eines Torsionsdrahts bestimmt wreden.\\\\
\textbf{Theorie} 
\begin{enumerate}[label=--]
        \item Neutrale Faser \hspace{25pt} 
                Durch den Schwerpunkt einer unter Kraft deformierten Fläche verläuft die der Kraft entsprechenden neutrale Faser, die ihre Länge bei dieser Kraft nicht ändert.
                Die Faser wird allerdings gekrümmt um den Radius $\rho $. 
                Für ein Balkenstück im Abstand $y$ zur neutralen Faser ist die Dehnung
                \begin{align} 
                        \varepsilon =\dfrac{\left(y+\rho \right)-\rho }{\rho }=\dfrac{y}{\rho }
                .\end{align} 
                Die damit einhergehende Zug-- oder Druckspannung folgt aus dem \textsc{Hook}'schen Gesetz
                \begin{align} 
                        \sigma =E\varepsilon =E\dfrac{y}{\rho }
                ,\end{align} 
                mit $E$ dem Elastizitätsmodul.
                Die in die Querschnitte zu übertragenden Drehmomente um den Durchstoßpunkt der neutralen Faser sind gegeben durch
                \begin{align} 
                        M=\iint\sigma y\td y\td x=\dfrac{E}{\rho }\iint_{}^{}y^2\td y\td x\equiv \dfrac{EI}{\rho }
                ,\end{align} 
                mit $I$ dem Flächenträgheitsmoment.
        \item Elastische Linie \hspace{25pt} 
                Die elastische Linie beschreibt die Kurve der neutralen Faser. 
                Es gilt
                \begin{align} 
                        \dfrac{1}{\rho \left(z\right)}=\dfrac{w''\left(z\right)}{\left(1+w'^2\left(z\right)\right)^{3/2}}
                .\end{align} 
                Für kleine Verbiegungen ist $w'^2\left(z\right)\ll 1$. 
                Es gilt dann in erster Näherung
                \begin{align} 
                        w''\left(z\right)=\dfrac{M\left(z\right)}{EI}
                .\end{align} 
                Das Drehmoment ist $M\left(z\right)=F\cdot \left(l-z\right)$. 
                Aus den Anfangsbedingungen $w\left(0\right)=w'\left(0\right)=0$ folgt dann
                \begin{align} 
                        w\left(z\right)=\dfrac{F}{EI}\left(\dfrac{lz^2}{2}-\dfrac{z^3}{6}\right)
                .\end{align} 
                Die maximale Strecke der Biegung ist am Balkenrand bei $z=l$, mit $c=\tfrac{F}{EI}\tfrac{l^3}{3}$.
        \item Knicklast \hspace{25pt} 
                Hier wird ein senkrecht gelagerter bereits ausgelenkter Stab betrachtet.
                Aus der DGL der elastischen Linie
                \begin{align} 
                        w''\left(z\right)+\dfrac{F_0}{EI}w\left(z\right)=0
                ,\end{align} 
                folgt mit den Anfangsbedingungen $w\left(0\right)=w\left(l\right)=0$ die Knicklast
                \begin{align} 
                        F_0=EI\dfrac{\pi ^2}{l^2}
                .\end{align} 
        \item Schubmodul \hspace{25pt} 
                Die Theorie für den Drehschwinger ist analog zu 104.
                Aus der Richtkonstante folgt das Schubmodul $G$ mit
                \begin{align} 
                        D=2\left(\dfrac{\pi }{2}\dfrac{r^4}{l}G\right)
                .\end{align} 
                Die Richtkonstante bestimmt man über den Geradenfit von $T^2$ gegen $a^2$ 
                \begin{align} 
                        T^2=\dfrac{4\pi ^2\left(\Theta _{\text{Stange}}+\Theta _{\text{Zusatzmasse}}\right)}{D}+\dfrac{8\pi ^2m}{D}a^2
                .\end{align} 
\end{enumerate}
\textbf{Messung}
\begin{enumerate}[label=--]
        \item Knicklast \hspace{25pt} 
                Um die Knicklast zu messen werden verschiedene Stäbe vertikal eingespannt und mit Gewichten von oben belastet.
                Die Auslenkung wird gegen die aufgelegte Kraft aufgetragen. 
                Die Knicklast ist an der Stelle, an der die Auslenkung stark ansteigt.
        \item Elastizitätsmodul \hspace{25pt} 
                Das Elastizitätsmodul kann aus der Steigung der Geraden bestimmt werden, wenn Auslenkung gegen Kraft aufgetragen wird.
\end{enumerate}
%}}}

%{{{ 110
\newpage
\section{110: Spezifische Wärmekapazität -- Adiabatenexponent von Luft}
In diesem Versuch werden die spezifischen Wärmekapazitäten von Metallen bestimmt, die \textsc{Dulong}--\textsc{Petit}'sche Regel bestätigt und der Adiabatenexponent von Luft bestimmt.\\
\textbf{Theorie} 
\begin{enumerate}[label=--]
        \item Wärme \hspace{25pt} 
                Die Wärme oder Wärmemenge ist der Teil der Energie der von einem thermodynamischen System aufgenommen oder Abgegeben wird.
                Sie ist näherungsweise
                \begin{align} 
                        Q=C\left(T_2-T_1\right)
                .\end{align} 
                Tauschen zwei Körper mit verschiedenen Temperaturen Energie aus, bis sie die gleiche Endtemperatur $T_=$ haben, gilt
                \begin{align} 
                        C\left(T_1-T_=\right)=C'\left(T_=-T_1'\right)
                .\end{align} 
        \item Wärmekapazität \hspace{25pt} 
                Die Wärmekapazität ist die Proportionalitätskonstante der Wärme.
                Sie ist proportional zur Masse oder Stoffmenge
                \begin{align} 
                        C=c _{\text{spez.}}m=c _{\text{mol.}}n
                .\end{align} 
                Die molare Wärmekapazität von (fast) idealen Gasen und Flüssigkeiten ist
                \begin{align} 
                        c _{\text{mol.}}=\dfrac{1}{2}fR
                .\end{align} 
                Näherungsweise für einen kleinen Temperaturbereich ist die Wärmekapazität von Festkörpern (mit $f=6$) gegeben durch die \textsc{Dulong}--\textsc{Petit}'sche Regel
                \begin{align} 
                        c _{\text{mol.}}=3R
                .\end{align} 
        \item Adiabatenkoeffizient \hspace{25pt} 
                Der Adiabatenkoeffizient ist definiert als 
                \begin{align} 
                        \kappa :=\dfrac{C_p}{C_V}
                ,\end{align} 
                mit ${}_p$ unter isobarer und ${}_V$ unter isochorer Temperaturänderung.
                Der Adiabatenkoeffizient findet sich in der Adiabatengleichung wieder
                \begin{align} 
                        p_1V_1^\kappa =p_2V_2^\kappa =\text{const.}
                .\end{align} 
\end{enumerate}
\textbf{Messungen} 
\begin{enumerate}[label=--]
        \item Wärmekapazität \hspace{25pt} 
                Die Wärmekapazität lässt sich mit einem Wasserkalorimeter bestimmen. 
                Dafür wird eine Masse in einen mit Wasser gefüllten Messingbecher gegeben und gewartet bis sich die Temperatur von Masse und Wasser + Messingbecher ausgeglichen haben. 
                Mit den gemessenen Ausgangstemperaturen und bekannter Wärmekapazität von Wasser und Messing lässt sich dann die Wärmekapazität der Masse bestimmen.\par
                Da der Wärmeaustausch nicht instantan ist und zudem Wärme an die Umgebung verloren geht wird die theoretische Ausgleichstemperatur nicht erreicht.
                Der theoretische instantane Temperaturausgleich lässt sich simulieren, indem eine senkrechte Linie durch den Punkt der größten Steigung gezeichnet wird und die Temperatur des Kalorimeters und die Ausgleichstemperatur an den Schnittstellen mit der Vor-- bzw.\ Nachkurve abgelesen werden.
        \item Adiabatenkoeffizient \hspace{25pt} 
                Der Adiabatenkoeffizient kann mit Hilfe einer freien Schwingung berechnet werden.
                Dafür wird ein Gefäß mit einem Gas gefüllt und an einer langen zylinderförmigen Öffnung mit einem frei Beweglichen Korken aus Metall \glqq verschlossen\grqq{}.
                Unter konstanter Gaszufuhr (Kraft nach außen) und der Gewichtskraft (Kraft nach innen) kann der Korken zu einer freien angeregten Schwingung gebracht werden.
                Diese Schwingung besteht darin, dass der Korken von dem einströmenden Gas aus dem Gefäß herausgedrückt wird, das Gas oben entweicht und der Korken dadruch wieder in das Gefäß fällt, bis genug Druck vom Gas vorhanden ist, um den Korken wieder herauszudrücken.
                Geht man vom Gleichgewicht aus gilt folgende Gleichung
                \begin{align} 
                        p_0=p _{\text{Gas}}=p _{\text{Luft}}+p _{\text{Gewichtskraft}}=p_L+\dfrac{mg}{\pi r^2}
                .\end{align} 
                Hieraus und aus der Adiabatengleichung (der Schwingvorgang verläuft so schnell, dass er quasi adiabatisch ist) kann die DGL des schwingfähigen System aufgestellt werden
                \begin{align} 
                        \ddot{x}+\underbrace{\dfrac{\pi ^2r^4p_0\kappa }{mV_0}}_{=\omega _0^2}x=0
                .\end{align} 
                Mit $T=\tfrac{2\pi }{\omega _0}$ folgt für den Adiabatenkoeffizient
                \begin{align} 
                        \kappa =\dfrac{4mV_0}{T^2r^4p_0}
                .\end{align} 
\end{enumerate}
%}}}

%{{{ 112
\newpage
\section{112: Wärmeausdehnung von Festkörpern}
In diesem Versuch werden Wärmeausdehnungskoeffizienten von verschiedenen Metallen experimentell bestimmt.\\\\
\textbf{Theorie} 
\begin{enumerate}[label=--]
        \item Ausdehnung von Festkörpern \hspace{25pt} 
                Ein Festkörper dehnt sich aus, da auf Grund von Temperaturzufuhr die kinetische Energie der Atome erhöht wird. Die Erhöhung von kinetischer Energie äußert sich in einer größeren Amplitude der Gitterschwingung. 
                Da das \textsc{Lennard-Jonas}--Potential asymmetrisch ist, erhöht sich der über die Schwingungsbewegung gemittelte Abstand der Atome.
        \item Längenänderung \hspace{25pt} 
                Die Längenänderung von Festkörpern bei einer Temperaturänderung $\Delta T$ lässt sich näherungsweise beschreiben durch
                \begin{align} 
                        l=l_0+\Delta l \approx l_0+l_0\alpha \Delta T=l_0\left(1+\alpha \Delta T\right)
                ,\end{align} 
                mit $\alpha $ dem Wärmeausdehnungskoeffizient (einer Materialeigenschaft).
\end{enumerate}
\textbf{Messungen} 
\begin{enumerate}[label=--]
        \item Wärmeausdehnungskoeffizient \hspace{25pt} 
                Dünne Rohre aus verschiedenen Materialien werden von sich erwärmenden Wasser durchflossen, wodurch sich die Rohre näherungsweise immer mit dem Wasser im thermischen Gleichgewicht befinden.
                Zur Auswertung kann $l$ gegen $\Delta T$ aufgetragen werden.
\end{enumerate}
%}}}

\include{Anhang.tex}

%}}}

\end{document}
