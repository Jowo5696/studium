%:LLPStartPreview
%:VimtexCompile(SS)

%{{{ Formatierung

\documentclass[a4paper,12pt]{article}

\usepackage{physics_notetaking}

%%% dark red
%\definecolor{bg}{RGB}{60,47,47}
%\definecolor{fg}{RGB}{255,244,230}
%%% space grey
%\definecolor{bg}{RGB}{46,52,64}
%\definecolor{fg}{RGB}{216,222,233}
%%% purple
%\definecolor{bg}{RGB}{69,0,128}
%\definecolor{fg}{RGB}{237,237,222}
%\pagecolor{bg}
%\color{fg}

\newcommand{\RN}[1]{\uppercase\expandafter{\romannumeral#1}}
\newcommand{\zz}{\mathrm{Z\kern-.3em\raise-0.5ex\hbox{Z} }}
\renewcommand{\refname}{Source}
\renewcommand{\sfdefault}{phv}
%\renewcommand*\contentsname{Contents}

%\bibliographystyle{alphadin}
\pagestyle{fancy}

\sloppy
%\pagenumbering{gobble}

\lhead{Jonas Wortmann}\chead{Notizen}\rhead{physik261}

%}}}

\begin{document}

%{{{ Inhaltsverzeichnis
\rhead{Inhaltsverzeichnis}
\tableofcontents

%}}}

\newpage

%{{{ physik261

\rhead{physik261}
\section{Statistik}
\subsection{Binomialverteilung}
Die Binomialverteilung ist eine diskrete Wahrscheinlichkeitsverteilung, welche besagt, mit welcher Wahrscheinlichkeit $P$ von $n$ unabhängigen \glqq ja/nein\grqq{}--Entscheidungen, deren jede mit der Wahrscheinlichkeit $p$ positiv verläuft, insgesamt $k$ positiv verlaufen, wobei $0\leq k\leq n$
\[ 
        P_B\left(k;p,n\right)={n\choose k}p^k\left(1-p\right)^{n-k}\qquad {n\choose k}=\dfrac{n!}{k!\left(n-k\right)!}
.\] 
Der \textbf{Erwartungswert} ist
\[ 
        \left\langle k\right\rangle =\sum_{k=0}^{n}kP_B\left(k;n,p\right)=np
.\] 
Die \textbf{Varianz} ist
\[ 
        V=\left\langle \left(k-\left\langle k\right\rangle \right)^2\right\rangle =np\left(1-p\right)
.\] 
Die \textbf{Standardabweiung} ist
\[ 
        \sigma =\sqrt[ ]{V}
.\] 

\subsection{Poisson--Verteilung}
Die Poisson--Verteilung fungiert als Binomialverteilung für sehr große $n$ mit sehr kleinen $p$
\[ 
        P_P\left(k;\mu \right)=\dfrac{\mu ^k}{k!}e^{-\mu }
.\] 
Der \textbf{Erwartungswert} ist
\[ 
        \left\langle k\right\rangle =\sum_{k=0}^{\infty}kP_P\left(k;\mu \right)=np=\mu \qquad n\gg1,p\ll1
.\] 
Die \textbf{mittlere Streuung} oder \textbf{Fehler} ist
\[ 
        \sigma _k=\sqrt[ ]{\left\langle k\right\rangle }
.\] 

\subsection{Gauss--Verteilung}
Die Gauss--Verteilung behält weiterhin ein sehr großes $n$ und ein beliebiges $p$ sodass gilt $\sqrt[ ]{np\left(1-p\right)}=\sigma \gg1$
\[ 
        P_G\left(k\right)=\dfrac{1}{\sqrt[ ]{2\pi \sigma ^2}}e^{-\tfrac{\left(k-\mu \right)^2}{2\sigma ^2}}
.\] 
Der \textbf{Mittelwert} ist
\[ 
        \left\langle k\right\rangle =\mu =np
.\] 
Die \textbf{Streuung} ist
\[ 
        \sigma =\sqrt[ ]{np\left(1-p\right)}=\sqrt[ ]{\mu \left(1-p\right)}=\sqrt[ ]{\left\langle k\right\rangle \left(1-p\right)}
.\] 
Die \textbf{Varianz} ist
\[ 
        V=\sigma ^2
.\] 
Für eine zu errechnende Größe $g$ und eine Messung $U_1,\hdots ,U_n$ mit Messunsicherheit $\Delta U_1,\hdots ,\Delta U_n$ ergibt sich
\[ 
        g=\sqrt[ ]{\sum_{i=1}^{n}\left(\diffp[1]{g}{U_i}\Delta U_i\right)^2}
.\] 

%}}}

\end{document}
