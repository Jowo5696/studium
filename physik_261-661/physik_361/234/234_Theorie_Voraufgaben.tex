%:LLPStartPreview
%:VimtexCompile(SS)

%{{{ Formatierung

\documentclass[a4paper,12pt]{article}

\usepackage{physics_notetaking}

%%% dark red
%\definecolor{bg}{RGB}{60,47,47}
%\definecolor{fg}{RGB}{255,244,230}
%%% space grey
%\definecolor{bg}{RGB}{46,52,64}
%\definecolor{fg}{RGB}{216,222,233}
%%% purple
%\definecolor{bg}{RGB}{69,0,128}
%\definecolor{fg}{RGB}{237,237,222}
%\pagecolor{bg}
%\color{fg}

\newcommand{\td}{\,\text{d}}
\newcommand{\RN}[1]{\uppercase\expandafter{\romannumeral#1}}
\newcommand{\zz}{\mathrm{Z\kern-.3em\raise-0.5ex\hbox{Z} }}

\newcommand\inlineeqno{\stepcounter{equation}\ {(\theequation)}}
\newcommand\inlineeqnoa{(\theequation.\text{a})}
\newcommand\inlineeqnob{(\theequation.\text{b})}
\newcommand\inlineeqnoc{(\theequation.\text{c})}

\newcommand\inlineeqnowo{\stepcounter{equation}\ {(\theequation)}}
\newcommand\inlineeqnowoa{\theequation.\text{a}}
\newcommand\inlineeqnowob{\theequation.\text{b}}
\newcommand\inlineeqnowoc{\theequation.\text{c}}

\renewcommand{\refname}{Source}
\renewcommand{\sfdefault}{phv}
%\renewcommand*\contentsname{Contents}

\pagestyle{fancy}

\sloppy

\numberwithin{equation}{section}

%}}}

\begin{document}

%{{{ Titelseite

\title{Versuch 234 $|$ Wechselstromwiderstände, Phasenschieber, RC--Glieder und Schwingungen}
\author{Jonas Wortmann}
\maketitle
\pagenumbering{gobble}

%}}}

\newpage

%{{{ Inhaltsverzeichnis

\fancyhead[L]{\thepage}
\fancyfoot[C]{}
\pagenumbering{arabic}

\tableofcontents

%}}}

\newpage

%{{{

\fancyhead[R]{\leftmark\\\rightmark}

\newpage
\section{Einführung}
In diesem Versuch sollen Kapazitäten und Induktivitäten mit einer Wechselstrombrücke gemessen werden, eine Phasenschieberschaltung aufgebaut und die komplexe Schreibweise und Darstellung von Wechselstromgeräten erlernt werden.
Oft sind in der Elektronik aus einer Signalspannung, die aus einem Gemisch von Frequenzen besteht, bestimmte Frequenzanteile einer vorgegebenen Spannung $\omega _{\,\text{grenz}\,}$ zu unterdrücken.
Dies geschieht mit Schaltungstypen, die aus frequenzabhängigen Widerständen bestehen.
\indent Wirkungsweise und Berechnung solcher Schaltungen werden behandelt.
Viele bei der erzwungenen Schwingung am Drehpendel werden beobachtet, auf einen elektrischen Schwingkreis übertragen und experimentell bestätigt.

\subsection{Theorie}

\newpage
\section{Voraufgaben}


%}}}

\end{document}
