%:LLPStartPreview
%:VimtexCompile(SS)

%{{{ Formatierung

\documentclass[a4paper,12pt]{article}

\usepackage{physics_notetaking}

%%% dark red
%\definecolor{bg}{RGB}{60,47,47}
%\definecolor{fg}{RGB}{255,244,230}
%%% space grey
%\definecolor{bg}{RGB}{46,52,64}
%\definecolor{fg}{RGB}{216,222,233}
%%% purple
%\definecolor{bg}{RGB}{69,0,128}
%\definecolor{fg}{RGB}{237,237,222}
%\pagecolor{bg}
%\color{fg}

\newcommand{\td}{\,\text{d}}
\newcommand{\RN}[1]{\uppercase\expandafter{\romannumeral#1}}
\newcommand{\zz}{\mathrm{Z\kern-.3em\raise-0.5ex\hbox{Z} }}

\newcommand\inlineeqno{\stepcounter{equation}\ {(\theequation)}}
\newcommand\inlineeqnoa{(\theequation.\text{a})}
\newcommand\inlineeqnob{(\theequation.\text{b})}
\newcommand\inlineeqnoc{(\theequation.\text{c})}

\newcommand\inlineeqnowo{\stepcounter{equation}\ {(\theequation)}}
\newcommand\inlineeqnowoa{\theequation.\text{a}}
\newcommand\inlineeqnowob{\theequation.\text{b}}
\newcommand\inlineeqnowoc{\theequation.\text{c}}

\renewcommand{\refname}{Source}
\renewcommand{\sfdefault}{phv}
%\renewcommand*\contentsname{Contents}

\pagestyle{fancy}

\sloppy

\numberwithin{equation}{section}

%}}}

\begin{document}

%{{{ Titelseite

\title{Versuch 234 $|$ Wechselstromwiderstände, Phasenschieber, RC--Glieder und Schwingungen}
\author{Jonas Wortmann}
\maketitle
\pagenumbering{gobble}

%}}}

\newpage

%{{{ Inhaltsverzeichnis

\fancyhead[L]{\thepage}
\fancyfoot[C]{}
\pagenumbering{arabic}

\tableofcontents

%}}}

\newpage

%{{{

\fancyhead[R]{\leftmark\\\rightmark}

\section{Einführung}
In diesem Versuch sollen Kapazitäten und Induktivitäten mit einer Wechselstrombrücke gemessen werden, eine Phasenschieberschaltung aufgebaut und die komplexe Schreibweise und Darstellung von Wechselstromgeräten erlernt werden.
Oft sind in der Elektronik aus einer Signalspannung, die aus einem Gemisch von Frequenzen besteht, bestimmte Frequenzanteile einer vorgegebenen Spannung $\omega _{\,\text{grenz}\,}$ zu unterdrücken.
Dies geschieht mit Schaltungstypen, die aus frequenzabhängigen Widerständen bestehen.
\indent Wirkungsweise und Berechnung solcher Schaltungen werden behandelt.
Viele bei der erzwungenen Schwingung am Drehpendel werden beobachtet, auf einen elektrischen Schwingkreis übertragen und experimentell bestätigt.

\subsection{Theorie}
Die Wheatstone--Brücke kann verwendet werden, um einen unbekannten Widerstand herauszufinden. Dabei wird ein Potentiometer zwischen den Widerständen angebracht und die Widerstände so variiert, dass keine Potentialdifferenz zwischen den Anschlusspunkten des Potentiometers zu messen ist. Diese Abgleichsbedingung ist
\begin{align} 
        \dfrac{R_X}{R_Y}=\dfrac{Z_1}{Z_X}
.\end{align} 
Für einen verlustfreien Kondensator gilt
\begin{align} 
        \dfrac{R_X}{R_Y}=\dfrac{C_X}{C_1}
.\end{align} 
Bei Spulen lassen sich diese Gleichungen nicht verwenden, da der Ohm'sche Widerstand der Spulen nicht vernachlässigt werden darf. Es ist also zunächst
\begin{align} 
        \dfrac{R_X}{R_Y}=\dfrac{R_1+\text{i}\omega L_1}{R_2+\text{i}\omega L_2}\Rightarrow \dfrac{R_X}{R_Y}=\dfrac{L_1}{L_2}=\dfrac{R_1}{R_2}
.\end{align} 
Da sich beide Bedingungen zugleich nicht erfüllen lassen wird ein weiteres Potentiometer verwendet. Dann ist die Abgleichsbedingung
\begin{align} 
        \dfrac{R_X}{R_Y}=\dfrac{L_1}{L_2}=\dfrac{R_1+R_A}{R_2+R_B}
.\end{align} 
\hfill\\\textbf{Phasenschieber}\\ 
Der Phasenschieber wird verwendet, um die Phase $\varphi $ der Ausgangsspannung $U_{AB}$ relativ zur Eingangsspannung $U_E$ zu verschieben, aber die Spannung selbst konstant zu lassen. Dabei wird die Phasenverscheibung von Kondensator und Widerstand von $\ang{90}$ verwendet.
\\\hfill\\\textbf{Elektrischer Schwingkreis}\\ 
Die Differentialgleichung für einen elektrischen Schwingkreis folgt aus
\begin{align} 
        U_L\left(t\right)*U_R\left(t\right)*U_C\left(t\right)&=U_E\cos \left(\omega t\right)\\
        L\dot{I}+RI+\dfrac{1}{C}\int_{}^{}I\td t&=U_E\cos \left(\omega t\right)\\
        L\ddot{q}+R\dot{q}+\dfrac{1}{C}q&=U_E\cos \left(\omega t\right)
.\end{align} 
Die Lösung für $q\left(t\right)$ ist
\begin{align} 
        q\left(t,\omega \right)&=q_0\left(\omega \right)\cos \left(\omega t-\alpha \right)\\
        q_0\left(\omega \right)&=\dfrac{U_E}{L}\dfrac{1}{\,\sqrt[]{\left(\omega _0^2-\omega ^2\right)^2+\tfrac{\omega _0^2\omega ^2}{Q^2} }}\\
        \tan \alpha &=\dfrac{1}{Q}\dfrac{\omega _0\omega }{\omega _0^2-\omega ^2}
.\end{align} 
Weiter erhält man durch Einsetzen
\begin{align} 
        \omega _0^2&=\dfrac{1}{LC}\\
        Q&=\omega _0\dfrac{L}{R}=\dfrac{1}{\omega _0RC}=\dfrac{1}{R}\,\sqrt[]{Z_CZ_L}\\
        \omega _{\,\text{max}\,}&=\omega _0\,\sqrt[]{1-\dfrac{1}{2Q^2}}
,\end{align} 
mit $\omega _0$ der Eigenfrequenz und $Q$ der Güte. Die Ladung kann als Spannung am Kondensator, mit $U\left(t,\omega \right)=\tfrac{q\left(t,\omega \right)}{C}$, gemessen werden. Die Resonanzkruve ist
\begin{align} 
        U\left(\omega \right)&=U_E\omega _0^2\dfrac{1}{\,\sqrt[]{\left(\omega _0^2-\omega ^2\right)+\tfrac{\omega _0^2\omega ^2}{Q^2}}}
.\end{align} 
Für $\omega =\omega _0$ hat $U$ ein Maximum, bei $U_0=U\left(\omega _0\right)=U_EQ$. Außerdem ist wie beim Drehpendel $\tfrac{\omega _0}{\Delta \omega }\approx Q$.


\newpage
\section{Voraufgaben}
\subsection{234.A}
$U_X$ und $U_Y$ sind nicht in Phase mit $U_1$ und $U_2$, was sie aber für die Abgleichsbedingung der Wheatstone--Brücke sein müssen. Deswegen wird ein weiteres Potentiometer mit den Spannungen $U_A$ und $U_B$ hinzugefügt, welches diese Phasenverschiebung kompensiert. [zeichne Zeigerdiagram]

\subsection{234.B}
$U_R$ und $U_C$ sind um $\psi =\ang{90}$ phasenverschoben. Wird der Betrag von diesen Spannungen variiert (wobei $U_E=U_R+U_C$ weiterhin gelten muss), so ändert sich die Phase $\phi $ von $U_{AB}$.\\\indent
Wenn $R_1\neq R_2$, dann liegt $A$ nicht mehr im Mittelpunkt des Kreises, woraus folgt, dass sich $|U_{AB}|$ und $\phi $ ändern.\\\indent
Die Widerstände können auch durch Kondensatoren ersetzt werden, wenn die Bedingung $Z_1=Z_2$ weiterhin gilt.\\\indent
Für eine Phase von $\phi =\ang{0}$ muss $U_C=0$ sein. Für eine Phase von $\phi =\ang{180}$ muss $U_R=0$ sein.\\\indent
[andere einfache schaltungen]

\subsection{234.C}
Der maximale Strom im RC--Kreis ist
\begin{align} 
        I&=\dfrac{U}{|Z|}\\
         &=\dfrac{U_R+U_C}{|Z_R+Z_C|}\\
         &=\dfrac{U_R+U_C}{\,\sqrt[]{R^2+\tfrac{1}{\omega^2 C^2} }}
.\end{align} 
Der Phasenwinkel im Zeigerdiagramm ist zwischen $U_R$ und $U_C$ zu finden.

\subsection{234.D}
Die Bewegungsgleichung des mechanischen Drehpendels lautet
\begin{align} 
        \theta \ddot{\varphi }+r\dot{\varphi }+D\varphi &=M\cos \left(\omega t\right)
,\end{align} 
mit $\varphi $ dem Auslenkungswinkel, $\theta $ dem Trägheitsmoment, $r$ Dämpfungsfaktor, $D$ der Richtkonstante $\omega $ der Anregerfrequenz und $M$ dem Drehmoment.

\subsection{234.E}
Die korrespondierenden physikalischen Größen sind
\begin{align} 
        \varphi &\equiv q\\
        L&\equiv \theta \\
        R&\equiv r\\
        \dfrac{1}{C}&\equiv D\\
        U_E&\equiv M
.\end{align} 
Die Auslenkung des Schwingkreises ist die Ladung $q$.

\newpage
\section{Fragen}
Bei den Abbildungen 234.1, 234.2 und 234.5 aus dem Praktikumsheft besteht Kurzschlussgefahr, da 

%}}}

\end{document}
