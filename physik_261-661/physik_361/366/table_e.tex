%{{{ Formatierung

\documentclass[a4paper,12pt]{article}

\usepackage{physics_notetaking}

%%% dark red
%\definecolor{bg}{RGB}{60,47,47}
%\definecolor{fg}{RGB}{255,244,230}
%%% space grey
%\definecolor{bg}{RGB}{46,52,64}
%\definecolor{fg}{RGB}{216,222,233}
%%% purple
%\definecolor{bg}{RGB}{69,0,128}
%\definecolor{fg}{RGB}{237,237,222}
%\pagecolor{bg}
%\color{fg}

\newcommand{\td}{\,\text{d}}
\newcommand{\RN}[1]{\uppercase\expandafter{\romannumeral#1}}
\newcommand{\zz}{\mathrm{Z\kern-.3em\raise-0.5ex\hbox{Z} }}

\newcommand\inlineeqno{\stepcounter{equation}\ {(\theequation)}}
\newcommand\inlineeqnoa{(\theequation.\text{a})}
\newcommand\inlineeqnob{(\theequation.\text{b})}
\newcommand\inlineeqnoc{(\theequation.\text{c})}

\newcommand\inlineeqnowo{\stepcounter{equation}\ {(\theequation)}}
\newcommand\inlineeqnowoa{\theequation.\text{a}}
\newcommand\inlineeqnowob{\theequation.\text{b}}
\newcommand\inlineeqnowoc{\theequation.\text{c}}

\renewcommand{\refname}{Source}
\renewcommand{\sfdefault}{phv}
%\renewcommand*\contentsname{Contents}

\pagestyle{fancy}

\sloppy

\numberwithin{equation}{section}

%}}}

\begin{document}

%{{{

\begin{tabular}{lll}
        $\tfrac{1}{\lambda ^2}$ $\left[\tfrac{1}{\text{m}^2}\cdot 10^{12}\right]$ & Brechungsindex & $\Delta $Brechungsindex \\
        \hline
        241.23 & 1.61082 & 0.00018 \\
        298.23 & 1.62128 & 0.00018 \\
        300.41 & 1.62161 & 0.00018 \\
        335.34 & 1.62485 & 0.00017 \\
        386.62 & 1.62923 & 0.00017 \\
        434.05 & 1.63379 & 0.00017 \\
        456.94 & 1.63613 & 0.00017 \\
        526.46 & 1.64362 & 0.00017 \\
        610.69 & 1.65286 & 0.00017 
\end{tabular}

%}}}

\end{document}
