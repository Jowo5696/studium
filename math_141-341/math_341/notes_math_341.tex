%{{{ Formatierung

\documentclass[a4paper,12pt]{article}

\usepackage{physics_notetaking}

%%% dark red
%\definecolor{bg}{RGB}{60,47,47}
%\definecolor{fg}{RGB}{255,244,230}
%%% space grey
%\definecolor{bg}{RGB}{46,52,64}
%\definecolor{fg}{RGB}{216,222,233}
%%% purple
%\definecolor{bg}{RGB}{69,0,128}
%\definecolor{fg}{RGB}{237,237,222}
%\pagecolor{bg}
%\color{fg}

\newcommand{\td}{\,\text{d}}
\newcommand{\RN}[1]{\uppercase\expandafter{\romannumeral#1}}
\newcommand{\zz}{\mathrm{Z\kern-.3em\raise-0.5ex\hbox{Z} }}

\newcommand\inlineeqno{\stepcounter{equation}\ {(\theequation)}}
\newcommand\inlineeqnoa{(\theequation.\text{a})}
\newcommand\inlineeqnob{(\theequation.\text{b})}
\newcommand\inlineeqnoc{(\theequation.\text{c})}

\newcommand\inlineeqnowo{\stepcounter{equation}\ {(\theequation)}}
\newcommand\inlineeqnowoa{\theequation.\text{a}}
\newcommand\inlineeqnowob{\theequation.\text{b}}
\newcommand\inlineeqnowoc{\theequation.\text{c}}

\renewcommand{\refname}{Source}
\renewcommand{\sfdefault}{phv}
%\renewcommand*\contentsname{Contents}

\pagestyle{fancy}

\sloppy

\numberwithin{equation}{section}

%}}}

\begin{document}

%{{{ Titelseite

\title{Notizen $|$ math341}
\author{Jonas Wortmann}
\maketitle
\pagenumbering{gobble}

%}}}

\newpage

%{{{ Inhaltsverzeichnis

\fancyhead[L]{\thepage}
\fancyfoot[C]{}
\pagenumbering{arabic}

\tableofcontents

%}}}

\newpage

%{{{

\fancyhead[R]{\leftmark\\\rightmark}

\section{Komplexe Zahlen}
Die \textbf{komplexen Zahlen} sind alle Terme der Form $x+y\text{i};x,y \in \mathbb{R}$. $\text{i}$ ist die \textbf{imaginäre Einheit}, mit $\text{i}^2=-1$. Die Menge der komplexen Zahlen ist $\mathbb{C}:=\left\{x+y\text{i}:x,y \in \mathbb{R}\right\}$.\\\indent
Die komplexen Zahlen können auch als eine Ebene aufgefasst werden, $x+y\text{i}\equiv \begin{pmatrix}
        x\\y
\end{pmatrix}$.\\
Die Gleichheit ist definiert als, $x+y\text{i}=a+b\text{i}\Leftrightarrow x=a\land y=b$.\\
Folgende Operationen sind definiert 
\begin{enumerate}[label=--]
        \item $\left(x+y\text{i}\right)+\left(a+b\text{i}\right):=\left(x+a\right)+\left(y+b\right)\text{i}$ 
        \item $\left(x+y\text{i}\right)\cdot \left(a+b\text{i}\right):=xa+xb\text{i}+ya\text{i}+yb\text{i}^2=\left(xa-yb\right)+\left(xb+ya\right)\text{i}$ 
.\end{enumerate}
Sei $z=x+y\text{i} \in \mathbb{C}$.
Dann ist die \textbf{komplex konjugierte} Zahl $\overline{z}:=x-y\text{i}$. Praktisch ist dann, $z\cdot \overline{z}=\left(x+y\text{i}\right)\cdot \left(x-y\text{i}\right)=x^2+y^2$.
Zudem ist die \textbf{Norm} einer komplexen Zahl $| |z| |=\,\sqrt[]{z\cdot \overline{z}}=\,\sqrt[]{x^2+y^2}$.\\\indent
Der \textbf{Kehrwert} einer komplexen Zahl $z=x+y\text{i}\neq 0+0\text{i}$ ist $\tfrac{1}{z}=\tfrac{\overline{z}}{z\overline{z}}=\tfrac{\overline{z}}{| |z| |^2}=\tfrac{1}{x^2+y^2}\left(x-y\text{i}\right)$.\\
Folgende Rechenregeln sind gültig
\begin{enumerate}[label=--]
        \item Kommutativität: $z_1+z_2=z_2+z_1$ und $z_1\cdot z_2=z_2\cdot z_1$ 
        \item Assoziativität: $z_1+\left(z_2+z_3\right)=\left(z_1+z_2\right)+z_3$ und $z_1\cdot \left(z_2\cdot z_3\right)=\left(z_1\cdot z_2\right)\cdot z_3$
        \item Distributivität: $z_1\left(z_2+z_3\right)=z_1\cdot z_2+z_1\cdot z_3$ 
        \item Kehrwertregel: $\tfrac{1}{z_1\cdot z_2}=\tfrac{1}{z_1}\cdot \tfrac{1}{z_2}$ 
        \item Konjugation: $\overline{z_1+z_2}=\overline{z_1}+\overline{z_2}$ und $\overline{z_1\cdot z_2}=\overline{z_1}\cdot \overline{z_2}$ 
\end{enumerate}
$\left(\mathbb{C},0,1,+,\cdot \right)$ ist ein Körper, der $\mathbb{R}$ enthält.

\subsection{Komplexe Zahlen als Matrizen}
\begin{align} 
        \left[\left(x+y\text{i}\right)\left(a+b\text{i}\right)\right]_{\mathbb{R}^{2}}=\begin{pmatrix}
                xa-yb\\xb+ya
        \end{pmatrix}=\begin{pmatrix}
        x&-y\\y&x
        \end{pmatrix}\begin{pmatrix}
                a\\b
        \end{pmatrix}
.\end{align} 
Eine komplexe Zahl kann also als Matrix dargestellt werden
\begin{align} 
        \left[x+y\text{i}\right]_{\mathbb{R}^{2\times 2}}:=\begin{pmatrix}
                x&-y\\x&y
        \end{pmatrix}
.\end{align} 
Es gilt dann
\begin{align} 
        \left[z_1\cdot z_2\right]_{\mathbb{R}^2}&=\left[z_1\right]_{\mathbb{R}^{2\times 2}}\cdot \left[z_2\right]_{\mathbb{R}^{2}}\\
        \left[z_1+z_2\right]_{\mathbb{R}^2}&=\left[z_1\right]_{\mathbb{R}^{2\times 2}}+\left[z_2\right]_{\mathbb{R}^{2}}\\
        \left[\dfrac{1}{z}\right]_{\mathbb{R}^{2\times 2}}&=\left[z\right]_{\mathbb{R}^{2\times 2}}^{-1}
.\end{align} 
Also $\mathbb{C}\subseteq \mathbb{R}^{2\times 2}$, $\mathbb{C}\equiv \left\{\begin{pmatrix}
                x&-y\\y&x
\end{pmatrix} \in \mathbb{R}^{2\times 2}:x,y \in \mathbb{R}\right\}=\,\text{span}\,\left(\begin{pmatrix}
                1&0\\0&1
\end{pmatrix},\begin{pmatrix}
                0&-1\\1&0
\end{pmatrix}\right)=\left\{x \begin{pmatrix}
                1&0\\0&1
\end{pmatrix}+y \begin{pmatrix}
                0&-1\\1&0
\end{pmatrix}:x,y \in \mathbb{R}\right\}$.\\\indent
Jede Matrix $\begin{pmatrix}
        x&-y\\y&x
\end{pmatrix}$ ist das Produkt einer Drehung $\begin{pmatrix}
        \cos \theta &-\sin \theta \\\sin \theta &\cos \theta  
\end{pmatrix}$ und einer Streckung $\begin{pmatrix}
        r&0\\0&r
\end{pmatrix}$.\\
Für jede komplexe Zahl $z=x+y\text{i} \in \mathbb{C}\backslash \left\{0\right\}$, gibt es eindeutige $r>0,\theta  \in [0,2\pi):z=r\left(\cos \theta +\sin \theta \text{i}\right)$.\\\indent
Hier ist $r=\,\sqrt[]{x^2+y^2}=| |z| |$ und $\theta =\arg\left(z\right)=\begin{cases}
        \arctan \tfrac{y}{x}&;x>0,y\geq 0\\
        \tfrac{\pi }{2}&;x=0<y\\
        \pi +\arctan \tfrac{y}{x}&;x<0\\
        \tfrac{3\pi }{2}&;x=0>y\\
        2\pi +\arctan \tfrac{y}{x}&;y<0<x
\end{cases}$ 

\subsection{Polarkoordinaten}
Die komplexen Zahlen können in \textbf{Polarkoordinaten} dargestellt werden
\begin{align} 
        \mathbb{C}\ni z=re^{\text{i}\theta }\qquad e^{\text{i}\theta }=\cos \theta +\sin \theta \text{i}
,\end{align} 
mit $r>0$ und $\theta  \in [0,2\pi )$. Die Eindeutigkeit von $r$ lässt sich zeigen durch
\begin{align} 
        ||z||=\,\sqrt[]{x^2+y^2}=||re^{\text{i}\theta }||=||r||\underbrace{||e^{\text{i}\theta }||}_{=1}=r
.\end{align} 
Die Eindeutigkeit von $\theta $ lässt sich zeigen durch
\begin{align} 
        e^{\text{i}\theta _1}=e^{\text{i}\theta _2}\Leftrightarrow \theta _2-\theta _1 \in 2\pi \mathbb{Z}
.\end{align} 
Die Multiplikation ist definiert durch
\begin{align} 
        z_1\cdot z_2=r_1r_2e^{\text{i}\left(\theta _1+\theta _2\right)}
;\end{align} 
und der Kehrwert
\begin{align} 
        \dfrac{1}{e^{\text{i}\theta }}=e^{\text{i}\left(-\theta \right)}
.\end{align} 

\subsubsection{Potenzen}
Wird eine komplexe Zahl potenziert, gilt
\begin{align} 
        k \in \mathbb{Z}:z^k=r^ke^{\left(\text{i}\theta \right)^k}=r^ke^{\text{i}\left(k\theta \right)}
.\end{align} 

\subsubsection{Beispiel}
\begin{align} 
        \left(1+\text{i}\right)^{100}
.\end{align} 
In Polarkoordinaten ist $1+\text{i}$, 
\begin{align} 
        r=\,\sqrt[]{1^2+1^2}=\,\sqrt[]{2}\qquad \theta =\arg\left(1+\text{i}\right)=\arctan\left(\dfrac{1}{1}\right)=\dfrac{\pi }{4}\qquad 1+\text{i}=\,\sqrt[]{2}e^{\text{i}\tfrac{\pi }{4}}
.\end{align} 
Die Potenz ist dann
\begin{align} 
        \left(1+\text{i}\right)^{100}&=\,\sqrt[]{2}^{100}e^{\text{i}\left(100\tfrac{\pi }{4}\right)}\nonumber \\
                                     &=2^{50}e^{\text{i}25\pi }\nonumber \\
                                     &=2^{50}e^{\text{i}\pi }\nonumber \\ 
                                     &=-2^{50}
.\end{align} 

%}}}

\end{document}
