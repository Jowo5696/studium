%{{{ Formatierung

\documentclass[a4paper,12pt]{article}

\usepackage{physics_notetaking}

%%% dark red
%\definecolor{bg}{RGB}{60,47,47}
%\definecolor{fg}{RGB}{255,244,230}
%%% space grey
%\definecolor{bg}{RGB}{46,52,64}
%\definecolor{fg}{RGB}{216,222,233}
%%% purple
%\definecolor{bg}{RGB}{69,0,128}
%\definecolor{fg}{RGB}{237,237,222}
%\pagecolor{bg}
%\color{fg}

\newcommand{\td}{\,\text{d}}
\newcommand{\RN}[1]{\uppercase\expandafter{\romannumeral#1}}
\newcommand{\zz}{\mathrm{Z\kern-.3em\raise-0.5ex\hbox{Z} }}

\newcommand\inlineeqno{\stepcounter{equation}\ {(\theequation)}}
\newcommand\inlineeqnoa{(\theequation.\text{a})}
\newcommand\inlineeqnob{(\theequation.\text{b})}
\newcommand\inlineeqnoc{(\theequation.\text{c})}

\newcommand\inlineeqnowo{\stepcounter{equation}\ {(\theequation)}}
\newcommand\inlineeqnowoa{\theequation.\text{a}}
\newcommand\inlineeqnowob{\theequation.\text{b}}
\newcommand\inlineeqnowoc{\theequation.\text{c}}

\renewcommand{\refname}{Source}
\renewcommand{\sfdefault}{phv}
%\renewcommand*\contentsname{Contents}

\pagestyle{fancy}

\sloppy

\numberwithin{equation}{section}

%}}}

\begin{document}

%{{{ Titelseite

\title{Notizen $|$ math341}
\author{Jonas Wortmann}
\maketitle
\pagenumbering{gobble}

%}}}

\newpage

%{{{ Inhaltsverzeichnis

\fancyhead[L]{\thepage}
\fancyfoot[C]{}
\pagenumbering{arabic}

\tableofcontents

%}}}

\newpage

%{{{

\fancyhead[R]{\leftmark\\\rightmark}

\section{Komplexe Zahlen}
Die \textbf{komplexen Zahlen} sind alle Terme der Form $x+y\text{i};x,y \in \mathbb{R}$. $\text{i}$ ist die \textbf{imaginäre Einheit}, mit $\text{i}^2=-1$. Die Menge der komplexen Zahlen ist $\mathbb{C}:=\left\{x+y\text{i}:x,y \in \mathbb{R}\right\}$.\\\indent
Die komplexen Zahlen können auch als eine Ebene aufgefasst werden, $x+y\text{i}\equiv \begin{pmatrix}
        x\\y
\end{pmatrix}$.\\
Die Gleichheit ist definiert als, $x+y\text{i}=a+b\text{i}\Leftrightarrow x=a\land y=b$.\\
Folgende Operationen sind definiert 
\begin{enumerate}[label=--]
        \item $\left(x+y\text{i}\right)+\left(a+b\text{i}\right):=\left(x+a\right)+\left(y+b\right)\text{i}$ 
        \item $\left(x+y\text{i}\right)\cdot \left(a+b\text{i}\right):=xa+xb\text{i}+ya\text{i}+yb\text{i}^2=\left(xa-yb\right)+\left(xb+ya\right)\text{i}$ 
.\end{enumerate}
Sei $z=x+y\text{i} \in \mathbb{C}$.
Dann ist die \textbf{komplex konjugierte} Zahl $\overline{z}:=x-y\text{i}$. Praktisch ist dann, $z\cdot \overline{z}=\left(x+y\text{i}\right)\cdot \left(x-y\text{i}\right)=x^2+y^2$.
Zudem ist die \textbf{Norm} einer komplexen Zahl $| |z| |=\,\sqrt[]{z\cdot \overline{z}}=\,\sqrt[]{x^2+y^2}$.\\\indent
Der \textbf{Kehrwert} einer komplexen Zahl $z=x+y\text{i}\neq 0+0\text{i}$ ist $\tfrac{1}{z}=\tfrac{\overline{z}}{z\overline{z}}=\tfrac{\overline{z}}{| |z| |^2}=\tfrac{1}{x^2+y^2}\left(x-y\text{i}\right)$.\\
Folgende Rechenregeln sind gültig
\begin{enumerate}[label=--]
        \item Kommutativität: $z_1+z_2=z_2+z_1$ und $z_1\cdot z_2=z_2\cdot z_1$ 
        \item Assoziativität: $z_1+\left(z_2+z_3\right)=\left(z_1+z_2\right)+z_3$ und $z_1\cdot \left(z_2\cdot z_3\right)=\left(z_1\cdot z_2\right)\cdot z_3$
        \item Distributivität: $z_1\left(z_2+z_3\right)=z_1\cdot z_2+z_1\cdot z_3$ 
        \item Kehrwertregel: $\tfrac{1}{z_1\cdot z_2}=\tfrac{1}{z_1}\cdot \tfrac{1}{z_2}$ 
        \item Konjugation: $\overline{z_1+z_2}=\overline{z_1}+\overline{z_2}$ und $\overline{z_1\cdot z_2}=\overline{z_1}\cdot \overline{z_2}$ 
\end{enumerate}
$\left(\mathbb{C},0,1,+,\cdot \right)$ ist ein Körper, der $\mathbb{R}$ enthält.

\subsection{Komplexe Zahlen als Matrizen}
\begin{align} 
        \left[\left(x+y\text{i}\right)\left(a+b\text{i}\right)\right]_{\mathbb{R}^{2}}=\begin{pmatrix}
                xa-yb\\xb+ya
        \end{pmatrix}=\begin{pmatrix}
        x&-y\\y&x
        \end{pmatrix}\begin{pmatrix}
                a\\b
        \end{pmatrix}
.\end{align} 
Eine komplexe Zahl kann also als Matrix dargestellt werden
\begin{align} 
        \left[x+y\text{i}\right]_{\mathbb{R}^{2\times 2}}:=\begin{pmatrix}
                x&-y\\y&x
        \end{pmatrix}
.\end{align} 
Es gilt dann
\begin{align} 
        \left[z_1\cdot z_2\right]_{\mathbb{R}^2}&=\left[z_1\right]_{\mathbb{R}^{2\times 2}}\cdot \left[z_2\right]_{\mathbb{R}^{2}}\\
        \left[z_1+z_2\right]_{\mathbb{R}^2}&=\left[z_1\right]_{\mathbb{R}^{2\times 2}}+\left[z_2\right]_{\mathbb{R}^{2}}\\
        \left[\dfrac{1}{z}\right]_{\mathbb{R}^{2\times 2}}&=\left[z\right]_{\mathbb{R}^{2\times 2}}^{-1}
.\end{align} 
Also $\mathbb{C}\subseteq \mathbb{R}^{2\times 2}$, $\mathbb{C}\equiv \left\{\begin{pmatrix}
                x&-y\\y&x
\end{pmatrix} \in \mathbb{R}^{2\times 2}:x,y \in \mathbb{R}\right\}=\,\text{span}\,\left(\begin{pmatrix}
                1&0\\0&1
\end{pmatrix},\begin{pmatrix}
                0&-1\\1&0
\end{pmatrix}\right)=\left\{x \begin{pmatrix}
                1&0\\0&1
\end{pmatrix}+y \begin{pmatrix}
                0&-1\\1&0
\end{pmatrix}:x,y \in \mathbb{R}\right\}$.\\\indent
Jede Matrix $\begin{pmatrix}
        x&-y\\y&x
\end{pmatrix}$ ist das Produkt einer Drehung $\begin{pmatrix}
        \cos \theta &-\sin \theta \\\sin \theta &\cos \theta  
\end{pmatrix}$ und einer Streckung $\begin{pmatrix}
        r&0\\0&r
\end{pmatrix}$.\\
Für jede komplexe Zahl $z=x+y\text{i} \in \mathbb{C}\backslash \left\{0\right\}$, gibt es eindeutige $r>0,\theta  \in [0,2\pi):z=r\left(\cos \theta +\sin \theta \text{i}\right)$.\\\indent
Hier ist $r=\,\sqrt[]{x^2+y^2}=| |z| |$ und $\theta =\arg\left(z\right)=\begin{cases}
        \arctan \tfrac{y}{x}&;x>0,y\geq 0\\
        \tfrac{\pi }{2}&;x=0<y\\
        \pi +\arctan \tfrac{y}{x}&;x<0\\
        \tfrac{3\pi }{2}&;x=0>y\\
        2\pi +\arctan \tfrac{y}{x}&;y<0<x
\end{cases}$ 

\subsection{Polarkoordinaten}
Die komplexen Zahlen können in \textbf{Polarkoordinaten} dargestellt werden
\begin{align} 
        \mathbb{C}\ni z=re^{\text{i}\theta }\qquad e^{\text{i}\theta }=\cos \theta +\sin \theta \text{i}
,\end{align} 
mit $r>0$ und $\theta  \in [0,2\pi )$. Die Eindeutigkeit von $r$ lässt sich zeigen durch
\begin{align} 
        ||z||=\,\sqrt[]{x^2+y^2}=||re^{\text{i}\theta }||=||r||\underbrace{||e^{\text{i}\theta }||}_{=1}=r
.\end{align} 
Die Eindeutigkeit von $\theta $ lässt sich zeigen durch
\begin{align} 
        e^{\text{i}\theta _1}=e^{\text{i}\theta _2}\Leftrightarrow \theta _2-\theta _1 \in 2\pi \mathbb{Z}
.\end{align} 
Die Multiplikation ist definiert durch
\begin{align} 
        z_1\cdot z_2=r_1r_2e^{\text{i}\left(\theta _1+\theta _2\right)}
;\end{align} 
und der Kehrwert
\begin{align} 
        \dfrac{1}{e^{\text{i}\theta }}=e^{\text{i}\left(-\theta \right)}
.\end{align} 

\subsubsection{Potenzen}
Wird eine komplexe Zahl potenziert, gilt
\begin{align} 
        k \in \mathbb{Z}:z^k=r^ke^{\left(\text{i}\theta \right)^k}=r^ke^{\text{i}\left(k\theta \right)}
.\end{align} 

\subsubsection{Beispiel}
\begin{align} 
        \left(1+\text{i}\right)^{100}
.\end{align} 
In Polarkoordinaten ist $1+\text{i}$, 
\begin{align} 
        r=\,\sqrt[]{1^2+1^2}=\,\sqrt[]{2}\qquad \theta =\arg\left(1+\text{i}\right)=\arctan\left(\dfrac{1}{1}\right)=\dfrac{\pi }{4}\qquad 1+\text{i}=\,\sqrt[]{2}e^{\text{i}\tfrac{\pi }{4}}
.\end{align} 
Die Potenz ist dann
\begin{align} 
        \left(1+\text{i}\right)^{100}&=\,\sqrt[]{2}^{100}e^{\text{i}\left(100\tfrac{\pi }{4}\right)}\nonumber \\
                                     &=2^{50}e^{\text{i}25\pi }\nonumber \\
                                     &=2^{50}e^{\text{i}\pi }\nonumber \\ 
                                     &=-2^{50}
.\end{align} 

\subsection{Wurzeln}
Löse die Gleichung $z^k=\alpha  \in \mathbb{C},k \in \mathbb{N}$, mit $z=re^{\text{i}\theta }$ und $\alpha =se^{\text{i}\beta }$. Der erste Teil der Lösung ist
\begin{align} 
        r^k=s=||\alpha || \Leftrightarrow r=\,\sqrt[k]{||\alpha || }\geq 0
.\end{align} 
Es muss also gelten
\begin{align} 
        e^{\text{i}k\theta }=e^{\text{i}\beta }\Leftrightarrow k\theta -\beta  \in 2\pi \mathbb{Z}
.\end{align} 
Diese Gleichung hat $k$ Lösungen
\begin{align} 
        \theta _1=\dfrac{\beta }{k},\theta _2=\dfrac{\beta }{k}+\dfrac{1}{k}2\pi ,\theta _3=\dfrac{\beta }{k}+\dfrac{2}{k}2\pi ,\hdots ,\theta _k=\dfrac{\beta }{k}+\dfrac{k-1}{k}2\pi 
.\end{align} 

\subsubsection{Beispiel}
Löse die Gleichung $z^4=r^4e^{\text{i}4\theta }=1$. $1=1e^{\text{i}0}$, also ist $s=1$ und $\beta =0$.
\begin{align} 
        r&=\,\sqrt[4]{1}=1&\theta _1&=\dfrac{0}{4}=0\nonumber \\
         &&\theta _2&=\dfrac{0}{4}+\dfrac{1}{4}2\pi =\dfrac{\pi }{2}\nonumber \\
         &&\theta _3&=\dfrac{0}{4}+\dfrac{2}{4}2\pi =\pi \nonumber \\
         &&\theta _4&=\dfrac{0}{4}+\dfrac{3}{4}2\pi =\dfrac{3\pi }{2}\nonumber 
.\end{align} 
Also $z=e^{\text{i}0},e^{\text{i}\tfrac{\pi }{2}},e^{\text{i}\pi },e^{\text{i}\tfrac{3\pi }{2}}$.

\subsubsection{Wurzelfunktion}
Die $k$--te Wurzelfunktion ist definiert als
\begin{align} 
        \,\sqrt[k]{z}=\,\sqrt[k]{re^{\text{i}\theta }}=\,\sqrt[k]{r}e^{\text{i}\tfrac{\theta }{k}}
,\end{align} 
für $\theta  \in [0,2\pi )$. Dann ist $\,\sqrt[k]{z}$ eine Lösung der Gleichung $\alpha ^k=z$.\\\indent
Die Wurzelfunktion $\,\sqrt[]{\cdot }:\mathbb{C}\rightarrow \mathbb{C}$ ist nicht stetig, da
\begin{align} 
        \,\sqrt[]{1}=\,\sqrt[]{1e^{\text{i}0}}=\,\sqrt[]{1}e^{\text{i}\tfrac{0}{2}}&=1\\
        \lim_{\varepsilon \rightarrow 0}\,\sqrt[]{e^{\text{i}\left(2\pi -\varepsilon \right)}}=\lim_{\varepsilon \rightarrow 0}e^{\text{i}\pi -\tfrac{\varepsilon }{2}}=e^{\text{i}\pi }&=-1
.\end{align} 

\subsection{Quadratische Gleichungen}
Löse die Gleichung $z^2+pz+q=0;p,q \in \mathbb{C}$. Mit quadratischer Ergänzung
\begin{align} 
        z^2+pz+q&=z^2+pz+\dfrac{1}{4}p^2+q-\dfrac{1}{4}p^2\\
                &=\left(z+\dfrac{p}{2}\right)^2+\left(q-\dfrac{1}{4}p^2\right)
.\end{align} 
Daraus folgt
\begin{align} 
        \left(z+\dfrac{p_2}{2}\right)^2&=\dfrac{1}{4}p^2-q\\
        z&=-\dfrac{p}{2}\pm \,\sqrt[]{\dfrac{1}{4}p^2-q}\\
        z&=-\dfrac{p}{2}\pm \,\sqrt[]{\left(\dfrac{p}{2}\right)^2-q}
.\end{align} 
Diese Gleichung ist die komplexe Wurzel, sie hat also immer mindestens zwei Lösung.

\subsection{Fundamentalsatz der Algebra}
Der \textbf{fundamentalsatz der Algebra} besagt, dass jedes $k$--Polynom 
\begin{align} 
        P\left(z\right)=\sum_{j=0}^{k}\alpha _jz^j,\alpha _k\neq 0
\end{align} 
insgesamt $k$ Nullstellen hat, also
\begin{align} 
        \exists z_1,\hdots ,z_k \in \mathbb{C}\Rightarrow P\left(z\right)=\alpha _k\prod_{j=1}^{k}\left(z-z_j\right)
.\end{align} 

\newpage
\section{Komplexe Funktionen}
Eine Folge $\left(z_k\right)_{k \in \mathbb{N}}\subseteq \mathbb{C}$ konvergiert gegen einen Grenzwert, wenn
\begin{align} 
        \,\forall \varepsilon >0\exists N \in \mathbb{N}\,\forall n\geq N:|z_k-z|\leq \varepsilon 
.\end{align} 
Sei $f:U\rightarrow \mathbb{C},U\subseteq \mathbb{C}$. Für ein $z \in U$
\begin{align} 
        \lim_{h\rightarrow z}f\left(h\right)=\alpha  \in \mathbb{C}\,\text{existiert}\,
,\end{align} 
falls $\,\forall $ Folgen $\left(h_k\right)_{k \in \mathbb{N}} \subseteq U$ mit $h_k\neq z,\lim_{k\rightarrow \infty}h=z$, gilt, dass $\lim_{k\rightarrow \infty}f\left(h_k\right)=\alpha $.

\subsection{Stetigkeit}
Sei $f:U\rightarrow \mathbb{C},U\subseteq \mathbb{C}$. $f$ heißt stetig in $z \in \mathbb{C}$, falls
\begin{align} 
        \lim_{h\rightarrow z}f\left(h\right)=f\left(z\right)
.\end{align} 

\subsection{Differenzierbarkeit}
Sei $U\subseteq \mathbb{C}$ offen und $f:U\rightarrow \mathbb{C}$. $f$ heißt \textbf{komplex differenzierbar} in $z \in U$, falls
\begin{align} 
        \lim_{h\rightarrow 0}\dfrac{f\left(z+h\right)-f\left(z\right)}{h}=:f'\left(z\right)\,\text{existiert}\,
.\end{align} 
Sei $f\left(z\right)=\alpha _jz^j$ ein Polynom von Grad $k \in \mathbb{N}$, mit $\alpha _0,\hdots ,\alpha _k \in \mathbb{C},\alpha _k\neq 0$. Dann ist
\begin{align} 
        f'\left(z\right)=j\alpha _jz^{j-1}
.\end{align} 
$f:\mathbb{R}^n\rightarrow \mathbb{R}^m$ ist total differenzierbar in $\vv{x} \in \mathbb{R}^n$ mit $Df \in \,\text{Lin}\,\left(\mathbb{R}^n,\mathbb{R}^m\right)=\mathbb{R}^{m\times n}$, falls
\begin{align} 
        \lim_{\vv{h}\rightarrow \vv{0} \in \mathbb{R}^n}\dfrac{f\left(\vv{x}+\vv{h}\right)-f\left(\vv{x}\right)-Df_{\left(x\right)}\vv{h}}{||\vv{h}|| }=\vv{0} \in \mathbb{R}^m \,\text{existiert}\,
.\end{align} 
Sei $U\subseteq \mathbb{C}$ offen und $f:U\rightarrow \mathbb{C}$. Dann gilt, dass $f'\left(z\right)$ in $z=x+\text{i}y$ existiert, genau dann wenn
\begin{align} 
        Df \begin{pmatrix}
                x\\y
        \end{pmatrix}\,\text{existiert und}\,Df \begin{pmatrix}
                x\\y
        \end{pmatrix}=\left[f'\left(z\right)\right]_{\mathbb{R}^{2\times 2}}
.\end{align} 
Falls $f'\left(z\right)=a+\text{i}b$, dann ist $Df \begin{pmatrix}
        x\\y
\end{pmatrix}=\begin{pmatrix}
a&-b\\b&a
\end{pmatrix}$. $F$ ist genau dann differenzierbar, wenn $\partial _1F_1=\partial _2F_2$ und $\partial _1F_2=\partial _2F_1$.

\subsubsection{Holomorph}
Sei $U\subseteq \mathbb{C}$ offen. $f:U\rightarrow \mathbb{C}$ heißt \textbf{holomorph} auf $U$ wenn $f'\left(z\right) \in \mathbb{C}\,\forall z \in U$ existiert und in jedem Punkt auf $U$ stetig komplex differenzierbar ist.

\subsection{Differentiationsregeln}
Sei $f:U\rightarrow \mathbb{C}$ in $z$ komplex differenzierbar, dann ist $f$ in $z$ stetig.\\\indent
Sei $U\subseteq \mathbb{C}$ offen, $f:U\rightarrow \mathbb{C}$ holomorph.
\begin{enumerate}[label=\roman*)]
        \item $\left(\alpha f\right)'\left(z\right)=\alpha f'\left(z\right),\alpha  \in \mathbb{C}$ 
        \item $\left(f_1+f_2\right)'\left(z\right)=f_1'\left(z\right)+f_2'\left(z\right)$ 
        \item $\left(f_1f_2\right)'\left(z\right)=f_1'\left(z\right)f_2\left(z\right)+f_1\left(z\right)f_2'\left(z\right)$ 
        \item $\left(\tfrac{f_1}{f_2}\right)'\left(z\right)=\tfrac{f_1'\left(z\right)f_2\left(z\right)-f_1\left(z\right)f_2'\left(z\right)}{f_2^2\left(z\right)},f_2\left(z\right)\neq 0$ 
\end{enumerate}
Sei $f:U\rightarrow V\subseteq \mathbb{C}$ offen, $g:V\rightarrow \mathbb{C}$ 
\begin{enumerate}[label=]
        \item[v)] $\left(g\circ f\right)'\left(z\right)=g'\left(f\left(z\right)\right)f'\left(z\right)$ 
\end{enumerate}

\newpage
\section{Reihen}
Sei $\left(z_k\right)_{k \in \mathbb{N}}\subseteq \mathbb{C}$ eine Folge. Die Reihe 
\begin{align} 
        \sum_{k=0}^{\infty}z_k
\end{align} 
\textbf{konvergiert}, falls die Partialsummen
\begin{align} 
        S_n:=\sum_{k=0}^{n}z_k
\end{align} 
in $\mathbb{C}$ gegen ein $z \in \mathbb{C}$ konvergieren. Die Reihe konvergiert absolut, falls
\begin{align} 
        \sum_{k=0}^{n}|z_k|
\end{align} 
konvergiert. Die Reihe \textbf{divergiert}, wenn sie nicht konvergiert.\\\indent
Jede \textsc{Cauchy}--Folge konvergiert in $\mathbb{C}$. $\mathbb{C}$ ist also \textbf{metrisch vollständig}.

\subsection{Potenzreihen}
Sei $\left(\alpha _k\right)_{k \in \mathbb{N}}$ eine komplexe Folge. Man definiert die \textbf{Potenzreihe} mit Koeffizienten $\alpha _k$ als
\begin{align} 
        f\left(z\right):=\sum_{k=0}^{\infty}\alpha _kz^k
.\end{align} 
Man definiert 
\begin{align} 
        R&:=\dfrac{1}{\lim_{k\rightarrow \infty}\text{sup}\,\sqrt[k]{|\alpha _k|}}&R&:=\dfrac{1}{\lim_{k\rightarrow \infty}\left|\tfrac{\alpha _k}{\alpha _{k+1}}\right|} \in \left[0,+\infty\right]
,\end{align} 
mit $0^{-1}:=+\infty$ und $\infty^{-1}:=0$. Dann
\begin{enumerate}[label=\roman*)]
        \item $\sum_{k=0}^{\infty}\alpha _k|z|^k$ konvergiert absolut für $|z|<R$.
        \item $\sum_{k=0}^{\infty}\alpha _kz^k$ divergiert für $|z|>R$.
\end{enumerate}
Sei $f\left(z\right)$ eine Potenzreihe $f\left(z\right):=\sum_{k=0}^{\infty}\alpha _kz^k$ mit Konvergenzradius $R \in \left[0,\infty\right]$. Man definiert $g\left(Z\right):=\sum_{k=1}^{\infty}k\alpha _kz^{k-1}$, dann hat $g$ den Konvergenzradius $R$ und $f'\left(z\right)=g\left(z\right)$, falls $|z|<R$.\\\indent
$f$ ist auf $\left\{z \in \mathbb{C}:|z|<R\right\}$ unendlich oft differenzierbar und
\begin{align} 
        f^{\left(n\right)}:=\sum_{k=n}^{\infty}\dfrac{k!}{\left(k-n\right)!}\alpha _kz^{k-n}
\end{align} 
mit Konvergenzradius $R$. Die Stammfunktion ist dann
\begin{align} 
        F\left(z\right):=\sum_{k=0}^{\infty}\dfrac{\alpha _k}{k+1}z^{k+1}+c\qquad c  \in \mathbb{C}
.\end{align} 

\subsubsection{Produkt von Potenzreihen}
Seien zwei Potenzreihen $f\left(z\right):=\sum_{k=0}^{\infty}\alpha _kz^k$ und $g\left(z\right):=\sum_{j=0}^{\infty}\beta _jz^j$. Das Produkt ist
\begin{align} 
        f\left(z\right)g\left(z\right)&=\left(\alpha _0+\alpha _1z+\hdots \right)\left(\beta _0+\beta _1z+\hdots \right)\\
                                      &=\alpha _0\beta _0+\left(\alpha _0\beta _1+\alpha _1\beta _0\right)z+\left(\alpha _0\beta _2+\alpha _1\beta _1+\alpha _2\beta _0\right)z^2+\hdots 
.\end{align} 
Die Konvergenzradien der Porenzreihen seien $R$ und $R' \in \left[0,\infty\right]$. Man definiert $h\left(z\right):=\sum_{n=0}^{\infty}\left(\sum_{k=0}^{\infty}\alpha _k\beta _{n-k}\right)z^n$. Dann hat $h$ einen Konvergenzradius von mindesten $R''\geq \,\text{min}\,\left(R,R'\right)$ und $f\left(z\right)g\left(z\right)=h\left(z\right)$ für $|z|<\,\text{min}\,\left(R,R'\right)$.

\subsubsection{Verschiebung}
Sei die Potenzreihe $f\left(z\right):=\sum_{}^{}\alpha _kz^k$. Die um $z_0$ verschobene Potenzreihe ist $g\left(h\right)=f\left(z_0+h\right),|z_0|<R$. Sei $|z_0|<R$ und 
\begin{align} 
        g\left(h\right):=\sum_{n=0}^{\infty}\dfrac{f^{\left(n\right)}\left(z_0\right)}{n!}h^n
,\end{align} 
dann ist der Konvergenzradius mindestens $R'\geq R-|z_0|$ und $g\left(h\right)=f\left(z_0+h\right)$, falls $|z_0|+|h|<R$.

\subsubsection{Vertauschnug}
Falls $\sum_{k=0}^{\infty}\sum_{n=0}^{\infty}|c _{kn}|<\infty$, dann dürfen die Summen vertauscht werden
\begin{align} 
        \sum_{k=0}^{\infty}\sum_{n=0}^{\infty}c _{kn}=\sum_{n=0}^{\infty}\sum_{k=0}^{\infty}c _{kn}
.\end{align} 

\subsection{Exponentialfunktion}
Die Exponentialfunktion hat eine Reihendarstellung
\begin{align} 
        \text{exp}\left(z\right)&:=\sum_{k=0}^{\infty}\dfrac{1}{k!}z^k
.\end{align} 
Die Ableitung der Exponentialfunktion ist wieder die Exponentialfunktion
\begin{align} 
        \text{exp}'\left(z\right)&=\sum_{k=0}^{\infty}k\dfrac{1}{k!}z^{k-1}\\
                                     &=\sum_{k=1}^{\infty}\dfrac{1}{\left(k-1\right)!}z^{k-1}\\
                                     &=\sum_{k=0}^{\infty}\dfrac{1}{k!}z^k\\
                                     &=\text{exp}\left(z\right)
.\end{align} 
Eine weitere wichtige Eigenschaft ist 
\begin{align} 
        \text{exp}\left(z_1+z_2\right)=\text{exp}\left(z_1\right)\text{exp}\left(z_2\right)
.\end{align} 
Sei $y \in \mathbb{R}$, dann ist
\begin{align} 
        |\text{exp}\left(y\text{i}\right)|=1
.\end{align} 
Wird $y$ verändert, dreht sich also die Exponentialfunktion um den Einheitskreis. Wird der Realteil ($x$) verändert, ändert sich der Betrag der Funktion.\par
Die Exponentialfunktion lässt sich auch durch andere Funktionen darstellen
\begin{align} 
        \text{exp}\left(z\right)&=\cosh \left(z\right)+\sinh \left(z\right)\\
        \text{exp}\left(\text{i}z\right)&=\cos \left(z\right)+\text{i}\sin \left(z\right)
.\end{align} 
Diese Relation lässt sich durch die Reihendarstellung von $\sin $ und $\cos $ zeigen.\par
Die Exponentialfunktion wird niemals 0 und ist surjektiv auf $\mathbb{C}\backslash \left\{0\right\}$, aber nicht injektiv.

\subsection{Logarithmus}
Auf dem Streifen $\mathbb{R}+[0,2\pi )\text{i}\subseteq \mathbb{C}$ ist $\text{exp}:\mathbb{R}+[0,2\pi )\text{i}\rightarrow \mathbb{C}\backslash \left\{0\right\}$ bijektiv. Die Umkehrfunktion nennt man den \textbf{Hauptzweig des komplexen Logarithmus}.
\begin{align} 
        \text{Ln}&:\mathbb{C}\backslash \left\{0\right\}\rightarrow \mathbb{R}+[0,2\pi )\text{i}\\
        \text{Ln}\left(z'\right)&:=\ln\left(|z'|\right)+\arg\left(z'\right)\text{i}
.\end{align} 
$\text{Ln}:\mathbb{C}\backslash \left\{0\right\}\rightarrow \mathbb{R}+[0,2\pi )\text{i}$ ist holomorph auf $\mathbb{C}\backslash [0,\infty)$ mit $\text{Ln}'=\tfrac{1}{z}$.

\subsection{Umkehrfunktionen}
Es sind
\begin{align} 
        \,\text{Polarkoordinaten}\,&:F\left(r,\theta \right)=r\text{e}^{\theta \text{i}}\\
        \,\text{inv.\ Polarkoordinaten}\,&:F^{-1}\left(z\right)=\left(|z|,\arg\left(z\right)\right)\\
        \,\text{Exponentialfunktion}\,&:\text{exp}\left(x+y\text{i}\right)=\text{e}^{x}\text{e}^{y\text{i}}\\
        \,\text{Hauptzweig des Logarithmus}\,&:\text{Ln}\left(z\right)=\ln\left(|z|\right)+\arg\left(z\right)\text{i}
.\end{align} 

\newpage
\section{Kurvenintegrale}
Eine \textbf{Kurve} ist eine Abbildung
\begin{align} 
        \gamma :\left[a,b\right]\rightarrow \mathbb{C}
,\end{align} 
mit $a\leq b \in \mathbb{R}$ und $\gamma $ stetig. Dann ist 
\begin{align} 
        \dot{\gamma }\left(t\right)=\left(\mathcal{R}\gamma \right)'\left(t\right)+\left(\mathcal{I}\gamma \right)'\text{i} \in \mathbb{C}
.\end{align} 
die Geschwindigkeit.\par
Man definiert dann das \textbf{Kurvenintegral} von $f$ entlang $\gamma $ als
\begin{align} 
        \int_{\gamma }^{}f\left(z\right)\td z&:=\int_{a}^{b}f\left(\gamma \left(t\right)\right)\dot{\gamma }\left(t\right)\td t
\end{align} 


%}}}

\end{document}
