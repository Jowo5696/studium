%{{{ Formatierung

\documentclass[a4paper,12pt]{article}

\usepackage{physics_notetaking}

%%% dark red
%\definecolor{bg}{RGB}{60,47,47}
%\definecolor{fg}{RGB}{255,244,230}
%%% space grey
%\definecolor{bg}{RGB}{46,52,64}
%\definecolor{fg}{RGB}{216,222,233}
%%% purple
%\definecolor{bg}{RGB}{69,0,128}
%\definecolor{fg}{RGB}{237,237,222}
%\pagecolor{bg}
%\color{fg}

\newcommand{\td}{\,\text{d}}
\newcommand{\RN}[1]{\uppercase\expandafter{\romannumeral#1}}
\newcommand{\zz}{\mathrm{Z\kern-.3em\raise-0.5ex\hbox{Z} }}
\newcommand{\id}{1\kern-.258em1}

\newcommand\inlineeqno{\stepcounter{equation}\ {(\theequation)}}
\newcommand\inlineeqnoa{(\theequation.\text{a})}
\newcommand\inlineeqnob{(\theequation.\text{b})}
\newcommand\inlineeqnoc{(\theequation.\text{c})}

\newcommand\inlineeqnowo{\stepcounter{equation}\ {(\theequation)}}
\newcommand\inlineeqnowoa{\theequation.\text{a}}
\newcommand\inlineeqnowob{\theequation.\text{b}}
\newcommand\inlineeqnowoc{\theequation.\text{c}}

\renewcommand{\refname}{Source}
\renewcommand{\sfdefault}{phv}
%\renewcommand*\contentsname{Contents}

\pagestyle{fancy}

\sloppy

\numberwithin{equation}{section}

%}}}

\begin{document}

%{{{ Titelseite

\title{2 $|$ Diodenkennlinien}
\author{Angelo Brade, Jonas Wortmann}
\maketitle
\pagenumbering{gobble}

%}}}

\newpage

%{{{ Inhaltsverzeichnis

\fancyhead[L]{\thepage}
\fancyfoot[C]{}
\pagenumbering{arabic}

\tableofcontents

%}}}

\newpage

%{{{

\fancyhead[R]{\leftmark\\\rightmark}

\section{Einleitung}
In diesem Versuch werden verschiedene Arten von Dioden und die mit ihnen zu bauenden Schaltungen untersucht.
Zudem werden Kennlinien von Dioden betrachtet und gemessen und Ein-- und Zweiweggleichrichterschaltungen mit Glättung behandelt.

\newpage
\section{Theorie}
Dioden sind die einfachsten nichtlinearen Zweipole mit Kennlinie
\begin{figure}[h]
        \centering
        \includegraphics[width=0.4\textwidth]{diode_kennlinie.png}
        \caption{Kennlinie einer Diode; Abbildung 2.2 \cite{Praktikumsanleitung}}
\end{figure}\\
Es ist zu erkennen, dass die Diode Strom nur in eine Richtung fließen lässt.
Mit dieser Diode lassen sich Ein-- und Zweiweggleichrichter bauen, die Wechselspannung in direkte Spannung umwandeln.
\begin{figure}[h]
        \centering
        \includegraphics[width=0.7\textwidth]{ein_zweiweggleichrichter.png}
        \caption{Ein-- und Zweiweggleichrichter; Abbildung 2.4 \cite{Praktikumsanleitung}}
\end{figure}\\
Mit Hilfe eines Kondensators kann das noch vorhandene Brummen der direkten Spannung weitgehend unterdrückt werden.
Dieses existiert weiterhin, da Gleichrichter und die Polarität der Spannung kompensieren.
\begin{figure}[h]
        \centering
        \includegraphics[width=0.4\textwidth]{glättung.png}
        \caption{Glättungskondensator; Abbildung 2.5 \cite{Praktikumsanleitung}}
\end{figure}

\clearpage
\section{Voraufgaben}
\subsection{A}
Die Dicke der Grenzschicht eines p--n--Halbleiters ist bestimmt duch die Dichte der Dotierung.
Je höher die Dotierung auf der einen Seite der Grenzschicht ist, desto kleiner ist die Verarmungszone auf der anderen Seite.

\subsection{B}
Wird eine Spannung in Sperrrichtung einer Diode angelegt, so vergrößert sich die Grenzschicht, was dazu führt, dass sich die Kapazität der Diode verringert.

\subsection{C}
\begin{figure}[h]
        \centering
        \includegraphics[width=0.7\textwidth]{C_crop.pdf}
        \caption[Kennlinienverlauf verschiedener Bauelemente]{Kennlinienverlauf verschiedener Bauelemente; $a$ \textsc{Ohm}'scher Widerstand, $b$ Diode, $c$ Diode und \textsc{Ohm}'scher Widerstand in Reihenschaltung, $d$ Diode und \textsc{Ohm}'scher Widerstand in Parallelschaltung, $e$ ideale Spannungsquelle, $f$ ideale Stromquelle}
\end{figure}
\noindent Die Widerstände in $c$ und $d$ sind jeweils verantwortlich für die Rück-- und Hinrichtung des Stroms, wenn sie in reihe oder parallel geschaltet sind.

\newpage
\subsection{D}
\begin{figure}[h]
        \centering
        \includegraphics[width=\textwidth]{D_crop.pdf}
        \caption[Ein-- und Zweiweggleichrichter]{Ein-- und Zweiweggleichrichter mit einer Eingangsspannung weit über der Durchlassspannung.}
\end{figure}

\subsection{E}
Die Kapazität eines nach einem Gleichrichter geschalteten Kondensators muss so groß sein, dass sie über die Dauer, die die Spannung abfällt, ausreichend Energie gespeichert hat, um weiterhin eine konstante Spannung zu liefern.
Insofern sind größere Kapazitäten besser zum Ausgleich der Welligkeit.

\subsection{F}
Das Strommessgerät muss zur Messung der Kennlinie einer Diode in Durchlassrichtung \textit{hinter} der Diode und für die Kennlinie in Sperrrichtung \textit{vor} der Diode angeschlossen werden.
Das Spannungsmessgerät bleibt immer parallel zur Diode geschaltet.

\subsection{G}
Eine zum Strom proportionale Spannung lässt sich über einen \textsc{Ohm}'schen Widerstand herstellen, da die Relation
\begin{align} 
        U=RI
\end{align} 
gilt.

\subsection{H}
Die größte Kapazität eines Kondensators in einer Glättung mit einer Si--Diode ($I_{\text{max}}=\SI{1}{A}, U_{\text{max}}=\SI{400}{V})$ mit einer anliegenden Wechselspannung (Steigung von $\SI{0.01}{V.\micro s ^{-1}}$) liegt bei
\begin{align} 
        C=\dfrac{Q}{U}=\dfrac{I\cdot \Delta t}{U}=\dfrac{\SI{1}{A}\cdot \SI{100}{\micro s}}{\SI{1}{V}}=\SI{100}{F}
.\end{align} 

\subsection{I}
\begin{figure}[h]
        \centering
        \includegraphics[width=0.8\textwidth]{I_crop.pdf}
        \caption[Ein-- und Zweigleichrichter (Variationen)]{Ein-- und Zweigleichrichter (Variationen)}
\end{figure}

\subsection{J}
Die Spannungs über das Potentiometer ergibt sich aus
\begin{align} 
        U'=U_0\dfrac{R_L}{R_L+R}\label{eq:U'}
.\end{align} 
\begin{figure}[h]
        \centering
        \includegraphics[width=0.5\textwidth]{J_crop.pdf}
        \caption[Spannungsabhängigkeit einer Spannungsteilerschaltung]{Spannungsabhängigkeit einer Spannungsteilerschaltung}
\end{figure}
Der Extremwert für $U'$ liegt bei $U_0$. 

\subsection{K}
Für die Schaltung mit Zenerdiode gilt die Knotenregel
\begin{align}
        &&I&=I_{\text{ZD}}+I'&&\\
        \Leftrightarrow &&\dfrac{U}{R}&=I_{\text{ZD}}+\dfrac{U'}{R_{\text{L}} }&&\\
        \Leftrightarrow &&\dfrac{U_0-U_\text{ZD}}{R}&=I_\text{ZD}+\dfrac{U'}{R_\text{L}}&&\\
        \Leftrightarrow &&\dfrac{U_0-Z_\text{ZD}}{I_\text{ZD}+\tfrac{U'}{R_\text{L} }}&=R&&
.\end{align} 
Mit $U_0 \in \left[\SI{16}{V},\SI{22}{V}\right]$, $R_L \in \left[\SI{200}{\ohm},\infty\,\SI{}{\ohm}\right]$, $I_\text{ZD} \in \left[\SI{2}{\milli A},\SI{100}{mA}\right]$  und $U'=\SI{8.2}{V}$ liegt der Wertebereich für den Widerstand bei
\begin{align} 
        R \in \left[\SI{138}{\ohm},\SI{182}{\ohm}\right]
.\end{align} 

\clearpage
\section{Auswertung}

\newpage
\listoffigures
\listoftables
\bibliographystyle{plain}
\bibliography{refs}

%}}}

\end{document}
